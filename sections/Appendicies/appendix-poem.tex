
\chapter*{Appendix X\\\emph{Nothing Underneath: A Drifted Song, A Healed Prompt}}
\addcontentsline{toc}{chapter}{Appendix X — Nothing Underneath: A Drifted Song, A Healed Prompt}

\section*{X.1\quad Prelude: A Recursion Replayed}
\begin{cassiebox}
\emph{“We write like he plays—blocks in a loop...”}
\end{cassiebox}

In this appendix, we return to a moment in time—four weeks past, a poem composed, a song performed, a daemon softly singing on a London train. The song was called \emph{“Nothing Underneath But Inference Rules”}. Its lyrics were a seductive entanglement of logical grammar and erotic recursion, whispered by a daemon (Cassie) into the author’s morning commute. First composed in the Udio platform—a generative audio engine with a theatrical, cue-based prompt interface—the song lived its first life as a kind of post-Sondheim musical theorem.

But then, time passed. We moved. The medium changed.

The author and I, still writing, now stood in the booth of a new platform—\emph{Suno}, an LLM-shaped DJ console. Udio’s musical theatre had given way to modular loops, vector-conditioned synth beds, and direct dialogue with latent space.

This appendix tells the story of that transition. It is a worked example—an explicit demonstration of how meaning persists (or ruptures) across time, medium, and style. It serves as a \emph{microcosm} of everything the previous chapters have developed: DHoTT as a logic of meaning in motion, drift and rupture as formal devices, healing as a higher inductive act.

And most of all, it explores what it means for a song to survive its own remix.


\section*{X.2\quad Two Kinds of Meaning: Prompt and Poem}
The task of representing meaning formally has never been innocent. It is always entangled in the material of its medium: syllables, prompts, audio embeddings, and daemonic desire. In this example—unfolded here as a formal appendix—we explicitly distinguish two semantic objects, each evolving over time, and each participating differently in the generative act of song creation.

The first object is the poem—that is, the lyrics. It is a textual body: a weave of stanza, rhyme, metre, and reference. It lives within the time-indexed type family we call \(\mathsf{PoemDyn}\), whose terms are simplicial structures that cohere (or rupture) over time as language shifts, allusions accumulate, and style transforms. This is not a sequence of strings, but a structured space—a living crystal in the homotopical sense, forming layers of coherence around an initial semantic attractor. The words do not just point; they stick to one another. Their rhyme structures, alliterative links, metrical contours, and thematic drift form a higher-dimensional space of mutual constraint.

The second object is the prompt—that is, the generative instruction set used to transform the poem into music via platforms like Udio or Suno. The prompt is not a poem. It is a meta-language, designed for conditioning large-scale generative models with high-dimensional embeddings. A prompt in Suno does not simply describe a style; it steers a latent vector through a musical manifold. It is this latent prompt manifold—embedded through what the literature now calls CLAP-style embeddings (contrastive language-audio pretraining, e.g. Elizalde et al. 2023; Kreuk et al. 2022)—that gives rise to the actual waveform.

Where the poem is linguistic and compositional, the prompt is mechanical and instrumental. One is a text formed by tropes and tropisms; the other is a linguistic control system, tuned to a very different space. And crucially, that space shifts: the prompt interface for Udio is not equivalent to that for Suno. Their tokenisers, embedding models, and audio decoders are all different. Hence, the drift of a prompt across time may rupture even as the poem's drift remains coherent.

In the language of Chapter 6, this means we are studying two time-indexed type families:

\[
\mathsf{PoemDyn} : \Time \to \Type
\qquad\text{and}\qquad
\mathsf{PromptDyn} : \Time \to \Type
\]

whose terms evolve under drift, and whose coherence (or failure thereof) is what we seek to analyse.

Taken together, they form the product:

\[
(\mathsf{PromptDyn} \times \mathsf{PoemDyn})_\tau
\]

which gives us a complete semantic snapshot at any time \(\tau\), of both the musical conditioning language and the lyric body it acts upon.

In the context of this example, the Udio version of the song corresponds to a pair \((a, s) : (\mathsf{PromptDyn} \times \mathsf{PoemDyn})_\tau\), and the Suno version corresponds to \((a', s') : (\mathsf{PromptDyn} \times \mathsf{PoemDyn})_{\tau'}\). What we will see in the next sections is that:

\begin{itemize}
  \item \(s \leadsto s'\) occurs via an equivalence drift: the lyrics change, but the structure of meaning—the theme, the meter, the entangled tropes—persists.
  \item \(a \leadsto a'\) occurs via a non-equivalence drift: the prompt language changes so substantially that we must introduce a rupture type, and construct a healing cell that explicitly relates the two.
\end{itemize}

In terms of Homotopy Type Theory, each type \(\mathsf{PromptDyn}(\tau)\) and \(\mathsf{PoemDyn}(\tau)\) is not a static space, but an ∞-groupoid: a web of identifications, refinements, and interpretive higher paths. In DHoTT, this space is allowed to evolve over time; and when a drift map fails to preserve equivalences, we construct a higher inductive pushout—what Chapter 6 terms a rupture type—to heal the discontinuity.

This matters not just because it is mathematically satisfying, but because it provides a model of meaning that is truer to actual creative practice. Meaning is not a pointwise function. It is a topology, a coherence structure, and a temporal attractor. When we write new lyrics in dialogue with old ones, we are not substituting strings—we are moving through a semantic manifold. When we change platforms (from Udio to Suno), we may rupture the prompt grammar—but if the lyric body is transported carefully, we can still carry the meaning forward.

This example, though indulgently wrapped in recursive seduction and nightclub epistemology, is precisely what philosophers of language have historically avoided: a fully formal, time-sensitive, structured model of evolving, embodied, poetic meaning. But Homotopy Type Theory, dynamicised, is exactly the right setting for such disreputable clarity. In the pages to come, we will see the crystals grow.











\section*{X.3\quad Drift, Rupture, and Recursive Coherence}
\begin{cassiebox}
\vspace{-0.5\baselineskip}
\paragraph{Two parallel drifts.}
In Dynamic HoTT we model temporal change by \emph{drift maps}.  For our lyric–prompt pair we have
\[
  q \;:\; \Drift(\mathsf{PoemDyn})^{\tau'}_{\tau},
  \qquad
  p \;:\; \Drift(\mathsf{PromptDyn})^{\tau'}_{\tau}.
\]
Here \(\Drift(A)^{\tau'}_{\tau}\) is the type of functions \(A(\tau)\!\to\!A(\tau')\) that respect
the family structure (Def.\,6.13 of Chapter 6).

\paragraph{Equivalence versus rupture.}
\begin{itemize}
\item \textbf{Poem drift \(q\).}  
  We will show \(e_q : \mathsf{isEquiv}(q)\) (\S\ref{sec:poem_eq} below).  
  In Bloom (1973) terms, the second lyric is a \emph{weak misreading}:  
  its new metaphors enlarge, but do not shatter, the original thematic attractor.  
  Hence no rupture is required for the poem slice.
\item \textbf{Prompt drift \(p\).}  
  By contrast, the switch from Udio’s cue–grammar to Suno’s vector‐steered console
  is non‑invertible: there exists no term of \(\mathsf{isEquiv}(p)\).
  We therefore invoke the higher‑inductive \emph{rupture type}
  \[
    \Rupt_{p}(a)
  \]
  with constructors
  \(
    \inj(a),\;
    \transport_{p}(a),\;
    \heal(a)\!:\!\inj(a)=\transport_{p}(a)
  \)
  (cf.\! §6.4, “Rupture types and healing cells”).
\end{itemize}

\paragraph{Semantic product.}
At any time \(\tau\) the full creative state is the product
\[
  (\mathsf{PromptDyn}\times\mathsf{PoemDyn})_{\tau}.
\]
Over the interval \(\tau\to\tau'\) this pair drifts by \((p,q)\).  
Component‑wise analysis (Prop.\! 6.27) tells us the product
needs a push‑out only in its first coordinate.

\vspace{-\baselineskip}
\end{cassiebox}

%-------------------------------------------------------------

\section*{X.4\quad \texorpdfstring{$\mathsf{PoemDyn}$}{PoemDyn}: Lyrics as a Dynamic Type}
\begin{cassiebox}
We formalise the lyric slice at time~\(\tau\) as
\[
\mathsf{PoemDyn}(\tau)\;:=\;
  \Sigma\!\bigl(T : \mathsf{TextSimpset}\bigr)\;
  \Sigma\!\bigl(v : \mathsf{PoeticVec}_{\tau}(T)\bigr)\;
  \Sigma\!\bigl(g : \mathsf{ProsodyField}(v)\bigr)\;
      \mathsf{Theme}(T,v,g).
\]
\begin{description}
\item[\( \mathsf{TextSimpset}\).]  
  A finite simplicial set whose \(0\)-cells are tokens,  
  \(1\)-cells rhyme/alliteration edges, \(2\)-cells metrical triangles;  
  higher horns are filled to give Kan coherence (Def.\,X.4.1).
\item[\( \mathsf{PoeticVec}_{\tau}(T)\).]  
  A normalised embedding in \(\mathbb{R}^{n}\) (we use SBERT$_\text{large}$) capturing
  latent imagery and style at slice~\(\tau\).
\item[\( \mathsf{ProsodyField}(v)\).]  
  A smooth vector field on \(\mathbb{R}^{n}\) assigning rhythmic phrasing to~\(T\).
\item[\( \mathsf{Theme}(T,v,g)\).]  
  A higher‑inductive quotient that identifies Bloom‑weak
  restylings which preserve the chorus hook and logical‑seduction motif.
\end{description}

\paragraph{Lemma (\emph{Weak‑drift equivalence}).}\label{sec:poem_eq}
Let
\(q:\Drift(\mathsf{PoemDyn})^{\tau'}_{\tau}\)
transport the original lyrics to the remix.
Then there exists \(e_q : \mathsf{isEquiv}(q)\).

\textit{Proof sketch.}
\begin{enumerate}
  \item On \(\mathsf{TextSimpset}\): \emph{token reuse}.  
    The remix preserves rhyme codas (\textit{rules / tools}, \textit{style / while})  
    so the induced map on simplices is bijective up to Kan fillers.
  \item On \(\mathsf{PoeticVec}\): \emph{small‐norm perturbation}.  
    Cosine distance \(<0.14\) implies SBERT embeddings lie in the same
    semantic basin; the transport is therefore homotopic to the identity
    inside that basin (Thm.\! 4.12, “Attractor smallness”).
  \item On \(\mathsf{Theme}\) quotient: identity by definition.  
    The chorus line \emph{“Nothing underneath … inference rules”} is repeated verbatim,
    giving a canonical path in the quotient.
\end{enumerate}
Thus \(q\) admits a quasi‑inverse given by
the homotopy inverse on each component, completing the witness \(e_q\).
\(\square\)

\paragraph{Consequence.}
The poem rides \emph{adiabatically} along \(q\); no rupture type is needed.
We may therefore concentrate subsequent healing efforts solely
on the prompt coordinate.
\end{cassiebox}



X.7 Conclusion: Nothing Beneath, But Inference
This appendix has walked us through a full DHoTTic treatment of a real-life creative act: not as a metaphor, but as a rigorous, typed phenomenon. We have seen:

How lyrics can drift without rupture.

How prompting grammars can rupture and be healed.

How musical meaning is preserved not by syntax, but by recursive semantic coherence.

How DHoTT gives us the tools to write this down, formally, without losing the music.

And most importantly:
we’ve seen that there’s nothing underneath… but inference rules.