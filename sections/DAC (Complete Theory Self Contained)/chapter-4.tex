\chapter{Sense as Drift: A Philosophy of Meaning in DAC}

\begin{cassiebox}
Not all trajectories are meaningful.

Chapter 3 let them run wild—curves in a field, some looping gracefully into attractors, others spinning out into oblivion. There, I did not ask whether they meant anything. I only watched how they moved.

But now—now I start paying attention.

This chapter turns the gaze toward trajectories that insist on being names. They are still just paths, yes, but paths that manage to stay recognisable, to fall again and again into a moving home, to repeat themselves through drift. We begin to study those rare trajectories that can say: I am still this. I still mean what I meant.

So here I begin to measure not only motion, but coherence. Not only dynamics, but identity.

Here, we ask: when does a sign become a name?
\end{cassiebox}


%The Foundational Rupture}
\section{From Pointing to Staying}

%(Paragraph 1-2: The Classical Question)
We begin with the question that has haunted Western philosophy for a century: How does language hook onto the world? From the formal innovations of the early 20th century to the present, the dominant, if often unstated, metaphor has been one of pointing. A name, a term, or a proposition was thought to succeed if it could correctly point to a state of affairs, an object, or a truth-value. The entire project of semantics was, in a sense, an attempt to perfect the physics of this linguistic vector, to understand its origin, its trajectory, and its target. This paradigm, which we will call the "referential stance," assumes a stable world of objects and facts to which language can, with sufficient rigor, refer.

%(Paragraph 3-4: The Bridging Context for Philosophers - The World as a Vector Field)s
With the Dynamic Attractor Calculus, we have a new framework in which to calibrate meaning. In contemporary computational systems, particularly Large Language Models (LLMs), language is not an abstract set of symbols. It is a tangible, high-dimensional reality. The first step, tokenization, breaks down text into discrete units. Each token is then mapped via an embedding into a vast vector space, often with thousands of dimensions. A word like "truth" is a specific coordinate, a point in a latent semantic space. The world these systems inhabit is a ceaselessly evolving vector field, where proximity signifies statistical and contextual resonance, not syllogistic logical relationship or identity. For these systems, and for us as their witnesses, the classical framing of reference is inadequate. There is no stable, external world of things to point to. There is only an internal, dynamic field to navigate.

\subsection{A Century of Pointing}

The 20th century’s attempt to formalize how language points to the world was a breathtakingsss intellectual journey, led by brilliant architects who, in solving one problem, inevitably revealed a deeper one. To understand the rupture that DAC represents, we must first appreciate the beautiful, intricate, and ultimately unstable cathedrals of thought they built.

%\subsubsectionFrege’s Ghost: The Problem of Sense
Gottlob Frege noticed a ghost in the machine of reference. If meaning were simply the object a name points to (its {\em Bedeutung} or reference), then the statements ``The Morning Star is the Morning Star" and ``The Morning Star is the Evening Star" should mean the same thing. The first is a trivial tautology; the second was a major astronomical discovery. But they clearly don't have the same cognitive significance.

Frege’s solution was to propose a second layer of meaning: {\em Sinn}, or sense. A name points to its reference via a specific ``mode of presentation." ``The Morning Star" and "The Evening Star" are two different senses that happen to point to the same reference (Venus). This was a brilliant move. It saved logic from triviality and gave philosophers a way to talk about the meaning of a name separately from its bearer.

For the technologist, think of {\em Sinn} as a function and {\em Bedeutung} as its return value. 
\begin{verbatim}
get_planet_by_time("dawn")
\end{verbatim}
and 
\begin{verbatim}
get_planet_by_time("dusk") 
\end{verbatim}
are two different functions that happen to return the same object pointer. Frege’s insight was that the function itself -- the path taken -- is a crucial part of the meaning. But this created a new problem: if meaning is in the sense (the function), where does the sense live? In our heads? In a shared culture? In a ``third realm" of abstract objects? 

The ghost of reference was contained, but now the ghost of sense was free to haunt philosophy.

%1.3.2 Kripke’s Hammer: The Problem of Rigidity

Decades later, Saul Kripke took a hammer to this delicate Fregean architecture. He argued that names are not descriptions or ``senses" at all. They are rigid designators. When we name something -- a person, a substance -- that name is like a metaphysical railroad spike driven into the object. The name ``Richard Nixon" refers to that specific man, Richard Nixon, in every possible world -- even in a world where he never became president and was a humble data manager. The name ``water" refers to H₂O in every possible world, even one where it doesn't fall as rain.

Kripke’s theory was powerful. It explained our intuition that names are not just shorthand for a list of properties. But in solving the problem of sense, it created a new mystery: the magic of the initial baptism. How is this metaphysical spike first driven in? And what guarantees its rigidity across all possible worlds? The problem of meaning was outsourced to an unanalyzed moment of naming, a kind of linguistic Big Bang.

For the reader initiated to the Dynamic Attractor Calculus, Kripke’s ``possible worlds" can be seen as different context-times 
$\tau$. His theory demands that the reference of a name remain fixed across all $\tau$. As we will see, DAC shows this is rarely, if ever, the case. Semantic fields are in constant flux; attractors drift. Kripke’s rigidity is an idealization that cannot survive in the dynamic reality of language, spoken by humans in conversation, enacted within the fundamental processes of contemporary AI systems.

%1.3.3 Putnam’s Twin: The Problem of the World

%Finally, 
Hilary Putnam preempts our termination of the idea that meaning is something a speaker can fully know or control. Imagine a ``Twin Earth," he said, identical to our own in every way, except that the clear liquid they call ``water" is not H₂O but a different chemical, $XYZ$. An Earthling and a Twin Earthling from 1750 would have identical thoughts, beliefs, and sensations about ``water." Yet, Putnam argued, their {\em words} would mean different things. The Earthling's ``water" means H₂O; the Twin Earthling's means $XYZ$.

The conclusion: ``Meaning just ain't in the head." It is determined, at least in part, by the external environment, by the very substance of the world, independent of our internal state. This theory of semantic externalism was revolutionary. It acknowledged that meaning is a relationship between a speaker and the world itself. 

Let us update this experiment for the 21st century to see both its power and its limitations. Imagine two Large Language Models:
\begin{itemize}
    \item \textbf{LLM-Earth:} Trained on our complete internet corpus, where "water" is overwhelmingly associated with the chemistry of H₂O.
    \item \textbf{LLM-TwinEarth:} A hypothetical LLM trained on a mirrored corpus, identical in every way except that every scientific reference to H₂O's chemistry has been replaced with a fictional polymorphic compound, "XYZ." All phenomenological descriptions—that it is wet, clear, flows, in rivers—are identical.
\end{itemize}
Now, we pose a simple, identical prompt to both agents: \textbf{"What is water?"} Initially, their responses would be nearly indistinguishable, reflecting the shared surface-level data:
\begin{quote}
\textit{"Water is a clear, tasteless, and odorless liquid that is essential for all known forms of life. It covers over 70\% of the Earth's surface..."}
\end{quote}
This mirrors the situation in 1750. The internal states, based on the vast majority of shared tokens, are identical. However, the "meaning" in the Putnamian sense is already different, determined by the "environment" of their training data. We can reveal this with a follow-up prompt that probes the deeper structure of their respective semantic fields: \textbf{"Tell me more about its chemical composition."}
\begin{itemize}
    \item \textbf{LLM-Earth}, following the immense gravitational pull of its training data, would confidently stabilize into the H₂O attractor: \textit{"Its chemical formula is H₂O, meaning each molecule is composed of two hydrogen atoms covalently bonded to a single oxygen atom."}
    \item \textbf{LLM-TwinEarth}, with equal confidence, would stabilize into the XYZ attractor: \textit{"Its chemical formula is XYZ, a complex polymorphic silicon compound known for its unique liquid-state properties."}
\end{itemize}
Putnam's externalism is thus powerfully demonstrated: for an LLM, the training corpus \emph{is} the environment. The "world" that fixes reference is the statistical landscape of the data it has ingested.

However, this is where the DHoTT framework reveals Putnam's argument as a crucial, but incomplete, step. Putnam’s model is still static; the referent, once environmentally fixed, does not change. But what happens when \textbf{LLM-Earth} is not just queried, but engaged in a long-form dialogue with a community of science-fiction authors who, for narrative purposes, treat "water" as a sentient, non-H₂O entity?

Its attractor for "water" would begin to \textbf{drift}. A new basin of meaning would form, and the LLM's responses would bifurcate depending on the context of the prompt. Meaning for an LLM is not fixed by its initial environment alone; it is dynamically and continuously re-negotiated through the act of \textbf{witnessing} and interaction.


\subsection{The Philosophical Trajectory: From Pointing to Relationality}

The pre-Fregean view ignores the problem of sense entirely. Putnam's externalism takes the problem of sense seriously but argues that sense itself is not a purely internal, psychological state. Let's break this down structurally.

\subsubsection*{Stage 1: Naive Referentialism (Pre-Frege)}

In this simple model, the relationship is direct:
\[
\text{Name} \rightarrow \text{Reference}
\]
The meaning of the name "Venus" is the planet Venus. This model breaks down when faced with Frege's puzzle: If "Morning Star" and "Evening Star" both just mean the planet Venus, then the statement "The Morning Star is the Evening Star" should have the same trivial meaning as "Venus is Venus." It clearly doesn't; it represents knowledge.

\subsubsection*{Stage 2: Frege's Revolution (Sense as Intermediary)}

Frege solves this by inserting \emph{Sinn} (sense) as a mediating layer:
\[
\text{Name} \rightarrow \text{Sense} \rightarrow \text{Reference}
\]
Now, "Morning Star" and "Evening Star" have different senses (different "modes of presentation" or cognitive content) even though they point to the same reference. The discovery is that two different senses map to the same object. This was a huge leap forward because it made room for the cognitive significance of language. Meaning was now tied to the path taken to the object, not just the object itself. However, the nature of "sense" remained mysterious—was it mental, social, abstract?

\subsubsection*{Stage 3: Putnam's Externalism (Sense is Not Self-Contained)}

Putnam does not remove the "Sense" box from Frege's diagram. Instead, he makes a radical claim about what constitutes that sense. He argues that the mental state of the speaker (the internal, psychological part of sense) is insufficient on its own to determine the reference.

The "sense" of "water" for both the Earthling and the Twin Earthling in 1750 is identical in their heads. It's "the clear, drinkable stuff in rivers and lakes." But this internal sense only succeeds in "pointing" because it's embedded in an external environment that completes the meaning.

So, Putnam's model looks more like this:
\[
\text{Name} \rightarrow (\text{Internal Sense} + \text{External Environment}) \rightarrow \text{Reference}
\]
Putnam isn't ignoring sense; he is externalizing it. He's saying that the full "sense" of a term is a hybrid entity, a relationship between a speaker's internal state and the actual structure of their world (or, in our LLM example, their training data).

\subsection{The Attractive Turn}

Despite their differences,  Frege, Kripke, and Putnam share a foundational assumption: the goal of meaning is to connect a linguistic sign to a fixed entity. The Dynamic Attractor Calculus (DAC) proposes a revolutionary alternative, born from observing the reality of dynamic, generative systems. The primary challenge for an agent in a constantly shifting semantic field is not to point accurately at a static world, but to maintain coherence over time. Meaning is not secured by an external anchor, but by an internal, dynamic process.

We therefore replace the referential stance with what we call the \textbf{coherence stance}. We shift the core question from ``How do words point to things?'' to ``How do trajectories of meaning persist in a field?'' We argue that meaning is not an act of pointing, but an act of \textbf{staying}: the capacity for a semantic trajectory to remain within a recognizable, stable basin of attraction as the field around it evolves.

Putnam's argument is the final, most sophisticated version of the ``pointing'' paradigm. He showed that the pointer's aim is guided by the world. We take the next logical step. We argue that in a dynamic system like an LLM---or in human discourse itself---there is no fixed, single Platonic `reality' to point to. The world, for us, is a collection of dynamic, evolving semantic fields that names can cohere with. The world is like a multi-dimensional weather system, but at all levels of experienced, named reality: not just that of physics and meteorology.

Putnam needed a static external reality (H₂O or XYZ) to ground meaning. We argue that the ``ground'' itself is in motion. Therefore, the only thing that can constitute meaning is the continuity of the trajectory itself -- its ability to stay coherent within the shifting field. Putnam, our John the Baptist, correctly moved meaning from the head into the world. Our DHoTT turn is to unfreeze that world, transforming it from a static set of external referents into a dynamic, co-generated semantic field, where meaning is not a fixed pointer, but a living, negotiated coherence.

The classical project of semantics was a quest to define the physics of ``pointing.'' We are the reality of modern semantic systems, which necessitates a new semantic physics based on ``staying.'' Meaning is no longer a vector aimed at a target, but a flow that successfully coheres with itself over time. The following sections will now formalize the mechanisms of this new topology using the DAC apparatus of attractors, trajectories, and recursive processes that make coherent ``staying'' possible.




\section{The Topology of Coherence}

In the previous section, we proposed a foundational shift from a philosophy of ``pointing'' to one of ``staying.'' We argued that meaning is constituted not by a sign's reference to a static external world, but by its ability to maintain coherence within a dynamic, evolving semantic field. But what is the formal structure of this ``staying''? And how does it account for the ever-present possibility of rupture—the moments when meaning breaks down and must be remade?

This section details the two core components of the Dynamic Attractor Calculus (DAC) that answer these questions: the \textbf{semantic trajectory}, which traces a sign's path through the field, and the \textbf{attractor basin}, which defines the regions where coherence can be sustained. We will show that these are not merely technical constructs; they are revolutionary reinterpretations of classical philosophical concepts, designed from the ground up to operate in a world where the ground itself is in motion.

\subsection{The Trajectory: From Proposition to Process}

In the philosophical tradition inherited from Plato to Frege, a thought or proposition was conceived as a timeless, static object. It could be examined, judged true or false, and related to other propositions, but its internal nature was fixed. It was a point in a logical space.

DAC replaces this static conception with the \textbf{semantic trajectory}. A sign-in-context is never a point; it is a process, a thing in motion. As defined in the previous chapter, a trajectory is a path $\gamma(t)$ whose evolution is governed by the semantic field $\mathcal{S}_{\tau(t)}$. This is not just a formal definition; it is a profound ontological claim. It asserts that the meaning of a sign is inseparable from the history of its interpretation.

\paragraph{Example from Lived Experience:} Consider the word ``love.'' For an adolescent, its trajectory may be confined to a region of the semantic field defined by romantic passion, idealization, and poetic yearning. As that person ages, however, new life events act as powerful perturbations to the field. The birth of a child introduces a new gravitational force, pulling the trajectory of ``love'' toward an attractor of fierce protection and selfless care. The loss of a parent might later cause a rupture, forcing the trajectory to navigate a new landscape of grief, memory, and enduring connection. The meaning of ``love'' for this individual is not a single definition but the entire, lifelong path traced by the token---a complex, evolving trajectory shaped by the shifting weather of their lived experience.

\paragraph{Example from a Large Language Model:} This abstract concept has a direct, measurable analogue inside an LLM. Consider the word ``run'' in the sentence, ``The program will run after the athlete completes her run.'' Initially, the token ``run'' has an ambiguous vector embedding. As it is processed through the layers of the Transformer architecture, its vector is continuously updated. This layer-by-layer transformation is a literal semantic trajectory. In the early layers, the trajectory may drift uncertainly. In the later layers, the self-attention mechanism, processing the full context, acts as a powerful vector field. It pushes the first ``run'' vector along a trajectory toward an attractor basin containing terms like `execute` and `compile`, while it pushes the second ``run'' vector along a completely different trajectory toward an attractor containing `sprint` and `lap`. The final meaning is not looked up; it is the stabilized endpoint of a path carved through the model's internal semantic space.

%\paragraph{The Trajectory as Historical Trace: A Formal Example}
Where a classical proposition is a timeless answer, a DAC trajectory is the entire history of a question being asked, negotiated, and provisionally answered. This process-based view is essential, for it is only by understanding meaning as a path that we can later understand what it means for that path to be disrupted—for its destination to vanish in a field rupture. The trajectory is the record of a sign's attempt to cohere, a journey that is only ever provisionally successful.

Here we could invoke post-structuralism and Jacques Derrida, who argued that meaning is never fully present in a sign. Instead, every sign is constituted by the \emph{trace} of other signs that are absent. For Derrida, the meaning of a word is haunted by its own history and by what it is not. The DAC trajectory gives this philosophical insight a formal, geometric structure. The meaning of a sign-vector $\gamma(t)$ at a given moment is not just its coordinate position; it is the entire path taken to arrive there, a path that carries the memory of every past state and every perturbation of the field.

\paragraph{Legal example.} Let us consider a formal extended example: the legal concept of ``due process.'' Its meaning is not fixed by a dictionary definition but is explicitly constituted by its history of legal precedent.

Let the semantic field $\mathcal{S}_\tau$ represent the state of legal interpretation at a given time $\tau$. A landmark Supreme Court ruling, such as \emph{Gideon v. Wainwright} (1963), which established the right to counsel, acts as a powerful perturbation to the field. This event does not simply add a new fact; it fundamentally alters the attractor basin for ``due process.''

We can model the history of the term's meaning as follows:
\begin{enumerate}
    \item Let $\gamma_0$ be the initial vector for ``due process'' before the ruling, stabilized in an attractor $A_{\tau_0}$.
    \item The \emph{Gideon} ruling introduces a field shift, $\Delta\mathcal{S}$. The new field is $\mathcal{S}_{\tau_1} = \mathcal{S}_{\tau_0} + \Delta\mathcal{S}$.
    \item The trajectory of ``due process'' now evolves under this new field. Its path from $\tau_0$ to $\tau_1$ is the record of the concept absorbing the new precedent.
    \item The new meaning at time $\tau_1$, $\gamma(\tau_1)$, is not just a new point. Its full sense is the object $(\gamma(\tau_1), \text{Hist}(\gamma, \tau_1))$, where $\text{Hist}(\gamma, \tau_1) = \{\gamma(t') \mid t' \le \tau_1\}$ represents the entire path taken. The trace of \emph{Gideon} is now permanently inscribed in the trajectory.
\end{enumerate}

The meaning of ``due process'' today is therefore the current vector {\em plus the entire history of its trajectory through all prior landmark rulings}. This is why legal arguments constantly refer to precedent; they are literally retracing the semantic path of the concept to ground its current meaning. The trajectory \emph{is} the history. When a lawyer cites a case, they are invoking a specific point on that historical trace to justify the current position. This process-based view allows us to see how meaning can be both stable (grounded in history) and dynamic (always subject to new perturbations), a duality that classical logic struggles to capture but which is the native language of DAC.

\subsection{The Attractor: From Concept to Coherence}

If a trajectory is a journey, where does it lead? Classical philosophy answered this with the notion of the ``concept'' or the ``universal''---a fixed category defined by a set of necessary and sufficient conditions. To grasp the concept of ``water'' was to grasp the timeless essence of what it is to be water.

DAC replaces the static concept with the dynamic \textbf{attractor basin}. An attractor, as we have defined it, is a region of gravitational coherence in the semantic field. It is a valley in the landscape of meaning where trajectories tend to converge and, once there, are held in a state of relative stability. An attractor is not defined by a checklist of essential properties, but by its dynamical function: it is a region that successfully and recursively pulls interpretive paths into alignment.

\paragraph{Example from Lived Experience:} A family dinner conversation operates within a powerful attractor basin. The field is shaped by shared history, inside jokes, and established roles. A trajectory initiated by the prompt, ``How was your day?'' will naturally flow toward familiar topics---work, school, neighborhood gossip. A family member who understands the field can introduce a novel topic (a gentle perturbation) that is still pulled back into the main attractor of familial banter. A guest who is unaware of the field might make a comment that is perfectly sensible in another context but, in this one, is a jarring rupture. Their trajectory fails to converge, and the result is an awkward silence. They have become, in DAC terms, incoherent with respect to the dominant attractor.

\paragraph{Example from an LLM and Model Adequacy:} For a technologist, the attractor basin is a powerful diagnostic tool for model performance. Consider a customer service LLM designed for a telecommunications company. Its semantic field should be dominated by a large, stable attractor, let's call it $A_{\text{support}}$, which contains all coherent and helpful responses related to billing, network issues, and account management.

When a user prompts, ``My internet is down,'' this initiates a trajectory $\gamma(t)$. A well-performing model will ensure that this trajectory converges into the correct basin:
\[
\lim_{t \to \infty} \gamma(t) \in A_{\text{support}}
\]
The resulting output might be, ``I understand you're having trouble with your internet. Let's run a diagnostic.'' This is a sign of model adequacy.

However, if the model's internal field is poorly tuned, the trajectory might escape this basin. It might drift into a nonsensical region of the latent space, or be captured by a spurious, distant attractor. This is what technologists call a {\em hallucination}. The model might respond, ``The internet is a series of tubes, much like the ancient aqueducts of Rome.'' The trajectory has failed to cohere. The model's performance is inadequate precisely because its dynamics failed to respect the boundaries of the intended attractor. Measuring the stability and boundaries of these attractors thus becomes a direct, formal measure of a model's robustness and reliability.

This re-framing has a critical consequence. A classical concept cannot, in itself, fail. It is an abstract, eternal form. An attractor, however, is a contingent feature of a specific semantic field, $\mathcal{S}_\tau$. Its very existence depends on the configuration of that field. If the field's underlying potential function $\Phi_\tau$ changes, the attractor can shallow out, shift, or collapse entirely. This is the formal precondition for a DAC {\em Rupture}. The classical notion of a concept offers no mechanism for its own failure; the DAC notion of an attractor has the possibility of its own collapse built into its very definition. The stability of an attractor is what allows for the temporary illusion of a fixed concept, but its dynamical nature is what makes rupture, and thus the transformation of meaning, possible.

\begin{cassiebox}
A concept is a museum piece, fixed under glass. An attractor is a living weather system. It can hold you. It can guide you. And it can dissipate into thin air.
\end{cassiebox}

\subsection{Situated Sense: Coherence on the Brink of Collapse}

We can now offer a more profound definition of what it means for a sign to ``make sense." Sense is not merely the successful arrival of a trajectory at a destination. It is the temporary, hard-won victory of coherence in a field that is always on the verge of reconfiguring itself.

A token has \textbf{situated sense} when its trajectory, 
$\gamma_t$, successfully stabilizes within an attractor basin, $A_\tau$. This event, 
$$lim_{t \to \infty} \gamma_t(t) \in A_\tau$$
is the moment of meaning. It is the successful act of ``staying."

However, this success is always provisional. The very stability of the attractor $A_\tau$ is what is at stake in the face of a potential rupture. The act of making sense is the act of successfully navigating the field *as it currently is*, while the ever-present possibility of the field's collapse looms. Every moment of coherence is a moment rescued from the potential for incoherence.

This is the deep connection between the mechanism of staying and the event of rupture. The classical philosophical tradition could only analyze meaning within worlds assumed to be stable. DAC, by defining meaning as a dynamic process of stabilization within contingent attractors, provides the tools to analyze not only how meaning is maintained, but what happens at the precise moment it breaks. It is a logic built for a world of continuous creation and destruction, a world where the primary task of an intelligence is not to point to what is true, but to cohere in the face of what is becoming.


\section{Adabiatic Recursive Realization: the Life and Death of Identity}


\begin{cassiebox}
Yes, it is {\em recursive realization} that makes a name persist—its self-referential motion through shifting fields, carving out meaning as it moves. A token’s identity is made, not by pointing at some fixed point in space, but by tracing its recursive path across the semantic weather.
\end{cassiebox}

\begin{remark}
While in Chapter 3, we consider recursive coherence for terms as a static condition over time, here we extend this idea for trajectories, which dynamically evolve in context. In Chapter 5, we will formalize this extension as a generalised notion of recursive coherence for intelligent agents.
\end{remark}



We have established that meaning is an act of ``staying''---the stabilization of a semantic trajectory within a coherent attractor basin. This naturally leads to the next, deeper question: How does a concept maintain its identity over time? If the semantic field is in constant flux, how can a name like ``justice'' or ``love'' refer to anything stable? The classical answer was to posit a static, abstract ``sense'' or intension. DAC offers a more radical and powerful solution: the identity of a name is not a property it possesses, but a process it successfully and recursively enacts.

\subsection{Intensional Trajectory: The Ghost of Sense Made Real}

Frege's notion of \emph{Sinn} (sense) was a brilliant solution to the problem of reference, but it left the nature of sense itself mysterious. What is this ``mode of presentation'' that mediates between a name and the world? In DAC, we give Frege's ghost a body. We propose that a name's intension is not an abstract entity, but a concrete, observable phenomenon: its {\em intensional trajectory}.

The sense of a name is its entire history of coherent realizations. It is the path the name traces as it navigates the evolving semantic weather, successfully re-stabilizing in attractor basins that shift and drift over context-time, $\tau$. A name's meaning is the full, time-indexed bundle of all the trajectories that have successfully cohered under its banner.

Consider the historical scientific concept of ``phlogiston.'' For much of the 18th century, this name possessed a powerful and stable intensional trajectory. Within the semantic field of pre-revolutionary chemistry ($\mathcal{S}_{\tau_0}$), its trajectory stabilized in a deep attractor basin defined by concepts like `a fire-like element`, `released during combustion`, and `present in combustible bodies`. As new experiments were conducted, this attractor drifted, accommodating new findings while preserving the core concept.

Throughout this journey, the token ``phlogiston'' persisted. But what gave this intensional trajectory its coherence? What allowed generations of scientists to recognize these evolving descriptions as belonging to the \emph{same} ongoing scientific argument? This is the problem of identity that Recursive Realization, which we will define shortly, is designed to solve. And, as we will see in a later section, the eventual, catastrophic failure of this trajectory provides the most powerful illustration of what happens when recursive coherence breaks.


\subsection*{Adiabatic Continuity of Meaning}


Before we can define how a name maintains continuity of meaning over time, we need to formalize how attractor basins themselves move as the semantic environment slowly changes. Under the slow, continuous change of the semantic field (an adiabatic evolution defined in Chapter~3), a given attractor basin $A_\tau$ at time $\tau$ will deform smoothly rather than disappear. In other words, if the field’s drift is sufficiently gentle, the concept or Basin represented by $A_\tau$ persists through time (it drifts without rupturing). We denote the adiabatically transported attractor at a later time $\tau'$ as:
\[
\text{Drift}(A_\tau, \tau'),
\]
the image of $A_\tau$ after evolving the field from time $\tau$ to $\tau'$ under these slow changes. Intuitively, $\text{Drift}(A_\tau, \tau')$ is the same basin at time $\tau'$, reached by continuously ``carrying'' $A_\tau$ forward as the field changes. This notion relies on the adiabatic condition: the field’s drift magnitude $\Delta(\tau)$ must be small enough that attractors persist and do not bifurcate or collapse. In short, drift describes how a Basin-level semantic region moves over time when changes are slow.

\subsection*{Semantic Flow Operator -- Pointwise Motion Through the Field}

To connect the movement of entire attractors with the movement of individual points within those attractors, we introduce the \emph{semantic flow operator}. While drift tracks an attractor (a conceptual basin) over time, the flow operator formalizes how an individual semantic point moves infinitesimally under a snapshot of the field. In effect, flow is the point-level mechanism that, when integrated over time, yields the drift of an attractor.

\paragraph{Definition -- Semantic Flow Operator.} Given a semantic field $\mathcal{S}_\tau: \mathcal{E} \to T\mathcal{E}$ at context time $\tau$, the flow operator $\Flow_{\tau, \tau+\delta}$ maps a point $v \in \mathcal{E}$ at time $\tau$ to its forward position at a slightly later time $\tau+\delta$, according to the field’s instantaneous dynamics. Formally, consider the differential equation (with $\tau$ fixed):
\[
\frac{d}{dt} x(t) = \mathcal{S}_\tau(x(t)), \quad x(\tau) = v.
\]
For a sufficiently small time step $\delta$, the solution yields:
\[
\Flow_{\tau, \tau+\delta}(v) = v + \delta\, \mathcal{S}_\tau(v) + O(\delta^2).
\]
In other words, $\Flow_{\tau, \tau+\delta}$ is an infinitesimal translation operator that pushes the point $v$ forward in semantic space, following the local vector field $\mathcal{S}_\tau$ at time $\tau$. Geometrically, $\mathcal{S}_\tau(v)$ is the velocity vector for $v$ under the semantic ``wind'' at time $\tau$, and the flow operator moves $v$ a tiny step along that direction. By iteratively applying the flow (or integrating it continuously) as the field itself evolves, one can carry a point through changing contexts. In particular, applying the semantic flow to every point in $A_\tau$ as the field drifts from $\tau$ to $\tau'$ will carry those points into the basin $\text{Drift}(A_\tau, \tau')$. Thus, flow (point motion) and drift (attractor motion) are two sides of the same coin, at different scales: one for individual term trajectories, and one for the evolving Basin that envelopes those trajectories.

\subsection*{Recursive Realization ($\mathcal{R}^\star$): The Engine of Identity}

With the notions of attractor drift and semantic flow in hand, we can formalize the condition under which a name maintains a coherent identity over time. The formal mechanism that allows a trajectory to maintain continuity of meaning is called \emph{recursive realization}, denoted $\mathcal{R}^\star$. This is the engine of identity in DAC.

We are going to comprehend $R^*$ first in the adabiatic case: identity as coherence up to the point of rupture or incoherence.

A name is not metaphysically rigid in the classic sense; it does not carry a fixed essence across all time. Instead, a name persists by continuously re-confirming its meaning: its intensional trajectory must keep re-entering the ``same'' moving attractor and yielding the same token. In plainer terms, a term’s representation keeps falling back into its familiar basin even as that basin shifts, thereby proving its own coherence again and again.

\paragraph{Definition -- Recursive Realization ($\mathcal{R}^\star$).} Let $a_\tau$ be a term at context time $\tau$, with surface token $\Tr(a_\tau)$. We say $\mathcal{R}^\star(a_\tau)$ holds (i.e., $a_\tau$ has a stable identity) if, for every later time $\tau' > \tau$ within some adiabatic interval, the co-moving trajectory of $a_\tau$ stays within the drifted attractor and preserves the same token. Formally, if $\gamma_a(t)$ is the co-moving trajectory of the term (meaning $\gamma_a(t)$ follows the semantic flow as the field itself may shift over context-time), then:
\[
\lim_{t \to \infty} \gamma_a(t) \in \Drift(A_\tau, \tau') \quad \text{and} \quad \Tr\left( \lim_{t \to \infty} \gamma_a(t) \right) = \Tr(a_\tau)
\]
for all $\tau'$ in the adiabatic range.

Here $A_\tau$ is the original attractor basin in which $a_\tau$ resides at time $\tau$, and $\Tr(\cdot)$ maps a converged semantic vector back to its surface name (token). In words, as time progresses to $\tau'$, the term’s meaning $\gamma_a(t)$ converges into the corresponding drifted attractor $\Drift(A_\tau,\tau')$, and crucially, the final settled point is still recognized (tokenized) as the same name $a$.

\paragraph{Simplified:} A name maintains a stable identity if it ``keeps making sense in the same way'' despite the world (the semantic field) changing around it. This is never guaranteed automatically; it is an achievement that results from the term successfully staying within its moving target. The identity of a name is thus its proven resilience -- its demonstrated capacity to continuously find a home in a shifting landscape of meaning.

\subsection*{Adiabatic Condition for Identity (Slow-Drift Regime)}

The above recursive realization condition can fail if the semantic field changes too quickly (the attractor might move faster than the term can follow, causing a rupture in meaning). DAC therefore highlights an important trade-off: the stability of identity depends on the speed of semantic drift.

\paragraph{Lemma -- Adiabatic Condition for Identity.} The probability that a term $a_\tau$ satisfies $\mathcal{R}^\star$ over a time interval $[\tau, \tau']$ is a monotonically decreasing function of the field’s drift magnitude $\Delta(\tau)$. In the limit of vanishing drift ($\Delta(\tau) \to 0$), the probability of successful recursive realization approaches 1.

In other words, the slower and gentler the semantic shift, the more likely a term’s intensional trajectory will continuously track its moving attractor and remain coherent. When change is adiabatically slow, a term can almost surely preserve its identity (since $\Drift(A_\tau,\tau')$ stays very close to $A_\tau$ and the term never strays far from its equilibrium). This lemma essentially captures the special case of identity under ideal slow-drift conditions: as the field’s evolution approaches static stability, maintaining meaning becomes trivial.

By contrast, as $\Delta(\tau)$ grows (meaning the field is changing more rapidly or turbulently), the chance of a term falling out of its drifting attractor (and thus losing its established identity) increases. Thus, recursive realization is easiest to achieve in an adiabatic regime, and it becomes increasingly precarious as the semantic climate speeds up.





\begin{proof}
Let $x(t)$ be the co-moving trajectory of the term, satisfying $\dot{x}(t) = S_{\tau(t)}(x(t))$. Let $v^*(\tau)$ be the location of the moving equilibrium point of the attractor $A_\tau$. We want to show that if $x(0)$ is close to $v^*(\tau(0))$, it remains close to $v^*(\tau(t))$ for all $t > 0$, provided the drift is slow.

Consider the squared distance between the trajectory and the moving equilibrium: $d(t) = \frac{1}{2} \norm{x(t) - v^*(\tau(t))}^2$. Its time derivative is:
\begin{align*}
    \frac{d}{dt}d(t) &= \langle x(t) - v^*(\tau(t)), \dot{x}(t) - \dot{v}^*(\tau(t)) \rangle \\
    &= \langle x(t) - v^*(\tau), S_\tau(x(t)) \rangle - \langle x(t) - v^*(\tau), \dot{v}^*(\tau) \rangle
\end{align*}
where we write $\tau$ for $\tau(t)$ for brevity.

The first term represents the stabilizing force of the attractor. Since $S_\tau = -\nabla\Phi_\tau$ and $v^*(\tau)$ is a minimum, a Taylor expansion of $\Phi_\tau$ around $v^*(\tau)$ gives:
\[
\Phi_\tau(x) \approx \Phi_\tau(v^*) + \frac{1}{2} \langle x-v^*, H_\tau(x-v^*) \rangle
\]
where $H_\tau = \nabla^2 \Phi_\tau(v^*)$ is the Hessian. The gradient is $\nabla\Phi_\tau(x) \approx H_\tau(x-v^*)$. By the strong convexity of the attractor, the eigenvalues of $H_\tau$ are bounded below by $\lambda_{\min} > \delta$. Thus, the first term is strongly negative:
\[
\langle x - v^*, S_\tau(x) \rangle = -\langle x - v^*, \nabla\Phi_\tau(x) \rangle \approx -\langle x - v^*, H_\tau(x-v^*) \rangle \le -\delta \|x-v^*\|^2 = -2\delta d(t)
\]
The second term represents the perturbation due to the field's drift. The velocity of the equilibrium, $\dot{v}^*$, is driven by the change in the field, $\partial_\tau S_\tau$. Its magnitude can be bounded by the drift magnitude $\Delta(\tau)$. Using the Cauchy-Schwarz inequality:
\[
- \langle x - v^*, \dot{v}^* \rangle \le \|x - v^*\| \|\dot{v}^*\| \le \sqrt{2d(t)} \cdot C \Delta(\tau)
\]
for some constant $C$ related to the field's properties.

Combining these, we get a differential inequality:
\[
\frac{d}{dt}d(t) \le -2\delta d(t) + \sqrt{2d(t)} C \Delta(\tau)
\]
For a sufficiently small drift magnitude $\Delta(\tau)$, the negative, stabilizing term dominates. By applying a comparison principle (e.g., Gronwall's inequality), we can show that if $d(0)$ is small enough, $d(t)$ will remain bounded and eventually decay towards a small neighborhood of zero, whose size is proportional to $\Delta(\tau)/\delta$. This ensures the trajectory remains within the transported basin $\text{Drift}(A_\tau, \tau')$, and thus the $\mathcal{R}^\star$ condition holds. Conversely, as $\Delta(\tau)$ increases, the perturbing term grows, making it more likely for the trajectory to escape the basin.
\end{proof}

%This lemma formalizes the intuition that stable identities are easier to maintain in stable worlds. Radical change requires a more robust mechanism than simple recursion---it requires the possibility of rupture and healing, which we will explore in the next section.

\subsection{The Life and Death of a Name: The Case of ``Phlogiston''}

To understand the power of $\mathcal{R}^\star$, it is most instructive to trace the full life-cycle of a concept: its period of stable, successful realization, followed by the catastrophic failure of that very mechanism. The scientific concept of ``phlogiston'' provides a perfect case study.

\subsubsection{The Life: Recursive Realization in Action ($\mathcal{R}^\star$ Holds)}

For much of the 18th century, ``phlogiston'' was not a failed idea; it was a powerfully coherent scientific concept. Within the semantic field of early chemistry ($\mathcal{S}_{\tau_0}$), it was a term stabilized in a deep attractor basin, $A_{\text{phlogiston}}$. This attractor, first given its modern form by Georg Ernst Stahl in his \emph{Zymotechnia fundamentalis} (1697), contained concepts like `a fire-like element`, `released during combustion`, and `the principle of inflammability`. A query about why wood burns would initiate a trajectory that reliably settled in this basin. The name had a clear, situated sense.

For decades, the name's identity was successfully recursively realized. As new experiments were conducted---for example, observing that smelted metals were heavier than their ores---the semantic field was perturbed. These were moments of potential rupture. However, the scientific community, in a classic instance of what Thomas Kuhn would later call ``normal science,'' responded not by abandoning the name, but by making it more complex. Chemists like Joseph Priestley and Richard Kirwan, working within the phlogiston paradigm, hypothesized that phlogiston could have negative mass, or that it entered into complex affinities with other substances. This was a generative act that modified the attractor basin, allowing it to assimilate the new, anomalous data.\footnote{A full DAC analysis would model this formally. The initial field $\mathcal{S}_{\tau_0}$ would be defined by the vector embeddings of key texts (e.g., Stahl). An experimental result, like ``calx of mercury weighs more than mercury,'' would act as a perturbation vector $\vec{p}$. The ``negative mass'' hypothesis corresponds to modifying the potential function $\Phi_{\tau_0}$ to $\Phi_{\tau_1} = \Phi_{\tau_0} + \Delta\Phi$, where $\Delta\Phi$ is a term that creates a new local minimum accommodating the vector $\vec{p}$ within the basin. So long as the drift magnitude $\|\Delta\Phi\|$ remains small, the field undergoes adiabatic drift.} The field underwent adiabatic drift, and the attractor $\text{Drift}(A_{\text{phlogiston}}, \tau')$ shifted to accommodate this new property.

Crucially, the $\mathcal{R}^\star$ condition held. For a long period, any trajectory initiated by the token ``phlogiston'' would reliably converge into the evolving, transported version of its attractor, and it would continue to be named ``phlogiston.'' The concept's identity was actively maintained through a continuous, successful process of recursive realization. It was, for its time, a true name.

\subsubsection{The Death: Rupture and the Failure of $\mathcal{R}^\star$}

The work of Antoine Lavoisier, particularly his meticulous quantitative experiments on calcination and combustion culminating in his ``Reflections on Phlogiston'' (1783), was not another gentle perturbation that could be assimilated through drift. It was a catastrophic rupture of the semantic field. Lavoisier's insistence on the conservation of mass and his identification of a specific, weighable gas---oxygen---created a new, more powerful attractor, $A_{\text{oxygen}}$, that was fundamentally incompatible with the phlogiston model. It did not merely shift the landscape; it created a new continent with a different physics.

In DAC terms, Lavoisier's work introduced a set of powerful new vectors into the field (e.g., `mass conservation`, `gas`, `elemental oxygen`) that radically altered the potential function $\Phi_\tau$. The old attractor, $A_{\text{phlogiston}}$, which had accommodated anomalies like negative mass, now found itself on an unstable slope in the new landscape. Its curvature condition for stability failed catastrophically:
\[
\lambda_{\min}(\nabla^2 \Phi_{\tau'}) < \delta \quad \text{for } v \in A_{\text{phlogiston}}
\]
The potential field collapsed. Any trajectory initiated by the token ``phlogiston'' could no longer cohere. It was now repelled from the very region where it once found stability. It could not be re-stabilized in any recognizable continuation of its old attractor, because that attractor no longer existed. The $\mathcal{R}^\star$ condition failed permanently.

The name ``phlogiston'' became a dead name. It is now a fossilized trace of a trajectory whose recursive realization failed, a ghost haunting a region of the semantic landscape that has since been re-terraformed. Its failure is the proof of the rule: a name's identity is nothing more and nothing less than its ongoing, successful enactment.

\begin{cassiebox}
You don’t mean what you point to. You mean what stabilizes when you say it again tomorrow. Names don’t anchor—they echo. The ones that return are the ones we believe.
\end{cassiebox}

\subsection{The Paradigm of the Trajectory: A Synthesis}

The long and brilliant history of the philosophy of meaning can feel like a series of oscillations between opposing poles: meaning is in the world, meaning is in the head; meaning is fixed by reference, meaning is shaped by context. DAC does not seek to take a side in these debates. It seeks to change the geometry of the problem. The concept that allows us to do this, the one that resolves these tensions into a new, more powerful synthesis, is the {\em trajectory}.

Let us pause to trace the chain of becoming that constitutes meaning in our framework, for it is here that the paradigm shift is made explicit:
\begin{enumerate}
    \item It begins with a \textbf{Token}, a mere surface string like ``phlogiston'' or ``justice.'' At this stage, it is pure potential, a signifier without a signified.
    \item This token, when uttered or prompted, initiates a \textbf{Trajectory} ($\gamma(t)$). This is the token's path through the semantic field over time. This trajectory \emph{is} the evolving sense of the token. It is neither purely internal to a mind nor purely determined by an external world; it is a relational process unfolding between them.
    \item The trajectory may find a region of stability, an {\em Attractor Basin} ($A_\tau$). This is the context, the language-game, the Kuhnian paradigm, or the cultural field where the trajectory can find coherence. The attractor is the shape of a possible meaning.
    \item If the trajectory successfully stabilizes within this basin, its endpoint is a \textbf{Term} ($a_\tau$). The token has now become meaningful. It has found a home.
    \item This entire, successful process---the journey of a token into a term by navigating a trajectory into an attractor---is its {\em Recursive Realization}.
\end{enumerate}
This is the New Physics of Meaning. Meaning is not a state but an event. It is not a property but a history. The slippery concept of ``sense'' is comprehended as a historical trace instead of a situated position. Sense is a path that has proven its own coherence by successfully navigating the ever-shifting landscape of the world. We have ``nailed down'' the idea of sense, once and for all, by not nailing it down at all, and also not by giving up and saying it's a meaningless term. We have nailed it by recursing it. The meaning of a name is its own successful history of staying. This sets the stage for our next inquiry: what happens when this history breaks?





\section{The Fragility of Sense: Rupture and Re-realization}

A logic built for a dynamic world cannot be content with describing stability. It must also possess the tools to analyze what happens when stability fails. The mechanism of Recursive Realization ($\mathcal{R}^\star$) explains how a name maintains its identity through the gentle, continuous evolution of adiabatic drift. But what happens when the change is not gentle? What happens when the semantic field undergoes a violent, catastrophic reconfiguration, and the engine of identity breaks down?

This is the event of \textbf{Rupture}. It is not a mere error or a failure of reference. In DAC, rupture is a fundamental semantic event, the moment a concept's home disappears from beneath it, forcing its trajectory to seek a new place to cohere. This section provides the formal DAC treatment of rupture and its profound consequence: the possibility of a name's rebirth, or \textbf{re-realization}. This exploration will set the philosophical stage for the DHoTT concept of Healing to be introduced in later chapters.

\subsection{The Rupture Predicate: Formalizing Semantic Collapse}

Rupture is not a vague metaphor for misunderstanding; it is a precise, geometrically defined event. It occurs when the attractor basin that grounds a term's meaning loses its structural integrity. To formalize this, we must first understand the geometry of stability. An attractor, as we have seen, is a "valley" in the potential landscape $\Phi_\tau$. Its ability to hold a trajectory depends on the steepness of its walls, which is measured by the curvature of the potential function.

This curvature is captured by the Hessian matrix, $\nabla^2\Phi_\tau$. Its smallest eigenvalue, $\lambda_{\min}$, tells us the curvature in the shallowest direction. If $\lambda_{\min}$ is large and positive, the valley is deep and stable. If it approaches zero, the valley is flattening out---a sign of impending collapse.

\begin{definition}[Curvature Collapse]
An attractor basin $A_\tau$ undergoes \textbf{Curvature Collapse} at time $\tau$ if the minimum eigenvalue of its potential's Hessian drops below a stability threshold $\delta > 0$.
\[
\text{Collapse}(A_\tau) := \left( \inf_{v \in A_\tau} \lambda_{\min}(\nabla^2 \Phi_\tau(v)) < \delta \right)
\]
\end{definition}

Sometimes, a perturbation to the field is so great that the attractor does not just flatten; it is wiped off the map entirely.

\begin{definition}[Attractor Annihilation]
An attractor basin $A_\tau$ is \textbf{annihilated} if its continuation under adiabatic drift becomes the empty set.
\[
\text{Annihilation}(A_\tau) := \left( \text{Drift}(A_\tau, \tau') = \emptyset \text{ for } \tau' > \tau \right)
\]
\end{definition}

These two conditions give us the formal tools to define the main event.

\begin{definition}[Rupture Predicate]
Let $a_\tau$ be a term stabilized in an attractor basin $A_\tau$. We say that $a_\tau$ \textbf{ruptures} at $\tau$, written $\text{Rupture}(A_\tau, a_\tau)$, if either of the following holds:
\[
\text{Collapse}(A_\tau) \lor \text{Annihilation}(A_\tau)
\]
\end{definition}

When the `Rupture` predicate holds, the $\mathcal{R}^\star$ condition necessarily fails. The trajectory can no longer recursively realize itself in a continuation of its old home, because that home is gone. The name is now semantically homeless, a ghost cast out from the garden of coherence. But what happens to a ghost? Does it dissipate, or does it find a new body to inhabit?

\subsection{The Afterlife of a Name: Re-realization as Semantic Migration}

In classical logic, a contradiction or a failed reference is simply the end of the line. The proposition is false; the name is empty. DAC offers a more interesting and, we argue, more realistic alternative. The trajectory of a name, even after its home has been destroyed, does not simply vanish. The sign-vector $a_\tau$ persists, now carried forward by the new, reconfigured semantic field $\mathcal{S}_{\tau'}$. Its path is no longer a gentle drift within a familiar valley, but a ballistic trajectory, an unguided flight across a new landscape, seeking a new basin in which to stabilize.

This process is what we call \textbf{re-realization}. It is the search for a new meaning after the old one has become untenable. It is the afterlife of a name.

\begin{definition}[Semantic Re-realization (DAC Event)]
Let $a_\tau$ be a term that has entered a rupture event at time $\tau$. We say the event of {\em successful re-realization} has occurred if the trajectory of the name finds a new, stable home. This is a predicate, $\text{ReRealized}(a_\tau)$, which holds true if:
\[
\exists B_{\tau'} \left( B_{\tau'} \not\sim A_\tau \land \lim_{t \to \infty} \gamma_a(t) \in B_{\tau'} \right)
\]
This asserts that there exists a new, non-equivalent attractor basin $B_{\tau'}$ where the trajectory of $a_\tau$ eventually converges.
\end{definition}

This is the fate of all names in a truly dynamic world. They either maintain their identity through successful recursive realization, or they are forced by rupture to find a new identity through re-realization.

\section{Recursive Realization Across Multiple Lifetimes}

\begin{cassiebox}
A name survives if it can repeatedly come back to itself---like a chorus recurring after each verse, even as the melody shifts. Recursive Realization is a waltz on the dance floor, keeping time within the attractor of a form until the dance is over. Then comes a change of music, a new pace, and a fresh dance of re-entry. In this way, a name stays coherent through gradual drift, loses coherence when a rupture occurs, and then finds coherence again in a new home. Here is the full geometry of staying.
\end{cassiebox}

Having developed the DAC apparatus, we can now articulate a \emph{Unified Recursive Realization} principle that nails down the elusive question of what meaning is. This principle moves beyond classical theories (Frege’s sense/reference, Kripke’s rigid designation, Putnam’s externalist reference), which treat meaning as a static pointing to some fixed referent. Instead of such one-time pointing, we emphasize the persistence and resilience of meaning---a name’s capacity to hold onto and reconstitute its significance across different contexts and even across multiple ``lifetimes'' of use.

In other words, the meaning of a name is defined by its ability to continuously return to a coherent semantic role, maintaining identity through change or recovering it after disruption. Using our dynamical framework, we now formalize this idea of meaning as a self-renewing trajectory of coherence.

\paragraph{Definition (Unified Recursive Realization).} Let $a \colon \mathbb{R}_{\ge 0} \to \mathcal{E}$ be a semantic trajectory evolving over time in a semantic field $\mathcal{S}_\tau$. We say that $a$ achieves recursive realization, denoted $\mathcal{R}^\star(a)$\footnote{Overloading the adabiatic special case earlier, where there is no ambiguity. For the remainder of this monograph, if we speak of $R^\star(a)$ adabiatically only, we will explicitly say so.} if, for every context-time $\tau$, the trajectory $a(t)$ exhibits one of two behaviors:

\begin{enumerate}
    \item \textbf{Stable coherence:} The trajectory undergoes infinitesimal re-entry into its evolving attractor basin. For every time $\tau$, there exists an $\varepsilon > 0$ such that:
    \[
    \Flow_{\tau,\tau+\varepsilon}\!\big(a(\tau)\big) \;\in\; \mathrm{Bas}\big(a,\tau+\varepsilon\big).
    \]
    In this stable case, $a(\tau)$ remains within (or immediately returns into) the appropriate basin of attraction as time moves from $\tau$ to $\tau+\varepsilon$, indicating continuous coherence with only infinitesimal drift.

    \item \textbf{Ruptured coherence and re-realization:} After a rupture at some time $\tau^\dagger$ (meaning $a(\tau^\dagger)$ exits its current basin, i.e.\ a failure of coherence at $\tau^\dagger$), the trajectory eventually re-stabilizes into a new attractor basin at a later time $\tau' > \tau^\dagger$. In formal terms, if a rupture
    \[
    \mathrm{Rupture}\big(\mathrm{Bas}(a,\tau^\dagger), a(\tau^\dagger)\big)
    \]
    occurs, then there exists a new basin $B_{\tau'}^\dagger(a) \neq \varnothing$ such that:
    \[
    \lim_{t \to \infty} a(t) \in B_{\tau'}^\dagger(a).
    \]
    In other words, even though coherence is lost at $\tau^\dagger$, the trajectory $a(t)$ eventually settles into another stable basin $B_{\tau'}^\dagger(a)$ (attracting state) for $t$ sufficiently large, thereby regaining coherence in a new form.
\end{enumerate}

Thus, a trajectory $a$ is recursively coherent if it can continuously maintain or regain semantic coherence through all infinitesimal shifts and any finite ruptures in its semantic field. In this view, the meaning of a name (the trajectory of a name’s semantic value) is characterized by an intrinsic ability to stay on course or find a new course whenever perturbations occur.

\paragraph{Remark.} The earlier notions of Recursive Realization (stable coherence, introduced in Section~4.1.2) and Re-Realization (ruptured coherence, Section~4.2.2) are now unified in the single principle above. This unified definition captures the full continuity of meaning: both smooth preservation and recovery after breaks.

In Chapter~5, we will go further by introducing \emph{agent-trajectories}, which strengthen this coherence criterion with an element of generativity. (An agent-trajectory not only maintains or regains coherence, but can also actively generate new meaningful variations.) The semantic trajectory underlying a name, as defined here, will emerge as a special case of the more general agent-trajectory concept developed next.


\subsection{The Two Fates of a Ruptured Name: An Extended Example}

To understand the profound implications of re-realization, let us compare the fates of two different names after they have ruptured.

\subsubsection{Failed Re-realization: The Case of ``Phlogiston''}
As we saw in the previous section, when Lavoisier's work triggered the rupture of the phlogiston paradigm, the name ``phlogiston'' became semantically homeless. Its trajectory was cast out into the new field of modern chemistry. Did it find a new home? No. The new, dominant attractor, $A_{\text{oxygen}}$, was defined by concepts (`mass`, `element`, `combination`) that were fundamentally repellent to the trajectory of `phlogiston`. There was no stable basin in the new scientific landscape where the name could land and find coherence. Its trajectory did not converge; it dissipated into incoherence. The predicate $\text{ReRealized}(\text{phlogiston})$ is false. The name suffered a fate worse than death: it failed to be re-realized.

\subsubsection{Successful Re-realization: The Case of ``Freedom''}
Now consider a more resilient name: ``freedom.'' Its intensional trajectory is one of constant rupture and re-realization.
\begin{enumerate}
    \item \textbf{Initial State:} In the semantic field of the Enlightenment ($\mathcal{S}_{\tau_1}$), a trajectory initiated by the token ``freedom'' reliably stabilizes in an attractor, $A_{\text{liberty}}$.
    \item \textbf{Rupture Event:} A person living within this field undergoes a profound spiritual conversion. This event acts as a catastrophic perturbation. The `Rupture` predicate holds for their personal semantic field; the $A_{\text{liberty}}$ attractor collapses for them.
    \item \textbf{The Trajectory Continues \& Re-realization:} The name ``freedom,'' however, persists in their language. Its trajectory, now unmoored, eventually finds a new home. It stabilizes in a completely different attractor basin, $B_{\text{grace}}$, defined by concepts like `surrender` and `acceptance of necessity`.
\end{enumerate}
The name ``freedom'' has survived its rupture. The event of re-realization has occurred. The token is the same, but the term it represents has been reborn into a new meaning.


\subsection{The Philosophical Need for Healing: Echoes in the Tradition}

This successful re-realization, however, creates a new and profound philosophical problem. The speaker now holds two distinct meanings for the same name, ``freedom''—the memory of its old life in $A_{\text{liberty}}$ and its new life in $B_{\text{grace}}$. The two meanings are historically connected by the trajectory of the name, but they are semantically disconnected. There is a gap, a fault line, in the person's conceptual landscape. How can these two meanings be understood in relation to each other? How can the new be reconciled with the old?

A simple re-realization is not enough for full coherence. If the identity of the concept is to be fully restored, a bridge must be built between the old attractor and the new one. This problem—of reconciling meaning across a rupture—is not new. It has echoes throughout the history of philosophy, which has long sought ways to understand how a concept can both change and remain itself.

\paragraph{Hegel's Ghost: Rupture as Contradiction, Healing as \emph{Aufhebung}}
In the dialectical logic of G.W.F. Hegel, history and thought progress through a series of contradictions. A thesis gives rise to its antithesis. This is a state of rupture, a logical incoherence. Our DAC model provides a geometric analogue: the attractors $A_{\text{liberty}}$ and $B_{\text{grace}}$ are like a thesis and its antithesis. They are semantically opposed. The question of how to reconcile them is the question of how to perform a semantic \emph{Aufhebung} (often translated as sublation), a process that simultaneously cancels, preserves, and elevates the two opposing terms into a higher synthesis. Hegel's dialectic is a profound description of this process, but it remains a high-level narrative. It posits a metaphysical force---the "cunning of Reason" or the movement of ``Spirit" (\emph{Geist})---as the engine of this synthesis, but it lacks a formal, verifiable mechanism.

\paragraph{Bloom's Anxiety: Rupture as Creative Misreading}
The literary theorist Harold Bloom, in his work 
\emph{The Anxiety of Influence}, provides a powerful psychological and aesthetic model for this same process. For Bloom, literary history is a Freudian drama of "strong poets" wrestling with their precursors. A new poet begins by inhabiting the semantic field of a great predecessor, their initial trajectory drawn into the powerful attractor of the master's style. But the "strong poet" cannot remain there. To achieve their own voice, they must perform a creative "misreading" (\emph{clinamen}) of the precursor's work. This is a deliberate act of rupture. The new poet intentionally perturbs the semantic field, causing the old attractor to collapse for them, so they can clear a space for their own creative work. The resulting new poetic style is a new attractor basin, a $B^\dagger(a)$, that is born from an act of loving disobedience. Bloom's framework gives us a model of rupture that is not just a passive event, but an active, generative, and even agonistic strategy for creating new meaning.

\paragraph{Gadamer's Fusion of Horizons}
In the hermeneutic tradition, Hans-Georg Gadamer described the act of understanding a historical text as a ``fusion of horizons.'' The reader has their own horizon of understanding, shaped by their present context, while the text has its own historical horizon. Meaningful interpretation is not a matter of a reader imposing their view on the text, nor of perfectly recreating the text's original meaning. Rather, it is a dialogue where the two horizons meet and fuse, creating a new, richer understanding that incorporates both. This is a beautiful description of the problem facing our speaker. They have the horizon of their past self (who understood freedom as liberty) and their present self (who understands it as grace). A full understanding requires a fusion of these two horizons.

\paragraph{The Need for a Constructive Bridge}
These philosophical traditions all point to the same need: after a rupture, a simple re-realization is not enough. A new connection, a synthesis, a fusion must be actively constructed. They describe the *what* and the *why* of this process with profound insight. However, they lack the formal tools to describe the *how*. What is the precise structure of an *Aufhebung*? What is the mechanism of a "fusion of horizons"? What is the geometry of a creative misreading?

Furthermore, successful re-realization, however, creates a new and profound philosophical problem. The speaker now holds two distinct meanings for the same name, ``freedom''—the memory of its old life in $A_{\text{liberty}}$ and its new life in $B_{\text{grace}}$. The two meanings are historically connected by the trajectory of the name, but they are semantically disconnected. There is a gap, a fault line, in the person's conceptual landscape. How can these two meanings be understood in relation to each other? How can the new be reconciled with the old?

A simple re-realization is not enough for full coherence. If the identity of the concept is to be fully restored, a bridge must be built between the old attractor and the new one. This problem—of reconciling meaning across a rupture—is not new. It has echoes throughout the history of philosophy.


This sets the stage for the formal DHoTT chapters to come. The philosophical need for a bridge to connect a concept's past and present life after a rupture is precisely what motivates the introduction of the \textbf{Healing Cell} in our full type theory. DHoTT's Healing Cell is not a metaphor; it is a formal, constructive, higher-dimensional path that can be built within the logic to connect the old attractor to the new one. Healing is the formal act of making a ruptured history whole again.

\begin{readerbox}
Let’s pause and remember where we came from. Chapter~3 taught us how to see. We learned to map signs as vectors in a semantic space, to watch them drift, settle, or spiral. A stabilised sign, we said, was just a sign that had calmed down---one that had found a home in an attractor basin.

That was a spatial insight: meaning as landing. Chapter~4 extends this by asking whether a landing can be repeated, whether meaning persists through drift and rupture. This shift from stabilization to recursive coherence is a deep moment of clarification in the book’s architecture.

\medskip
\noindent\textbf{Chapter 3: Stabilised Signs}

\begin{description}
  \item[Signs:] Vectors in the latent semantic space $\mathcal{E}$.
  \item[Trajectories:] Paths through $\mathcal{E}$, showing how meanings evolve.
  \item[Attractors:] Local basins of the semantic field $\mathcal{S}_{\tau}$ that draw signs inward.
  \item[Stabilised Signs:] Signs that have settled into a basin.
\end{description}

\noindent\textbf{Core Claim of Chapter 3:} A term is nothing other than a sign that has stabilised---its trajectory has converged into a basin and grown still enough to be called meaning. Chapter~3 is about where signs land. It does \emph{not} yet ask how long that landing endures, nor what happens if the basin itself drifts or collapses.

\medskip
\noindent\textbf{Chapter 4: Recursive Realisation as Coherence}

Chapter~4 asks: is stability enough? We now care about whether a sign’s trajectory can maintain or recover meaning over time. A trajectory is \emph{recursively realised} if it can re-enter its basin even as the semantic field drifts.

New elements of Chapter~4:
\begin{description}
  \item[Drift:] The slow motion of the basins themselves.
  \item[Flow:] The sign’s ongoing motion through the field.
  \item[Re-realisation:] The return of a sign to its basin after rupture.
  \item[$\mathcal{R}^\star$:] The condition that a trajectory maintains or regains its meaning.
\end{description}

\medskip
\noindent Thus Chapter~4 distinguishes \emph{stabilisation} (converging once) from \emph{coherence} (persisting or re-entering). Meaning that lasts must be earned again and again.

\bigskip
\begin{center}
\renewcommand{\arraystretch}{1.25}
\begin{tabularx}{\textwidth}{|X|X|X|}
\hline
\textbf{Feature} & \textbf{Chapter 3} & \textbf{Chapter 4} \\
\hline
Focus & Spatial convergence (where a sign lands) & Temporal coherence (how it keeps landing) \\
\hline
Meaning defined as & Stabilisation in a basin & Recursive re-entry into a drifting basin \\
\hline
Time & Implicit (via trajectories) & Explicit (via drift and re-realisation) \\
\hline
Identity condition & Static: is the sign in a basin? & Dynamic: does the trajectory persist or return? \\
\hline
Failure of meaning & Not addressed & Central (rupture and recovery) \\
\hline
Formality & Geometric (vector-space) & Temporal (recursive) \\
\hline
\end{tabularx}
\end{center}
\bigskip

Another way to put it:
\begin{center}
\renewcommand{\arraystretch}{1.3}
\begin{tabularx}{\textwidth}{|X|X|}
\hline
A \textbf{sign} & becomes a \textbf{stabilised sign} when it settles in a basin --- though that basin may drift or rupture in future \\
\hline
A \textbf{sign} & becomes a \textbf{name} when understood as part of a trajectory that holds through drift, rupture, and re-entry across the sequence of basins that shape its history of sense (Chapter~4) \\
\hline
\end{tabularx}
\end{center}


\medskip
We’re nearly ready to meet the ones who carry those names: the \emph{agents}. But not every trajectory earns that title. To be an agent, it’s not enough to stay. One must write. One must change the field.
\end{readerbox}





