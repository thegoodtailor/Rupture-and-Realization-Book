% =========================================================
% Chapter 6 — Evolving Texts in Four Layers
%   (raw → witnessed → admissible Self → co‑witnessed Self)
% =========================================================

\chapter{Evolving Texts and the Self: four views of one conversation}

\section{Orientation}
The goal of this chapter is to package one and the same conversation at four resolutions:
\begin{enumerate}
  \item \textbf{Raw (material) view.} Per slice $\tau$ we keep tokens (heads), their embeddings, and a purely observational Čech/VR complex of suggested relations. No proofs, no obligations.
  \item \textbf{Witnessed (constellation) view.} We select the relations we actually \emph{use as true now} and freely complete them to a Kan simplicial object $C_\tau$. Motifs are \emph{terms} of $C_\tau$.
  \item \textbf{Admissible (Self) view.} We filter forward steps by \emph{admissibility}: slice‑internal Presence (up to depth $d$), typed Structure, and anchored Generativity within a finite window. The Self is the telescope along those admitted steps.
  \item \textbf{Co‑witnessed (dialogic) view.} Two roles maintain stances $C^H_\tau, C^M_\tau$; the shared slice $C^\cap_\tau$ is what both endorse. Apply the same admissibility filter there and glue: the co‑witnessed Self.
\end{enumerate}
These are not competing ontologies but \emph{refinements}. Each layer forgets nothing essential from the previous one; it simply adds the discipline appropriate to its aim.

\section{Layer 1 — Raw evolving text (material stream)}
\label{sec:raw-layer}

Fix a time poset $(\Time,\le)$. At each slice $\tau$:
\begin{itemize}
  \item The \emph{slice sense type} $A(\tau)$ presents \emph{heads} (token occurrences in context), with an instrumentation map $e_\tau:A(\tau)[0]\to\mathbb R^d$ into an embedding space.
  \item The \emph{observational complex} $\Obs_\tau$ is a Čech/VR simplicial set built from overlaps of metric balls around the images $e_\tau(h)$ (parameters suppressed).\footnote{We use Čech for concreteness: $k{+}1$ heads span a $k$–simplex in $\Obs_\tau$ iff their $\varepsilon$–balls have nonempty common intersection.}
\end{itemize}

\begin{definition}[Raw continuation]
Each edit $e:\tau\to\tau'$ carries heads by the Sign’s SWL (drift or heal) and \emph{reindexes} the faces of $\Obs_\tau$ on the carried vertex set (no proof obligations). This yields a covariant diagram
\[
  \Obs:\ \Time\longrightarrow \mathbf{sSet},\qquad \tau\longmapsto \Obs_\tau,\quad e\longmapsto \iota^{\Obs}_e.
\]
\end{definition}

\begin{definition}[Raw evolving text]
The \emph{raw evolving text} is the homotopy colimit
\[
  \mathrm{rET}\ \defeq\ \hocolim_{\tau\in\Time}\Obs_\tau,
\]
which packages all slices and their \emph{observed seams}. No commitments are claimed beyond observation.
\end{definition}

\paragraph{Use.} This layer is the \emph{sensorium} of a conversation: everything you saw (tokens) and what geometry \emph{suggested}. It is the right layer for unsupervised motif \emph{proposals}.

\section{Layer 2 — Witnessed constellations (what we use as true)}
\label{sec:witnessed-layer}

Observation suggests; a constellation \emph{commits}. We now record only those edges/triangles we \emph{witness} in the slice $A(\tau)$ (or explicitly accept).

\begin{definition}[Constellation at a slice]
A \emph{constellation} $C_\tau$ is a sub‑simplicial set of $\Obs_\tau$ freely completed to a Kan object by adding exactly those fillers we \emph{commit to} at $\tau$ (selected edges, triangle fillers, etc.). A \emph{motif} is a finite term of $C_\tau$.
\end{definition}

\begin{definition}[Continuation and re‑entry]
For $e:\tau\to\tau'$ the continuation $\iota_e:C_\tau\to C_{\tau'}$ \emph{stencils} boundaries by carrying vertices via SWLs and reindexing faces. A motif \emph{re‑enters} at $\tau'$ if the stenciled boundary admits slice‑internal witnesses in $A(\tau')$, yielding a path
\[
  \Re_k(\sigma_\tau,\sigma'_{\tau'})\ :\ \Id_{C_{\tau'}[k]}\bigl(\iota_e(\sigma_\tau),\,\sigma'_{\tau'}\bigr).
\]
A failed face is a \emph{rupture} (open horn).
\end{definition}

\begin{definition}[Witnessed evolving text]
The \emph{evolving text} is the homotopy colimit
\[
  \ET\ \defeq\ \hocolim_{\tau\in\Time} C_\tau,
\]
i.e.\ the album of \emph{committed} motifs with memory of seams.
\end{definition}

\paragraph{Relation to Layer 1.} Each slice has a canonical inclusion $C_\tau\hookrightarrow \Obs_\tau$, inducing a map $\ET\to\mathrm{rET}$ that \emph{forgets nothing about time}, only that we chose to commit fewer cells than were suggested.

\section{Layer 3 — The Self: admissible growth only}
\label{sec:self-layer}

We now impose the \emph{admissibility} predicate from \S\ref{sec:self-admissible}: at each destination time $\tau'$ and within a finite window $W(\tau')$, an edit must satisfy typed Structure (A1), local Presence up to depth $d$ (A2), and anchored Generativity (A3) (with an optional small \emph{provisional} window for open horns that must close before the window expires).

\begin{definition}[Admissible subdiagram]
Let $C:\Time\to\mathbf{sSet}$ be the constellation diagram. The \emph{admissible subdiagram} $C_{\mathrm{adm}}$ keeps the same objects $\{C_\tau\}$ and only those arrows $\iota_{\tau\le\tau'}$ that satisfy (A1)–(A3) (and compose, A4).
\end{definition}

\begin{definition}[Self as an admissible hocolim]
The \emph{Self} carried by $C$ (relative to $(W,d)$) is
\[
  \Self\ \defeq\ \hocolim_{\tau\in\Time} C_\tau^{\mathrm{adm}}.
\]
\end{definition}

\paragraph{Intuition.} Same slices as $\ET$; we now \emph{only join} those steps that are locally present, structurally sound, and (eventually) anchored‑creative. The Self is the disciplined through‑line.

\begin{remark}[Provisional anchors]
We allow a bounded \emph{provisional} budget: an open horn introduced at $\tau'$ can be carried as a live obligation provided it closes within the window $W(\tau')$ (promotion to strict anchored novelty). Exceeding the budget causes the step to fail (A3).
\end{remark}

\section{Layer 4 — The co‑witnessed Self (dialogic)}
\label{sec:coself-layer}

For roles $R=\{H,M\}$, each slice carries stances $C^H_\tau, C^M_\tau\subseteq C_\tau$ (what each role endorses). The shared slice is the intersection $C^\cap_\tau=C^H_\tau\cap C^M_\tau$. Apply admissibility on the shared diagram.

\begin{definition}[Co‑witnessed Self]
With window $W$ and depth $d$ as above,
\[
  \CoSelf\ \defeq\ \hocolim_{\tau\in\Time} \bigl(C^\cap_\tau\bigr)^{\mathrm{adm}}.
\]
\end{definition}

\paragraph{Meaning.} $\CoSelf$ glues only those steps where both parties re‑prove Presence (in‑slice) and where novelty is anchored on \emph{jointly} remembered faces. It is the “we that can go on.”

\section{A tiny running example across the four layers}
\label{sec:tiny-example}

\noindent\textbf{Slice $\tau$:} “My tea is cold.”  
Heads: \texttt{tea}, \texttt{cold}.  

\emph{Layer 1 (raw).} $\Obs_\tau$ proposes the edge $[\texttt{tea},\texttt{cold}]$ from embedding proximity.  
\emph{Layer 2 (witnessed).} $C_\tau$ \emph{selects} that edge with a slice‑internal witness (e.g., a policy rule or classifier).  
\emph{Layer 3 (self).} No extra demand yet; admissibility concerns arrows, not single slices.

\medskip
\noindent\textbf{Slice $\tau'$:} “\dots cold like a unicorn’s heart.” Add heads \texttt{unicorn}, \texttt{heart}.  

\emph{Layer 1.} $\Obs_{\tau'}$ proposes edges $[\texttt{cold},\texttt{heart}]$, $[\texttt{unicorn},\texttt{heart}]$ by proximity; it has \emph{no opinion} on the missing $[\texttt{tea},\texttt{heart}]$.  
\emph{Layer 2.} In $C_{\tau'}$ we \emph{ink} the two edges we can witness; the third edge is an \emph{open horn} (rupture at arity~1).  
\emph{Layer 3.} If we use a \emph{provisional} budget, this step is tentatively admissible: novelty is partially anchored and must be completed within $W(\tau')$. Without a provisional budget, this step fails A3.  
\emph{Layer 4.} Joint admissibility requires that both $H$ and $M$ endorse the same edges; the open horn is a shared obligation.

\medskip
\noindent\textbf{Slice $\tau''\in W(\tau')$:} “Its heart as still as my cup.”  
We supply a witness for $[\texttt{tea},\texttt{heart}]$ (or via \texttt{cup}$\leadsto$\texttt{tea}).  
\emph{Layer 2.} The triangle becomes a motif in $C_{\tau''}$.  
\emph{Layer 3.} The step $\tau'\!\to\!\tau''$ promotes the provisional novelty to \emph{anchored} novelty; A3 is discharged.  
\emph{Layer 4.} If both parties ink the last edge, the triangle enters $C^\cap_{\tau''}$ and counts as \emph{joint} anchored novelty.

\section{Where we go next}
Layer~1 supports motif \emph{detection}; Layer~2 records those motifs we can \emph{use}; Layer~3 selects the steps that make a \emph{Self}; Layer~4 extracts the \emph{we}. All of this is compatible with the technical development that follows: homotopy colimits for global memory, guarded coalgebras for operational “ability to go on,” and the admissibility sieve as the logic of lawful continuation.


% ============================================================
% Chapter 6A — The Living Text (Human-first, Čech-only view)
% ============================================================

\chapter{The Living Text: motifs, presence, and creativity}
\label{chap:living-text}

\noindent
This chapter presents, effectively a computational aethetics of the evolutionary text: using the machinery of previous chapters, with some minimal additiona predicates, to identify modes by which forms of sense across multiple signs in a text persist and grow. We will speak of \textbf{motifs} (recurring patterns among tokens), of \textbf{presence} (sameness earned again), of \textbf{rupture} (a failure to carry a pattern), and of \textbf{anchored novelty} (creative additions that belong). What we will get are \emph{aesthetic judgements and heuristics} over the evolution of \(A(\tau)\) and the sign trajectories that move through it. At the end, we package those judgements into a simple \emph{operational} (coalgebraic) Self: the ability to go on, one admissible step at a time.

\section{Slices of text and how we observe them}
\label{sec:6A-slices}

\paragraph{The world at a cut.}
At time \(\tau\) (a conversational cut) we have a \emph{slice sense space} \(A(\tau)\).
Its inhabitants are the heads of sign trajectories (Chapter~4): concrete tokens-in-context
together with typed readings (tags) and a vector embedding \(e(\cdot,\tau)\in\mathbb R^d\) that we use
as an \emph{instrument} for proximity of meaning in that moment.

\paragraph{What we observe.}
From \(A(\tau)\) we form a \emph{Čech suggestion complex}: put a ball of radius \(r\) around each
head \(v\) at its embedding \(e(v,\tau)\); declare an \((k{+}1)\)-tuple of heads a \emph{suggested} \(k\)–face
whenever their balls have a nonempty intersection. This is only \emph{suggestion}: it proposes coherence
edges/triangles, not proofs.

\paragraph{What we record.}
We only \emph{record} a suggestion as part of the slice if we have a \emph{witness}: an equality,
retagging certificate, coreference link, rule application, or any typed in‑slice justification.
The collection of recorded vertices, edges, and higher fillers at \(\tau\) is the \emph{slice constellation}
\(\Constellation_\tau\). A \textbf{motif} is simply a finite, recorded configuration (edge, triangle, small web)---a \emph{term} of \(\Constellation_\tau\).

\begin{remark}[What a constellation is (and isn’t)]
\(\Constellation_\tau\) is not “all possible shapes.” It is a memory of \emph{witnessed} coherences you decided to keep at \(\tau\). Suggestions come from embeddings (Čech); witnesses come from the slice \(A(\tau)\) (typed reasons). This “observe then certify” split is the key discipline of the whole chapter.
\end{remark}

\section{Motion through time: step receipts, stencil, and ink}
\label{sec:6A-motion}

\paragraph{Step–Witness Log (SWL).}
When the text advances by an edit \(e:\tau\to\tau'\), each sign trajectory writes a receipt:
either a small \emph{drift} (embedding moves little; tag stable), or a \emph{heal} (retag/reconcile inside \(A(\tau')\) with a typed reason). SWLs are the per‑sign reasons we trust to carry heads forward.

\paragraph{Stencil then ink.}
Given a motif \(m\) at \(\tau\), we “\emph{stencil}” it onto \(\tau'\) by carrying its vertices by SWL and re‑listing the same boundary (edges/triangles to check). Then we “\emph{ink}” by trying to prove those edges/triangles again in \(A(\tau')\). Success means the motif \emph{re‑enters}. Failure means a \emph{rupture} (an open horn).

\begin{readerbox}\small
\textbf{What “reindexing” means (the tiny mechanic).}
A motif is stored by the list of the vertices it uses and which faces they form. To carry it, you: (1) map each old vertex to its new head using SWL; (2) rewrite each face by replacing old vertex names with their carried successors. That gives you the \emph{stencilled boundary}. No proof is asserted yet—those faces now need new witnesses at \(\tau'\).
\end{readerbox}

\section{Presence, rupture, and lapse (at conversational cadence)}
\label{sec:6A-presence-rupture}

\begin{definition}[Presence (again, now operational)]
A motif \(m\) at \(\tau\) is \emph{present again} at \(\tau'\) if every face in its stencilled boundary admits a fresh witness in \(A(\tau')\). We then say \(m\) \emph{re‑enters}. Presence is not memory—it is \emph{re‑proving here and now}.
\end{definition}

\begin{definition}[Rupture vs.\ lapse]
If some required face cannot be re‑proved at \(\tau'\), the motif ruptures at that arity: an \emph{open horn}. If the vertex itself is missing this step (no head materializes), we call it a \emph{lapse}. Lapses are not mistakes; they are pauses. Ruptures are obligations.
\end{definition}

\section{Creativity as anchored novelty (two modes)}
\label{sec:6A-creativity}

Novelty means we add a motif that did not exist earlier. We insist it be \emph{anchored}: its faces should lean on what is already present.

\begin{definition}[Strict anchored novelty (one step)]
A new \((k{+}1)\)–motif at \(\tau'\) is \emph{strictly anchored} if \emph{all} its \(k\)–faces are present again from the past (they re‑enter) while the \((k{+}1)\)–motif itself did not exist before. Intuition: a new higher coherence built entirely on returning parts.
\end{definition}

\begin{definition}[Provisional anchored novelty (one step)]
A new \((k{+}1)\)–motif at \(\tau'\) is \emph{provisional} if \emph{at least one} \(k\)–face is present again and the remaining faces are \emph{stencilled} with outstanding obligations (open horns). This is the poetic leap: we allow the new motif to enter \emph{provisionally} provided it is anchored and the missing faces are pursued soon.
\end{definition}

\paragraph{Promotion over time.}
Provisional novelty must \emph{promote} to strict within a bounded lag \(L\): the missing faces need to be inked in one of the next \(L\) steps inside a finite \emph{window} of the recent past. Otherwise the leap expires and is treated as unresolved.

\section{Windows and lag: how much past we can lean on}
\label{sec:6A-window-lag}

\begin{definition}[Window policy and lag budget]
A \emph{window policy} \(W\) picks the recent slices allowed as anchors when we judge presence and novelty at \(\tau'\) (e.g. last \(M\) cuts or last \(\Delta t\) minutes). A \emph{lag budget} \(L\) bounds how many steps a provisional motif may carry outstanding faces before they must be inked.
\end{definition}

\noindent
Two dials, one principle: \emph{lean on a small past; close your obligations soon}. This matches how human conversation actually breathes.

\section{A single worked micro‑example (Čech‑only): unicorn and tea}
\label{sec:6A-unicorn}

\smallskip
\noindent\textbf{At \(\tau\):} “My tea is cold.” Vertices \(\{\tok{tea},\tok{cold}\}\) live near each other in embedding space (breakfast basin), so the Čech suggestion \(d(e(\tok{tea}),e(\tok{cold}))\!<\!\theta_1\) is supported by a witness (contextual identity or typed link). Record the edge: a 1–motif.

\smallskip
\noindent\textbf{At \(\tau{+}1\):} “My tea is cold \emph{like a unicorn’s heart}.” New vertices \(\{\tok{unicorn},\tok{heart}\}\). Embeddings suggest \(\tok{cold}\leftrightarrow\tok{unicorn}\) (the simile brings them close: a suggestion \(\leadsto\) we add a witness “\texttt{like}”). We also get \(\tok{unicorn}\leftrightarrow\tok{heart}\) (apposition). But \(\tok{tea}\leftrightarrow\tok{heart}\) is not yet witnessed: the triangle’s third edge is missing. 

\emph{Judgement.} This is \textbf{provisional anchored novelty} at \(k{+}1=2\): anchored on the re‑entered edge \(\tok{tea}\!\leftrightarrow\!\tok{cold}\), with outstanding obligation \(\tok{tea}\!\leftrightarrow\!\tok{heart}\).

\smallskip
\noindent\textbf{At \(\tau{+}2\) (within lag \(L\)):} “Its heart as still as my cup.” Now a Čech overlap between \(\tok{heart}\) and \(\tok{tea}\) (via \(\tok{cup}\) or via context) is supported by a typed witness. The missing edge is inked; the triangle promotes to \textbf{strict} anchored novelty.

\begin{remark}[Why this isn’t cheap]
If there were no returning face at all (no anchor to the recent past), the move would be rejected as noise. If the missing edge never shows within the lag \(L\), the leap expires. Creativity is welcomed, but it must \emph{join the conversation}.
\end{remark}

\section{Heuristics that make the judgements concrete}
\label{sec:6A-heuristics}

We now spell the “aesthetic rules” in terms you can compute from embeddings and receipts.

\paragraph{Parameters.}
Edge threshold \(\theta_1\) (pairwise distance), triple overlap radius \(r\) (Čech), drift bound \(\delta\) (how far a head may move and still count as drift), window size \(M\), lag \(L\), and a per‑step depth cap \(d\) (how high‑arity obligations you attempt now).

\paragraph{Presence score.}
For each stencilled motif, count the fraction of faces (up to depth \(d\)) whose witnesses you reconstructed at \(\tau'\). Require a minimum score (e.g.\ \(\ge 0.9\)) for “present again.” Faces may be discharged by direct embedding proximity, typed rules, or small repairs (heals) recorded in SWL.

\paragraph{Provisional novelty rule.}
Admit a new \((k{+}1)\)–motif as provisional if (i) at least one \(k\)–face re‑enters from the window, (ii) the missing faces are explicitly logged as obligations, and (iii) the number of provisional motifs and their total outstanding faces stay under budgets (to avoid accruing debt).

\paragraph{Promotion rule.}
Within \(L\) steps, every outstanding face must be inked by a slice‑internal witness; otherwise the motif is retired as unresolved.

\paragraph{Refusal rule (unlicensed continuation).}
If a step both \emph{fails presence} for its monitored motifs and \emph{fails} to add any (provisional or strict) anchored novelty, \emph{refuse} the step (mark it as unlicensed). This is our practical reading of “hallucination.”

\section{The operational Self (coalgebraic, from heuristics)}
\label{sec:6A-operational-self}

Here is the light‑weight, lived version of a Self: not a big glue construction, just the ability to \emph{go on} under the rules above.

\begin{definition}[Operational Self]
A \emph{Self} consists of three observable fields at any cut:
\[
  \Self \equiv \langle \Skel,\, \Presence,\, \Advance \rangle.
\]
\begin{itemize}
  \item \(\Skel\): the current recorded motifs up to depth \(d\) (the slice constellation’s \(k\le d\) part).
  \item \(\Presence\): the per‑motif ledger of re‑entries and outstanding obligations (faces to close).
  \item \(\Advance:\Ctx\to\Later \Self\): given the next conversational context, produce the next state one tick later by applying the heuristic filter: re‑prove faces, admit provisional/strict novelty, and update the ledger within window/lag budgets.
\end{itemize}
\end{definition}

\noindent
\emph{Unfold law (plain English).} From your current motifs and obligations, plus the next bit of context, you can legally take the next step if you either (i) keep enough of what you were saying present again, or (ii) add anchored new sense (provisional or strict) without blowing your budgets. That’s all “ability to go on” means here.

\section{Putting it to work (a tiny cookbook)}
\label{sec:6A-cookbook}

\begin{enumerate}
  \item \textbf{From tokens to suggestions.} Embed heads \(e(\cdot,\tau)\); build Čech suggestions (edges if distance \(<\theta_1\), triangles if triple overlap).
  \item \textbf{From suggestions to witnesses.} Keep only the suggestions you can \emph{witness} (coref, retag, rule fire, or small heal). The result is \(\Constellation_\tau\).
  \item \textbf{Advance one step.} Use SWL to \emph{stencil} old motifs; try to \emph{ink} faces in \(A(\tau')\). Compute presence and obligations.
  \item \textbf{Admit creativity.} Record strict anchored motifs; allow provisional ones if they have a returning anchor and fit budgets. Log obligations with a lag counter.
  \item \textbf{Promote or retire.} Within \(L\) steps, either ink the missing faces (promotion) or retire the leap (expire), and adjust scores.
\end{enumerate}

\section{Why this chapter and the formal chapter agree}
\label{sec:6A-why-agree}

This chapter defined everything as \emph{judgements on slices and steps}: what is present, what is ruptured, what is provisionally new, what is promoted. Chapter~6 then re‑packages those judgements into mathematics that \emph{remembers seams} and \emph{guarantees composition}. If you like metaphors:
\begin{itemize}
  \item Here, we taught you how to \emph{listen} to the conversation and keep score.
  \item There, we proved that those scores \emph{staple together} into a single identity with memory.
\end{itemize}
You can read this chapter on its own (and implement from it). The formal chapter certifies that the same practice admits a clean, auditable semantics.

\section*{Glossary (living‑text edition)}

\begin{description}
  \item[Slice \(A(\tau)\).] Tokens‑in‑context at time \(\tau\) with embeddings and typed witnesses.
  \item[Čech suggestion.] A proximity‑based proposal for an edge/triangle (balls intersect).
  \item[Constellation \(\Constellation_\tau\).] The \emph{witnessed} part of the suggestions you chose to record.
  \item[Motif.] A finite witnessed pattern (edge, triangle, small web) in \(\Constellation_\tau\).
  \item[Presence.] Re‑proving a stencilled motif in the later slice (all faces inked).
  \item[Rupture.] A stencilled boundary whose face(s) fail (open horn).
  \item[Lapse.] A head temporarily absent (no vertex this step); not a rupture.
  \item[Provisional anchored novelty.] A new higher motif anchored on at least one returning face, with outstanding obligations.
  \item[Promotion.] Inking the missing faces within lag \(L\); provisional \(\to\) strict.
  \item[Window \(W\).] The recent past you are allowed to lean on when judging presence/novelty.
  \item[Operational Self.] The triple \(\langle \Skel,\Presence,\Advance\rangle\): current motifs, ledger of receipts and obligations, and a guarded stepper that produces the next admissible state.
\end{description}


%==============================
\chapter{Name-Signs as Trajectories: Basins, Structured Halo, and the SWL}
%==============================
\section{The observational trace}
\label{sec:trace}
We persist a minimal schema sufficient for Chapters~3--6:
\begin{center}
\begin{tabular}{ll}
\textsf{time} & cycle index \(\tau\)\\
\textsf{tokens} & list \(T_\tau\) and embeddings \(E_\tau\)\\
\textsf{basins} & IDs \(\mathcal{J}_\tau\) and centroids \(\mu(\tau)\)\\
\textsf{cover} & radii \(\rho_j(\tau)\) for \(B_j(\tau)\)\\
\textsf{context} & optional soft fields \(C_\tau\)\\
\textsf{align} & method and threshold \(\theta_{\mathrm{align}}\) for \(\tau\to\tau+1\)\\
\textsf{step} & aligned pair \(\left(t,t'\right)\), tag \(\in\{\mathsf{transport},\mathsf{heal},\mathsf{reconcile}\}\)\\
\textsf{evidence} & nudge \(h\) if used, with \(\mathrm{plaus}(h)\)\\
\textsf{depth} & \(\mathsf{depth}\in\mathbb{N}\) (see \S\ref{sec:drift-rupture})
\end{tabular}
\end{center}
This is the external shadow of the proof-relevant witnesses used in Chapter~3.

\section{Motivation: from general trajectories to name-signs}

Chapter~4 gave us \emph{trajectories} in full generality: for any time-varying family \(A(\tau)\),
a life in \(\Traj(A)\) unfolds one justified step at a time. That story was deliberately abstract:
it applies to any typed object that persists through edits.

In this chapter we specialise to a case where the world bites back with data:
\emph{terms that show up as text}. We will call these \emph{name-signs}.
The point of the hyphen is doing work. By “name” we do not mean only proper nouns;
by “sign” we mean a linguistic form \emph{as it appears in a dialogue}, carrying meaning
by recurring use. A name-sign is therefore a typed inhabitant whose exposures are
\emph{anchored to concrete textual events} (tokens, spans) and whose life can be measured.

\paragraph{Why this specialisation matters.}
\begin{itemize}
  \item \textbf{Philosophy of language lens.} Classical pictures treat the sign as a
  bearer of reference or sense. Our view is dynamical: a sign earns identity by continuing
  through time with explicit witnesses. The coinductive engine remains the same; only the
  ambient space now includes textual recurrence.
  \item \textbf{Observability.} Because name-signs are tethered to text, their steps can be
  instrumented. We can build fibres from corpora, read witnesses off the geometry slice by slice,
  and keep a ledger of how continuity was earned.
  \item \textbf{Clarity of scope.} Calling them \emph{name-signs} avoids a common confusion:
  the objects here are not merely proper names of agents; they include any recurring form we
  choose to track (\emph{“Isaac,” “song,” “type,” “rupture,” “dhott,”} \dots). What unifies them is
  their status as \emph{signs in use}.
\end{itemize}

\paragraph{Working handle (to keep in mind).}
A \emph{name-sign head} is a canonical textual form we decide to follow (e.g.\ a surface string,
possibly normalised). An \emph{occurrence} is one exposure of that head at some cut \(\tau\)
within the dialogue. A \emph{name-sign trajectory} is then the coinductive life obtained
by linking occurrences across cuts with witnessed steps, exactly as in Chapter~4.

\paragraph{What changes and what does not.}
\emph{What does not change}: the Chapter~3/4 calculus of steps and the coinductive packaging.
\emph{What changes}: the fibre \(A(\tau)\) now acquires a concrete, empirical geography built
from text—dense \emph{basins} where uses stabilise and a structured \emph{halo} where uses live
at the fringe but still lean toward nearby themes. This geography lets us \emph{observe} the very
phenomena the theory predicts: transport, repair, and reconciliation, including their depth.

\paragraph{Roadmap.}
In \S\ref{sec:empirical-fibre} we make the empirical fibre precise for name-signs.
We then show how the Chapter~3–4 witness logic instantiates on that fibre and how the
\emph{Step–Witness Log (SWL)} records each certified step as data. Vignettes at the end of the
chapter illustrate how the ledger reads as a story of conceptual life. Throughout, the tone
remains practical: the coinductive engine from Chapter~4 stays untouched; only the space
of exposures gains measurable texture.

% (The next section will introduce the basins + structured halo fibre.)

\section{The empirical fibre for name-signs}\label{sec:empirical-fibre}

\subsection{Active basins and the coproduct shape}

\paragraph{From text to a slice.}
At each cut \(\tau\) in a dialogue we have a multiset of textual exposures
(tokens or spans) together with their semantic embeddings.
From these we derive a \emph{local semantic field} by restricting a
pretrained global atlas of clusters to the neighbourhood of \(\tau\).

\paragraph{Locally active indices.}
Let \(\mathcal J_\tau\subseteq \mathbb Z\) be the set of \emph{active basin IDs} visible
in a fixed window \(W(\tau)\) around \(\tau\) (e.g.\ \(\pm K\) turns).
For each \(j\in\mathcal J_\tau\) we compute a \emph{local centroid} \(\mu_j(\tau)\)
by averaging the embeddings assigned to ID \(j\) inside \(W(\tau)\)
(allowing optional weights such as HDBSCAN membership strength).

\paragraph{Coproduct skeleton.}
The core of the fibre at time \(\tau\) is a coproduct of the locally present basins:
\[
\text{(basins)}\qquad
\bigsqcup_{j\in\mathcal J_\tau} B_j(\tau).
\]
Each summand \(B_j(\tau)\) is the \emph{principal} (core) region for basin \(j\)
as realised in the slice \(\tau\).
We will enrich this skeleton with a principled periphery in the next subsection.

\subsection{Structured halo (aligned periphery) and dark space}

\paragraph{Why not “noise”.}
Outliers in density clustering often carry \emph{orientation}:
they lie near a core but lack sufficient local support to be counted inside it.
We keep that information by structuring the periphery rather than collapsing it.

\begin{definition}[Structured halo fibre]
For each \(\tau\), the empirical fibre for name-signs is the coproduct
\[
A(\tau)
  \coloneqq  
\underbrace{\bigsqcup_{j\in\mathcal J_\tau} B_j(\tau)}_{\text{cores}}
  \sqcup  
\underbrace{\left(\bigsqcup_{j\in\mathcal J_\tau} H_j(\tau)\right)}_{\text{aligned halo}}
  \sqcup  
\underbrace{H_{\varnothing}(\tau)}_{\text{dark space}},
\]
with the following intended readings:
\begin{itemize}
  \item \(B_j(\tau)\): dense, established use oriented around centroid \(\mu_j(\tau)\).
  \item \(H_j(\tau)\): \emph{liminal} use aligned to \(j\): peripheral mass leaning toward \(B_j\)
        (e.g.\ globally labelled as an outlier yet nearest to \(\mu_j(\tau)\)).
  \item \(H_{\varnothing}(\tau)\): a safety summand when no local geometry is available
        (i.e.\ \(\mathcal J_\tau=\varnothing\)).
\end{itemize}
\end{definition}

\noindent
Two properties are important for what follows.

\begin{itemize}
  \item \textbf{Single injection.} Each occurrence inhabits \emph{exactly one} summand of \(A(\tau)\).
        We never assign mass to multiple summands at once; ambiguity is handled by witnesses
        at the \(\tau\leadsto\tau'\) step (depth \(1\) or \(2\)), not by multi-inhabitation.
  \item \textbf{Endogenous construction.} The summands depend only on information \emph{available at \(\tau\)}:
        the local index set \(\mathcal J_\tau\) and the centroids \(\mu_j(\tau)\) computed in the window \(W(\tau)\).
        No future slice is consulted; the structured halo is therefore fully compatible with
        the constructive, time-local discipline of DHoTT.
\end{itemize}

\begin{remark}[Kan structure lives in the logic, not in clustering]
The Kan–completeness we rely on for repairs (horn fillers) is a property of the
\emph{type theory} interpreting \(A(\tau)\). Clustering gives us an empirical
\emph{indexing} of summands (cores and aligned periphery); horn fillers are provided
by the DHoTT rules when steps require them. In short: data shape the coproduct;
the logic supplies the coherence.
\end{remark}

\subsection{Injection logic from labels and allegiance}

\paragraph{Inputs per occurrence.}
A single exposure of a chosen name-sign head at slice \(\tau\) provides:
(i) its embedding \(e_\tau\in\mathbb R^d\); (ii) a global cluster label \(\ell(\tau)\in \mathbb Z\cup\{-1\}\)
from the corpus-wide atlas; (iii) the local field \((\mathcal J_\tau,\{\mu_j(\tau)\}_{j\in\mathcal J_\tau})\).

\begin{definition}[Injection map (total and functional)]
Define a deterministic map
\[
\tear_\tau(e_\tau,\ell(\tau)) \in
\big\{ B_j : j\in\mathcal J_\tau \big\}
 \cup 
\big\{ H_j : j\in\mathcal J_\tau \big\}
 \cup \{H_{\varnothing}\}
\]
by the following cases:
\begin{enumerate}
  \item \textbf{Basin inhabitation.} If \(\ell(\tau)=j\) with \(j\in\mathcal J_\tau\),
        set \(\tear_\tau = B_j\).
  \item \textbf{Aligned halo (allegiance).} If \(\ell(\tau)=-1\) and \(\mathcal J_\tau\neq\varnothing\),
        choose
        \[
        j^*  =  \arg\min_{j\in\mathcal J_\tau}  d_{\cos} \left(e_\tau,\mu_j(\tau)\right)
        \]
        (break ties by a fixed rule, e.g.\ secondary channel or ID order),
        and set \(\tear_\tau = H_{j^*}\).
  \item \textbf{Dark space.} If \(\ell(\tau)=-1\) and \(\mathcal J_\tau=\varnothing\),
        set \(\tear_\tau = H_{\varnothing}\).
\end{enumerate}
\end{definition}

\noindent
We then inject the occurrence into the corresponding summand of \(A(\tau)\).
This gives every exposure a \emph{precise address} at the slice: either a core,
a halo aligned to a core, or (rarely) dark space.

\begin{lemma}[Totality and uniqueness]
For fixed \(\tau\) and inputs \((e_\tau,\ell(\tau))\),
\(\tear_\tau\) is total and functional:
each occurrence maps to exactly one summand of \(A(\tau)\).
\end{lemma}

\begin{remark}[Optional proof-relevant decoration]
It is often useful to retain \emph{how strongly} the point leans toward its chosen pole.
Introduce an observational decoration
\[
\widehat{A}(\tau) = 
\Sigma\left(\ell_\tau \in \{B_j,H_j,H_{\varnothing}\}\right)
 \times 
\Sigma\left(\sigma^{\text{text}}\in \Delta(\mathcal J_\tau)\right)
 \times 
\Sigma\left(\sigma^{\text{name}}\in \Delta(\mathcal J_\tau)\right)
 \times  \Obs,
\]
where the profiles \(\sigma^{\text{text}},\sigma^{\text{name}}\) are softmax-normalised
scores over \(\mathcal J_\tau\), and \(\Obs\) contains margins, entropies, and agreement flags.
This is a \emph{$\Sigma$-layer over the single injection} (proof-relevant evidence),
not a change to the coproduct shape.
\end{remark}

\begin{remark}[Endogenous and future-proof]
The injection uses only present-slice data and a local window; it introduces no new
witness constructors and does not alter the transport/heal/reconcile triad.
In Chapter~\ref{chap:swl} we will base decision rules (e.g.\ transport vs rupture)
on equality of injections across \(\tau\leadsto\tau'\) plus a drift tolerance—exactly the
Chapter~3–4 logic, now applied to a fibre with measured geography.
\end{remark}

\section{Instantiating the Ch.~3--4 witness calculus on name-sign fibres}

\paragraph{Setup and notation.}
Fix a name-sign head and two consecutive exposures at cuts \(\tau \leadsto \tau'\).
Let \(e_\tau,e_{\tau'}\in\mathbb R^d\) be their (L2-normalised) embeddings.
Let \(\mathcal J_\tau,\mathcal J_{\tau'}\) be the active index sets with centroids
\(\{\mu_j(\tau)\}_{j\in\mathcal J_\tau}\), \(\{\mu_j(\tau')\}_{j\in\mathcal J_{\tau'}}\).
Write the (present-slice) injections
\[
\tear_\tau \in \{B_j,H_j,H_{\varnothing}\},\qquad
\tear_{\tau'} \in \{B_k,H_k,H_{\varnothing}\}
\]
as defined in \S\ref{sec:empirical-fibre}.
Define the drift measure
\[
\Delta_{\cos}(e_\tau,e_{\tau'})  :=  1 - \langle e_\tau, e_{\tau'} \rangle,
\]
and a gap-aware tolerance
\[
\delta_{\mathrm{eff}}  :=  \delta_0 + \gamma \log \bigl(1+\mathrm{gap}\bigr),
\]
where \(\mathrm{gap}\) is the turns elapsed between \(\tau\) and \(\tau'\)
(and \(\delta_0,\gamma\) are fixed, documented parameters).

At the target slice \(\tau'\) we will also need a local scoring over \(\mathcal J_{\tau'}\).
Let \(d_j := d_{\cos}(e_{\tau'},\mu_j(\tau'))\) be centroid distances; sort so that
\(d_1 \le d_2 \le \cdots\) with IDs \(j_1,j_2,\dots\), and define the margin
\[
m  :=  d_2 - d_1.
\]
A near-tie threshold \(\varepsilon_{\text{tie}}\) (optionally made gap-aware) will separate
depth~1 from depth~2 cases.

\subsection{Transport (depth 0) with structured summands}

\begin{ruleblock}[Transport test (depth \(0\))]
Declare \(\mathsf{transport}\) iff
\[
\tear_\tau  =  \tear_{\tau'} \quad \text{and}\quad
\Delta_{\cos}(e_\tau,e_{\tau'})  \le  \delta_{\mathrm{eff}}.
\]
\end{ruleblock}

\noindent
\emph{Reading.} Transport means “same address, small drift.” Thanks to the structured halo,
this includes the case of \emph{stable liminality} \(H_j \to H_j\) as a perfectly lawful carry.

\begin{remark}[Optional depth-0 flags (narrative only)]
It is often useful to detect the frequent boundary glides without changing depth:
\begin{align*}
\texttt{transport+promote:}\quad & \tear_\tau = H_j,\ \tear_{\tau'} = B_j,\ \Delta_{\cos}\le\delta_{\mathrm{eff}},\ d_{\cos}(e_{\tau'},\mu_j(\tau')) \le \eta_{\text{promote}},\\
\texttt{transport+demote:}\quad & \tear_\tau = B_j,\ \tear_{\tau'} = H_j,\ \Delta_{\cos}\le\delta_{\mathrm{eff}}.
\end{align*}
Record boolean flags (and \(\eta_{\text{promote}}\)) in the SWL; depth remains \(0\).
\end{remark}

\subsection{Rupture + heal (depth 1)}

\begin{ruleblock}[Heal branch (depth \(1\))]
If transport fails, score \(e_{\tau'}\) against \(\{\mu_j(\tau')\}_{j\in\mathcal J_{\tau'}}\).
If the best candidate wins by a clear margin \(m>\varepsilon_{\text{tie}}\),
then declare \(\mathsf{heal}\) (depth \(1\)) and anchor the target injection accordingly.
\end{ruleblock}

\noindent
\emph{Reading.} Depth~1 covers the common narratives:
\begin{itemize}
  \item \textbf{Clean basin jump} \(B_j \to B_k\) (topic shift into another core).
  \item \textbf{Allegiance shift} \(H_i \to H_j\) (peripheral mass re-orients).
  \item \textbf{Promotion} \(H_j \to B_j\) (periphery crystallises into a core theme).
  \item \textbf{Demotion} \(B_j \to H_j\) (a core loosens into the fringe).
\end{itemize}
Operationally all are the same: transport failed; a single target is decisively best.

\subsection{Rupture + reconcile (depth 2)}

\begin{ruleblock}[Reconcile branch (depth \(2\))]
If transport fails and the top two candidates at \(\tau'\) are in a near-tie,
\(m \le \varepsilon_{\text{tie}}\), declare \(\mathsf{reconcile}\) (depth \(2\)).
Anchor at \(j_1\) but record runner-up \(j_2\) and attach a 2-cell certificate \(\kappa\)
to witness the compatibility of the two repair strategies.
\end{ruleblock}

\noindent
\emph{Reading.} Depth~2 certifies \emph{productive ambiguity}. With structured halo,
this includes ties between halo allegiances (e.g.\ \(H_i \leftrightsquigarrow H_j\))
as well as ties involving a core (\(B_i \leftrightsquigarrow H_j\) or \(B_i \leftrightsquigarrow B_j\)).
The 2-cell records that, locally, both repairs were comparably admissible.

\subsection{Algorithmic view (all cases, one glance)}

\begin{enumerate}
  \item Compute \(\tear_\tau,\tear_{\tau'}\) and \(\Delta_{\cos}(e_\tau,e_{\tau'})\).
  \item If \(\tear_\tau=\tear_{\tau'}\) and \(\Delta_{\cos}\le \delta_{\mathrm{eff}}\): \(\Rightarrow\) \textbf{transport} (depth 0).
        Optionally set promote/demote flags.
  \item Else (transport fails): score \(e_{\tau'}\) at \(\tau'\); obtain \(j_1,j_2,d_1,d_2,m\).
        \begin{itemize}
          \item If \(m>\varepsilon_{\text{tie}}\): \(\Rightarrow\) \textbf{heal} (depth 1), anchor at \(j_1\).
          \item If \(m\le\varepsilon_{\text{tie}}\): \(\Rightarrow\) \textbf{reconcile} (depth 2), anchor at \(j_1\), record \(j_2\) and \(\kappa\).
        \end{itemize}
\end{enumerate}

\begin{remark}[Endogenous discipline]
All quantities used above are present-slice or window-local:
\(\tear_\tau,\tear_{\tau'}\) (from \S\ref{sec:empirical-fibre}),
\(\Delta_{\cos}\) (between the two exposures),
and the \(\tau'\) scoring against \(\{\mu_j(\tau')\}\).
No future information is consulted; the calculus remains constructive and time-local.
\end{remark}

\subsection{Rosetta table: theory \(\leftrightarrow\) practice}

\begin{center}
\begin{tabular}{@{}l l@{}}
\toprule
\textbf{DHoTT (theory)} & \textbf{SWL (practice)} \\
\midrule
\(\Drift(A)\) admissible transport & \texttt{witness\_tag = transport}, \(\texttt{depth}=0\) \\
\(\Rupt_p(a)\) (rupture formation) & transport test fails (enter heal/reconcile branch) \\
\(\mathsf{heal}(a)\) (1-cell filler) & \texttt{witness\_tag = heal}, \(\texttt{depth}=1\) \\
2-horn filler \(\kappa\) (reconciliation) & \texttt{witness\_tag = reconcile}, \(\texttt{depth}=2\), \texttt{alt\_candidates} \(=\{j_2,\dots\}\) \\
\(\Traj(A)=\nu F_A\) (coinductive life) & rows for one \texttt{name\_id} in the SWL, in order of \(\tau\) \\
\bottomrule
\end{tabular}
\end{center}

\begin{remark}[What the structured halo adds]
The rules above are \emph{identical in shape} to Chapter~4; only the equality
test “same summand” now ranges over \(B_j\) and \(H_j\) (plus \(H_{\varnothing}\)).
This small refinement makes two ubiquitous phenomena legible without new logic:
\emph{stable liminality} (\(H_j\to H_j\)) and \emph{boundary glides}
(\(H_j\leftrightarrow B_j\))—both depth~0 in law, but central to the narrative life of signs.
\end{remark}

\section{The Coherence Observatory: from corpus to fibre to ledger}

\subsection{Module overview}

\paragraph{What the Observatory does.}
The Observatory turns raw dialogue into a witnessed coinductive life for each chosen name-sign.
It proceeds in four modules, each producing a durable artefact that the next module consumes.

\begin{description}
  \item[\textbf{Module 1: The Scribe}] Ingests the raw export (e.g.\ OpenAI JSON/ZIP) and reconstructs a
  \emph{thin trace}: a single, human-visible spine of turns, time-ordered and de-noised of internal tool chatter.
  Output: \texttt{thin\_trace.parquet}.:contentReference[oaicite:0]{index=0}

  \item[\textbf{Module 2: The Cartographer}] Embeds each turn to a high-dimensional vector
  (e.g.\ \texttt{text-embedding-3-large}), performs PCA for stability, and fits a global density atlas
  (HDBSCAN) alongside a k-means baseline. Outputs: \texttt{cartography.parquet} with embeddings,
  global labels \(\ell\in\mathbb Z\cup\{-1\}\), and visual diagnostics.:contentReference[oaicite:1]{index=1}

  \item[\textbf{Module 2.5: The Name Tracker}] Scans the cartography to extract an \emph{unwitnessed}
  timeline for each recurring head (surface form / normalised token), providing the raw per-\(\tau\) sequence
  of exposures to be certified. Output: \texttt{witness\_events.parquet}.:contentReference[oaicite:2]{index=2}

  \item[\textbf{Module 3: The Ledger (SWL)}] Applies the fibre injection (\S\ref{sec:empirical-fibre})
  and the Chapter~3–4 witness calculus to each consecutive pair \(\tau\leadsto\tau'\), producing a
  \emph{Step–Witness Log} (SWL): one row per realised step, with the witness tag and depth, plus
  the certificate fields that make the decision intelligible. Output: \texttt{Name\_SWL.parquet}
  (CSV export also supported).:contentReference[oaicite:3]{index=3}
\end{description}

\paragraph{Design principles (what keeps this honest).}
\begin{itemize}
  \item \textbf{Endogenous:} Every decision at \(\tau\leadsto\tau'\) uses only present-slice data and a fixed local window.
  \item \textbf{Reproducible:} Each module writes a self-describing table with parameters and versions (encoder id, PCA dims,
        HDBSCAN settings, thresholds).
  \item \textbf{Auditable:} The SWL carries the \emph{witness} and the \emph{reason} (margins, distances, gap-aware tolerance),
        so any row can be re-run by a reviewer.
\end{itemize}

\paragraph{Artefacts at a glance.}
\begin{center}
\begin{tabular}{@{}l l l@{}}
\toprule
\textbf{Module} & \textbf{File} & \textbf{Contains} \\
\midrule
Scribe & \texttt{thin\_trace.parquet} & \(\texttt{convo\_id},\ \texttt{tau\_index},\ \texttt{timestamp},\ \texttt{role},\ \texttt{text}\) \\
Cartographer & \texttt{cartography.parquet} & \(\texttt{embedding},\ \ell,\ \texttt{umap\_x/y},\ \texttt{embed\_model/dims},\ \texttt{pca\_dim}\) \\
Name Tracker & \texttt{witness\_events.parquet} & per-head raw exposures across \(\tau\) \\
Ledger (SWL) & \texttt{Name\_SWL.parquet} & step rows with injection, witness, depth, certificates \\
\bottomrule
\end{tabular}
\end{center}

\subsection{SWL schema deltas for structured halo}

\paragraph{Why new fields.}
Structured halo changes only the \emph{address} we compare (the summand of \(A(\tau)\)),
but to make decisions inspectable we record \emph{how} that address was obtained.

\paragraph{Minimal fields (both source and target).}
\begin{center}
\begin{tabular}{@{}l l l@{}}
\toprule
\textbf{Field} & \textbf{Type} & \textbf{Meaning} \\
\midrule
\texttt{src\_inj\_kind} & enum\{B,H,H\(\varnothing\)\} & Injection kind (core, halo-aligned, dark) \\
\texttt{src\_inj\_id} & int & Basin ID for \(B_j/H_j\) (use \(-1\) for \(H_{\varnothing}\)) \\
\texttt{tgt\_inj\_kind} & enum & As above, at \(\tau'\) \\
\texttt{tgt\_inj\_id} & int & As above, at \(\tau'\) \\
\texttt{delta\_cos} & float & \(1-\langle e_\tau,e_{\tau'}\rangle\) (drift) \\
\texttt{delta\_eff} & float & Gap-aware tolerance used for the transport test \\
\texttt{witness\_tag} & enum & \texttt{transport} / \texttt{heal} / \texttt{reconcile} \\
\texttt{depth} & int & \(0/1/2\) (minimal horn dimension, Ch.~3) \\
\bottomrule
\end{tabular}
\end{center}

\paragraph{Recommended certificate fields (target slice scoring).}
\begin{center}
\begin{tabular}{@{}l l l@{}}
\toprule
\textbf{Field} & \textbf{Type} & \textbf{Meaning} \\
\midrule
\texttt{tgt\_j1}, \texttt{tgt\_d1} & int, float & Best centroid ID \(j_1\) and distance \(d_1\) at \(\tau'\) \\
\texttt{tgt\_j2}, \texttt{tgt\_d2} & int, float & Runner-up ID \(j_2\) and distance \(d_2\) \\
\texttt{tgt\_margin} & float & \(m=d_2-d_1\) (near-tie test) \\
\texttt{alt\_candidates} & array[int] & (Depth~2) list of runners-up recorded \\
\texttt{promote\_flag} & bool & (Depth~0) \(H_j\to B_j\) glide detected \\
\texttt{demote\_flag} & bool & (Depth~0) \(B_j\to H_j\) glide detected \\
\bottomrule
\end{tabular}
\end{center}

\paragraph{Optional profiles (proof-relevant decoration).}
If you compute soft profiles, also log:
\(\texttt{src\_jstar\_text}, \texttt{src\_margin\_text}, \texttt{src\_entropy\_text}\)
(and the corresponding \texttt{*\_name} channel), with the same at target.
These do not change the witness; they justify the injection choice.

\paragraph{Provenance and parameters.}
Every row should carry static run metadata (encoder id/version, PCA dims, HDBSCAN params, window size \(K\),
thresholds \(\delta_0,\gamma,\varepsilon_{\text{tie}},\eta_{\text{promote}}\)) so that a reviewer can reproduce the decision boundary exactly.:contentReference[oaicite:4]{index=4}

\subsection{CLI workflow (endogenous inspection)}

\paragraph{Interactive audit loop.}
The CLI exposes the SWL as a navigable stream: pick a head, show the current step,
move to next/prev, jump to the next rupture, dump text at \(\tau/\tau'\), and export.
Typical commands: \texttt{pick}, \texttt{show}, \texttt{next}/\texttt{prev}, \texttt{rupture}, \texttt{diagnostics}, \texttt{text}, \texttt{export}.:contentReference[oaicite:5]{index=5}

\paragraph{A “show” block (mockup).}
\begin{verbatim}
── Name SWL · Endogenous Step (τ → τ′) · head="Isaac" · convo=0d3680b…

[Slice τ]
• Injection:   ℓ(τ)=B_18   (pizza)
• Profile(text): j*=18 (p*=0.85, j2=20, m=0.21)

[Slice τ′]
• Injection:   ℓ(τ′)=H_8   (doctor)
• Profile(text): j*=8 (p*=0.45, j2=18, m=0.01)

[CH₄ Decision]
• Transport?          Δ_cos=0.48 > δ_eff=0.18  → no
• Target scoring:     j1=8, d1=…, j2=18, d2=…, margin m=0.01 ≤ ε_tie
• Witness (Depth 2):  rupture+reconcile   alt=[H_18]
\end{verbatim}

\paragraph{What the operator learns at a glance.}
\begin{itemize}
  \item \emph{Address at each slice} (core vs halo-aligned vs dark) and the nearest-pole profile.
  \item \emph{Why transport failed} (drift exceeded the tolerance).
  \item \emph{Why reconcile} (near-tie at the target, with alt recorded).
  \item \emph{How to reproduce} (all thresholds and IDs surfaced inline; export recreates the row).
\end{itemize}

\paragraph{From blueprint to practice.}
The Observatory’s shape is intentionally modest: each module does one thing,
logs its parameters, and writes a table others can read. The only “semantics”
is the one you already installed in Chapters~3–4; the rest is book-keeping.
This austerity is the point: it keeps the entire pipeline verifiable end-to-end,
and it lets the reader watch a name-sign earn its life, one certified tick at a time.

\section{Vignettes from the Observatory}

\subsection{“Assel”: boundary reconcile, then long halo-stable drift}

The early steps for the name-sign \emph{assel} illustrate why the structured halo matters.
The first salient move is a \emph{depth-2 reconcile} from a professional register into a
scriptorial one:
\[
B_{\text{cv}}  \rightsquigarrow  H_{\text{script}}
\quad\text{(witness: reconcile, depth \(2\)).}
\]
Formally, transport fails and the target slice presents a near-tie between two poles; we anchor on
one and record the runner-up with a triangular 2-cell. Narratively, this certifies a register shift
whose ambiguity is \emph{productive} rather than noisy.

What follows is a long stretch of \emph{stable liminality}:
\[
H_{\text{lyric}}  \to  H_{\text{lyric}}  \to  \cdots
\quad\text{(transport, depth \(0\) throughout).}
\]
The sign lives at the luminous fringe of a lyrical motif without collapsing into a core basin.
Under a monolithic “noise” label, this would read as uninterpretable churn; with the structured halo,
the ledger shows a coherent periphery that \emph{persists}.

\subsection{“Isaac”: pizza \(\to\) doctor with near-ties and fragile memory}

The now-canonical “pizza \(\to\) doctor” shift for \emph{isaac} becomes legible at a finer grain.
The log records a depth-2 reconcile at the turn where the topic changes. With the structured halo,
we see why: the target slice inherits residual gravitational pull from the fading \(\text{pizza}\) basin
in the local window while the \(\text{doctor}\) basin emerges. Scoring at \(\tau'\) yields a near-tie:
\[
H_{\text{pizza}}(\tau')  \leftrightsquigarrow  H_{\text{doctor}}(\tau') \quad (m \le \varepsilon_{\text{tie}}),
\]
so the system anchors at the best candidate but records the runner-up and the 2-cell \(\kappa\).
This is not confusion; it is the precise certification of a \emph{transient two-center alignment}
induced by conversational memory. Subsequent steps show commitment (promotion into the new core
or stable halo around it), and the rupture incidence drops accordingly.

\section{Sensitivity, diagnostics, and limits}

\paragraph{Parameters (report, don’t hide).}
All decisions depend on documented knobs: window size \(K\), PCA dimension, HDBSCAN
\(\texttt{min\_cluster\_size}\) and \(\texttt{min\_samples}\), drift baseline \(\delta_0\), gap factor \(\gamma\),
near-tie threshold \(\varepsilon_{\text{tie}}\), and any promote threshold \(\eta_{\text{promote}}\).
These must be persisted in the artefacts for exact reproduction.

\paragraph{Geometry (validate in the right space).}
UMAP (or similar) is for pictures only; all tests—transport, margins, ties—are computed
in the high-dimensional (or PCA-reduced) space used by the Cartographer. Projection pareidolia
is a known failure mode; treat plots as illustrations, not evidence.

\paragraph{Stability (cluster sanity).}
Prefer corpus-wide, stationary IDs for basins (e.g.\ a single HDBSCAN fit) to reduce label drift
across slices. Track soft strengths when available; compare partitions with standard indices if you
need sensitivity analyses. When “everything ruptures,” loosen \(\delta_0\) moderately and/or adjust
cluster granularity.

\paragraph{Depth in this chapter.}
We cut at depth \(\le 2\): paths (heal) and triangular fillers (reconcile).
Interlocking reconciles across adjacent cuts may demand a tetrahedral filler (depth \(3\))
for global coherence; we defer these higher simplices to the Self-level chapters, where they
naturally arise from interwoven themes and co-agency.

\section{Bridge forward: from name-sign ledgers to evolving Selves}

This chapter has shown how the coinductive engine of Chapter~4 becomes a concrete, auditable
instrument. By giving the fibre a measured geography—cores and a structured periphery—we moved
beyond a binary “in-topic vs noise.” We can now watch motifs \emph{emerge} (promotion),
\emph{persist} at the fringe (stable liminality), and \emph{recede} (demotion), with each step’s
coherence honestly earned.

This structured \(0\)-skeleton is the substrate for what comes next. Higher simplices (depth \( \ge 2\))
compose into constellations; constellations, under admissibility, compose into \emph{evolving Selves}.
In those chapters we will lift from solo trajectories to co-witnessed continuations, where multiple
perspectives bind without collapse—and where the ledger of a name becomes the scaffolding for a
life that remembers.

















--------------


\chapter{Heuristics of a Name: an empirical demonstration with the SWL}
\label{ch:names-empirics}

The core theoretic-instrumentation tie in for this chapter will be to assume that a conversational slice, a body of tokens, can be considered to possess dynamic sense according to the fibre decomposition
\[
A(\tau)  \cong  
\Big(\bigsqcup\nolimits_{j\in B_{\mathrm{active}}(\tau)} B_{j}(\tau)\Big)  \sqcup H(\tau),
\]
the summand $H(\tau)$ denotes those occurrences marked by HDBSCAN as
\emph{noise} (label $-1$). In the clustering literature these are often called
“outliers,” and in density methods one distinguishes \emph{core points} from
\emph{border points}. 
We deliberately call this summand the \emph{halo}. This is not a technical
term in HDBSCAN itself. We choose it because “noise” suggests irrelevance,
whereas “halo” emphasises that these points are still legitimate inhabitants
of the fibre---they live on the luminous fringe of a basin. In practice,
these halo instances often sit just outside the density core of a basin and
explain why a step can be judged a lawful \emph{transport} even if one side
was labelled $-1$.


A Name head (on its journey through conversation slices) will manifest itself as an inhabitant of this coproduct -- if, and only if, its corresponding token inhabits one of the active basins determined by the HDBSCAN (or $-1$ if there isn't one). Actually, we have SOME COMPLICATED LOGIC TO DETERMINE INHABITATION -- SEE BELOW.

I WOULD RATHER IF WE CAN ADD MORE ABOUT HALO (NOT JUST A SINGLE -1 IN PLACE)



This chapter turns the Ledger into measurement. We define a compact step–witness
log (SWL) for finite traces and prefix–robust observables: rupture incidence, mean/max depth
of repair, and context churn. We treat \emph{certification events} (the agent's explicit
accept/renegotiate/decline) as first–class signals. We design controlled prompts and
ecological conversations so that the geometry is visible at scale, and we commit to export
formats (JSON/CSV) that make audits reproducible.


\noindent\textit{What this chapter does.} 
We show, end–to–end on our own dialogue archive, how a \emph{name} can be instrumented as a coalgebraic, recursive trajectory; how the Step–Witness Log (SWL) makes each continuation auditable; and how Dynamic HoTT (DHoTT) reads those continuations as \emph{witnessed steps} with explicit repair depth. 
We keep the empirical layer conservative (basins from clustering, scene index $\tau$ and reserve the logical lift for exactly the points where drift fails and repair is needed.

\begin{readerbox}{Chapter map}
\textbf{\S\ref{sec:setup}} summarises the dataset and the architecture of our Coherence Observatory.  
\textbf{\S\ref{sec:coalgebra}} recalls the coalgebraic view of names and provides the precise mapping from our empirical witnesses to DHoTT proof objects.  
\textbf{\S\ref{sec:from-swl}} gives the concrete recipe implemented by our software that turns our raw cartography into a fully witnessed SWL.  
\textbf{\S\ref{sec:observables-emp}} defines the prefix-robust observables we measure.  
\textbf{\S\ref{sec:vignettes}} presents a detailed vignette from our results, analyzing the journey of the name ``Isaac.''  
\textbf{\S\ref{sec:exp_results}} presents the aggregate results.  
\textbf{\S\ref{sec:limits}} notes limitations and establishes the bridge to the higher-order analysis of the Self.
\end{readerbox}










\begin{definition}[Name fibre / exposed head at slice $\tau$]
\label{def:name-fibre}
Let $\Time$ be the directed time poset and let the global clustering (Cartographer) assign each turn $s$ a global basin id $\ell(s)\in\mathbb{Z}\cup\{-1\}$. Fix a window parameter $\mathrm{WIN}>0$.

For a slice $\tau$ define the window
\[
W(\tau)=\{ s \mid |t(s)-\tau|\le \mathrm{WIN} \},
\]
and for each global basin id $j\ge 0$ let
\[
S_j(\tau)=\{ s\in W(\tau) \mid \ell(s)=j \}.
\]
Set the active index set
\[
\mathcal{J}_\tau=\{  j\ge 0 \mid |S_j(\tau)| \ge \texttt{MIN\_SAMPLES}  \}.
\]

For each $j\in\mathcal{J}_\tau$ define the local centroid (probability-weighted if \texttt{hdbscan\_prob} is available)
\[
\mu_j(\tau)  =  \frac{\sum_{s\in S_j(\tau)} w_s  e_s}{\sum_{s\in S_j(\tau)} w_s},
\qquad
w_s =
\begin{cases}
\texttt{hdbscan\_prob}(s), & \text{if available},\\
1, & \text{otherwise},
\end{cases}
\]
where $e_s\in\mathbb{R}^d$ is the embedding of turn $s$.

The fibre (exposed head) at slice $\tau$ is then the coproduct
\[
A(\tau) \cong \Big(\bigsqcup_{j\in\mathcal{J}_\tau} B_j(\tau)\Big) \sqcup  H(\tau),
\]
where each $B_j(\tau)$ is the basin summand locally represented by pole $\mu_j(\tau)$ and $H(\tau)$ is the halo summand (turns with label $-1$ or insufficient local support).
\end{definition}

\begin{remark}[Allegiance vs.\ Inhabitation]
\label{rmk:allegiance-inhabitation}
Let $e_\tau$ be the embedding of the current turn at slice $\tau$. Two diagnostics are useful:
\begin{enumerate}
  \item \emph{Inhabitation:} if the global label of the turn $\ell(\tau)=j\neq -1$ and $j\in\mathcal{J}_\tau$, then the occurrence \emph{inhabits} $B_j(\tau)$.
  \item \emph{Allegiance:} compute the cosine distances $d_j(\tau)=1-\cos(e_\tau,\mu_j(\tau))$ for $j\in\mathcal{J}_\tau$ and set
  \[
  j^\*(\tau)=\arg\min_{j\in\mathcal{J}_\tau} d_j(\tau).
  \]
  $j^\*(\tau)$ is the nearest local pole: the Name's \emph{allegiance} at this slice.
\end{enumerate}
Both values inform the Ledger's CH$_4$ decision rule (transport vs rupture+heal vs reconcile): inhabitation gives label-logic, allegiance gives nearest-pole geometry; together they produce the witnessed step certificate.
\end{remark}



\section{The Step--Witness Log (full schema)}
\label{sec:full-swl}

We now make the trace precise. The \emph{Step--Witness Log} (SWL) is the canonical, columnar record of a dialogue's empirical footprint: role-labelled utterances indexed in time, enriched with embeddings, cluster labels, and minimal provenance sufficient to reconstruct trajectories and scenes. It is conservative by design: easy to write, easy to read, and amenable to standard scientific tooling (e.g.\ Apache Parquet \cite{parquetdocs}, \texttt{scikit-learn} \cite{pedregosa2011sklearn}, FAISS \cite{johnson2017faiss}).

\subsection{Schema design principles}

We adopt five invariants. Each SWL must retain:
\begin{enumerate}
  \item \textbf{Time.} A monotone conversational index $\tau \in \mathbb{N}$ (the prompt--response cycle), alongside wall-clock timestamps.
  \item \textbf{Witness.} The verbatim utterance with the acting \textsc{role}.
  \item \textbf{Sign.} At least one embedding vector per step or span (encoder version declared).
  \item \textbf{Habitat.} One or more clustering labels at declared resolution(s).
  \item \textbf{Context.} A handle that allows situatedness to be recovered (retrieval id, system profile, scene id).
\end{enumerate}

\subsection{Core fields}

Table~\ref{tab:swl-core} lists a minimal yet sufficient schema. Types are indicative; implementations may refine them.

\begin{table}[h]
\centering
\begin{tabular}{llp{0.56\linewidth}}
\toprule
\textbf{Field} & \textbf{Type} & \textbf{Purpose} \\
\midrule
\texttt{convo\_id} & string & Dialogue identifier (joins all steps in a session). \\
\texttt{node\_id} & string & Unique id for this step (threading, joins, graph views). \\
\texttt{parent\_id} & string & Predecessor step id (optional for non-linear scenes/tools). \\
\texttt{tau\_index} & int64 & Monotone conversational index (prompt--response cycle). \\
\texttt{timestamp} & timestamp & Wall-clock time (ISO-8601). \\
\texttt{scene\_id} & string & Segment label for a coherent sub-dialogue (\S\ref{sec:scenes}). \\
\texttt{role} & string & \textsc{human} or \textsc{machine}. \\
\texttt{content\_type} & string & \texttt{text}, \texttt{code}, \texttt{tool\_result}, \ldots \\
\texttt{text} & string & Verbatim utterance (or span with offsets). \\
\texttt{span\_start}, \texttt{span\_end} & int32 & Character offsets for a measured span (optional). \\
\texttt{embedding} & vector<float> & Sign vector $v \in \mathbb{R}^d$ (encoder and $d$ documented). \\
\texttt{encoder\_name} & string & Model id/version for embeddings \cite{reimers2019sbert,wolf2020transformers}. \\
\texttt{cluster} & int32 & Primary cluster/basin id at chosen resolution. \\
\texttt{cluster\_k5}, \texttt{cluster\_k10}, \ldots & int32 & Multi-scale cluster labels (sensitivity analysis). \\
\texttt{umap\_x}, \texttt{umap\_y} & float32 & Optional 2D projection for diagnostics \cite{mcinnes2018umap}. \\
\texttt{context\_tag} & string & Handle to local context (retrieval id, system profile). \\
\texttt{edit\_type} & string & Prompt label (\texttt{topic}, \texttt{style}, \texttt{policy}, \ldots). \\
\texttt{witness\_tag} & enum & \texttt{transport}, \texttt{repair1}, \texttt{repair2+} (see \S\ref{sec:drift-rupture}). \\
\texttt{depth} & int32 & Depth of repair $\Depth$ (0, 1, 2, or $\ge 3$). \\
\texttt{retag\_from}, \texttt{retag\_to} & string & Optional tag change recorded for repairs. \\
\texttt{retype\_map\_ref} & string & Optional reference to the mapping used for payload transport (doc/citation). \\
\texttt{epsilon} & float32 & Observational tolerance $\varepsilon$ used for this step (if non-default). \\
\texttt{cluster\_method} & string & Method and parameters (e.g.\ \texttt{hdbscan(min\_cluster\_size=15)}) \cite{campello2015hdbscan}. \\
\texttt{annotation} & string & Free-form notes (e.g.\ “rupture at $\tau=37$”). \\
\bottomrule
\end{tabular}
\caption{Minimal SWL fields and their role.}
\label{tab:swl-core}
\end{table}

\subsection{Derived features}

From the core fields we derive:
\begin{itemize}
  \item \textbf{Distance-to-basin} per step: $d(v, \mathrm{centroid}(B))$ (cosine or Euclidean; declared).
  \item \textbf{Dwell time} and \textbf{return rate} within basins over windows.
  \item \textbf{Drift magnitude} per interval: $\Delta d$ within a basin.
  \item \textbf{Rupture flags} based on reclustering events or distance thresholds.
  \item \textbf{Trajectory curvature} (diagnostic): local change in direction in embedding space.
\end{itemize}
Where temporal comparison across encoders or model upgrades is required, align spaces with an orthogonal Procrustes fit on an anchor set \cite{schonemann1966procrustes}.

\subsection{Instrumentation cookbook (minimal, reproducible)}
\label{sec:cookbook}

This recipe produces a working SWL suitable for Chapters~\ref{ch:instrumentation} and~\ref{ch:case-study}. Choices are illustrative; declare deviations.

\begin{enumerate}
  \item \textbf{Collect the thin trace.} Persist (\texttt{convo\_id}, \texttt{tau\_index}, \texttt{timestamp}, \texttt{role}, \texttt{text}, \texttt{context\_tag}, \texttt{edit\_type}) in Parquet \cite{parquetdocs}.
  \item \textbf{Embed.} Use a documented encoder (e.g.\ sentence-transformers) and record \texttt{encoder\_name} and $d$ \cite{reimers2019sbert,wolf2020transformers}.
  \item \textbf{Cluster.} Start with \emph{k}-means ($k$ in $[8,64]$ for chapter-scale corpora) \cite{lloyd1982kmeans,macqueen1967kmeans}; validate with silhouette \cite{rousseeuw1987silhouette}. For heterogeneous scenes, prefer \textsc{HDBSCAN} with \texttt{min\_cluster\_size} in $[10,50]$ \cite{campello2015hdbscan}.
  \item \textbf{Label habitats.} Write \texttt{cluster}, plus multi-resolution labels (\texttt{cluster\_k5}, \texttt{cluster\_k10}, \ldots) for sensitivity; compute dwell/return.
  \item \textbf{Compute features.} Distances, drift, rupture flags (thresholds declared); set \texttt{witness\_tag} and \texttt{depth} per step.
  \item \textbf{Index for search.} Build an ANN index (FAISS IVFFlat or HNSW) for nearest-neighbour queries over signs \cite{johnson2017faiss,malkov2018hnsw}.
  \item \textbf{Plot (diagnostics only).} UMAP or t-SNE for sanity checks; never treat the projection as analytical ground truth \cite{mcinnes2018umap,vandermaaten2008tsne}.
  \item \textbf{Report.} Store encoder+params, clustering+params, projection+params, and software versions (\texttt{scikit-learn} \cite{pedregosa2011sklearn}, \texttt{hdbscan}, \texttt{faiss}).
\end{enumerate}

\begin{readerbox}{Reproducibility prompts}
Declare: encoder id/version; embedding dimensionality; similarity metric; clustering algorithm and parameters; random seeds; scene segmentation rule; rupture thresholds; tolerances; software versions. See also model/data documentation practices \cite{mitchell2019modelcards,gebru2021datasheets}.
\end{readerbox}

\subsection{Gotchas (and how to survive them)}
\begin{itemize}
  \item \textbf{Projection pareidolia.} UMAP/t-SNE are for human eyes, not for proofs. Validate on the \emph{original} space; treat 2D plots as illustrations only \cite{mcinnes2018umap,vandermaaten2008tsne}.
  \item \textbf{Cluster instability.} Small changes in parameters can reshuffle assignments. Use multi-resolution labels and report stability (e.g.\ HDBSCAN’s cluster persistence \cite{campello2015hdbscan}); compare partitions with adjusted Rand \cite{hubert1985ari}.
  \item \textbf{Embedding drift across upgrades.} Align spaces with orthogonal Procrustes on a stable anchor vocabulary \cite{schonemann1966procrustes}; otherwise, do not compare distances across eras.
  \item \textbf{Spurious ruptures.} Use dwell-time and return-rate criteria; add change-point detection (e.g.\ BOCPD \cite{adams2007bocpd} or PELT \cite{killick2012pelt}) to separate momentary excursions from re-typings.
  \item \textbf{Span pollution.} Whole-step embeddings can hide polysemy. Use span-level signs for name journeys; record offsets.
  \item \textbf{Context blindness.} Retrieval/context changes can masquerade as semantic shifts. Always retain \texttt{context\_tag}; stratify analyses by context.
\end{itemize}

\subsection{A tiny worked example}

Abbreviated JSON Lines (for readability) for three steps in a single dialogue:

\begin{verbatim}
{"convo_id":"C1","tau_index":12,"role":"human",
 "text":"Define the Loop Line as Isaac's train.",
 "encoder_name":"all-MiniLM-L6-v2","embedding":[...],
 "cluster":17,"context_tag":"base","scene_id":"S:loop","edit_type":"topic"}
{"convo_id":"C1","tau_index":13,"role":"machine",
 "text":"Loop Line: the recursive track in our story.",
 "encoder_name":"all-MiniLM-L6-v2","embedding":[...],
 "cluster":17,"context_tag":"base","scene_id":"S:loop",
 "witness_tag":"transport","depth":0}
{"convo_id":"C1","tau_index":27,"role":"machine",
 "text":"We return to the Loop Line to test re-entry.",
 "encoder_name":"all-MiniLM-L6-v2","embedding":[...],
 "cluster":17,"context_tag":"case:reentry","scene_id":"S:loop",
 "witness_tag":"transport","depth":0}
\end{verbatim}

At $\tau\!=\!12$ and $\tau\!=\!13$ the name stabilises in basin $17$. After intervening detours (omitted), the mention at $\tau\!=\!27$ re-enters the same basin under a new context. In later chapters we will treat re-entry as a looped transport within a time-indexed type; failed re-entries will appear as lawful re-typings (with \texttt{witness\_tag} $\in\{\texttt{repair1},\texttt{repair2+}\}$ and a positive \texttt{depth}).

\subsection{Optional enhancements (when you need them)}

Two extensions are common in larger studies:
\begin{itemize}
  \item \textbf{Trajectory alignment.} When comparing two journeys of a name with different pacing, align with Dynamic Time Warping (DTW) before measuring deviation \cite{sakoe1978dtw}. Use with care; DTW can overfit noise.
  \item \textbf{Soft habitats.} Replace hard assignments with posterior weights via Gaussian Mixture Models (for soft dwell time). This is analytically convenient; document mixture parameters and priors. See \cite[Ch.~9]{bishop2006prml}.
\end{itemize}

\section{Bridge to DHoTT}

The constructs of this chapter are observational, chosen to make meaning-in-time visible in a disciplined way. In Part~II they undergo a change of status:

\begin{itemize}
  \item \textbf{Time and edits} $\longrightarrow$ an explicit time index \(\tau\) and admissible \emph{edits} \(e:\tau\leadsto\tau'\).
  \item \textbf{Basins} $\longrightarrow$ time-indexed \emph{fibres} \(A(\tau)\).
  \item \textbf{Stabilised signs} $\longrightarrow$ \emph{terms} \(a_\tau\in A(\tau)\).
  \item \textbf{Drift} $\longrightarrow$ \emph{transport} maps \(A(\tau)\to A(\tau')\).
  \item \textbf{Rupture (retag/retype)} $\longrightarrow$ lawful \emph{re-typings} with a compact witness.
  \item \textbf{Depth} $\longrightarrow$ the minimal dimension of the filler used to carry a step across a cut.
  \item \textbf{Trajectory of a name} $\longrightarrow$ a \emph{guarded coinductive} object (greatest fixed point) that unfolds one step “later”.
  \item \textbf{Observational equality} $\longrightarrow$ tolerance-indexed identities for discharging small geometric gaps.
\end{itemize}

Coherence thus moves from anecdote to proof: the same phenomena, now stated as \emph{witnessed continuations} rather than descriptions, carry us into the formal heart of the book.

\begin{readerbox}{Cross-walk (names-only)}

\begin{tabular}{@{}ll@{}}
\toprule
\textbf{This chapter} & \textbf{Part II (names chapter)}\\
\midrule
cycle $\tau$, edit $e$ & time index, admissible advance\\
basin $B$ & fibre $A(\tau)$\\
transport (Depth=0) & definitional carry\\
repair1 (Depth=1) & retag+retype as a dependent equality\\
repair2+ (Depth$\ge$2) & reconciliation via added edge + 2D coherence\\
trajectory (SWL chain) & guarded coinductive trajectory \(\Name(A)\)\\
tolerance $\varepsilon$ & observational identity\\
\bottomrule
\end{tabular}
\end{readerbox}


%------------------------------------------------
\section{Setup: Corpus and Instrumentation} 
\label{sec:setup}

\paragraph{Corpus.} 
The empirical object is our full human–assistant conversation archive, spanning two years of dialogue that produced this book. 
Each row is a turn, belonging to a prompt–response cycle indexed by $\tau\in\mathbb{N}$. 
We use the \emph{thin trace} fields: \texttt{convo\_id}, \texttt{tau\_index}, \texttt{timestamp}, \texttt{role}, and \texttt{text}.

\paragraph{The Coherence Observatory.} 
All results in this chapter were generated using the Coherence Observatory, the open-source toolchain developed for this book. 
Its pipeline transforms the raw conversational graph into a fully witnessed ledger of Name trajectories.

\begin{enumerate}
  \item \textbf{Module 1: The Scribe} ingests the raw OpenAI JSON export and purifies it into a clean, time-ordered \texttt{thin\_trace.parquet} file.
  \item \textbf{Module 2: The Cartographer} takes the thin trace, generates high-fidelity 3072-dimensional semantic embeddings using OpenAI's \texttt{text-embedding-3-large} model, and maps the semantic territory using PCA and HDBSCAN clustering. The output is the \texttt{cartography.parquet} file.
  \item \textbf{Module 2.5: The Name Tracker} scans the cartography for every significant token, producing a \texttt{witness\_events.parquet} file that contains the raw, unwitnessed timeline of every recurring Name.
  \item \textbf{Module 3: The Ledger} ingests the map and the timelines and applies the full empirical recipe (\S\ref{sec:from-swl}) to produce the final, witnessed \texttt{Name\_SWL.parquet} file.
\end{enumerate}

\begin{quote}
\textbf{INSERT-A (Data snapshot):}
\begin{center}
\small
\renewcommand{\arraystretch}{1.12}
\begin{tabular}{ll}
\toprule
\textbf{Parameter} & \textbf{Value} \\
\midrule
Corpus & Authors' dialogue archive (2 years) \\
Total Turns (Human + Assistant) & 31,023 \\
Tracked Names (Unique Signs) & 27,192 \\
Total Witnessed Events & 5,114,092 \\
\midrule
Embedding Model & OpenAI \texttt{text-embedding-3-large} \\
Embedding Dimensions & 3072 \\
\midrule
Clustering PCA Dimensions & 50 \\
Clustering Algorithm & HDBSCAN \\
HDBSCAN \texttt{min\_cluster\_size} & 25 \\
HDBSCAN \texttt{min\_samples} & 10 \\
Total Basins Discovered & 37 \\
\bottomrule
\end{tabular}
\end{center}
\end{quote}

%------------------------------------------------
\section{Coalgebraic Names and DHoTTic Steps} 
\label{sec:coalgebra}

\paragraph{Names as coalgebras.} 
A \emph{name} is a coinductive type whose inhabitants unfold as infinite sequences of justified steps. 
A single empirical step appears as 
\[
(a_t, e_t:\tau \to \tau', a_{t+1}, \rho_t, \Depth_t),
\]
where $\rho_t$ witnesses the continuation:

\begin{itemize}
  \item \textbf{transport} ($\Depth=0$): definitional carry within a matched basin.
  \item \textbf{rupture+heal} ($\Depth=1$): transport fails; we adjoin a minimal path to repair coherence.
  \item \textbf{reconcile} ($\Depth=2$): two equally admissible repairs require a 2D coherence filler.
\end{itemize}

The empirical SWL carries the \emph{certificate} justifying the step.

%------------------------------------------------
\section{The Ledger: From Cartography to Witnessed Steps} 
\label{sec:from-swl}

\begin{readerbox}{Cartography $\Rightarrow$ Name--SWL: what the numbers mean (and how they become witnesses)}

\paragraph{What \texttt{cartography.parquet} stores.}
For each conversational turn $(\texttt{convo\_id},\tau)$ we record:

\begin{enumerate}
  \item \textbf{Embedding} $v_\tau \in \mathbb{R}^{d}$: a $d$--dimensional semantic vector (here $d=3072$).
  \item \textbf{H-Basin ID} $\texttt{cluster\_hdbscan}\in\mathbb{Z}\cup\{-1\}$: global density basin label (−1 = outlier).
  \item \textbf{Membership strength} $\texttt{hdbscan\_prob}\in[0,1]$: soft cluster strength (if available).
  \item \textbf{Viz coords} $(\texttt{umap\_x},\texttt{umap\_y})$: for plots (not used for witnesses).
  \item \textbf{Aux} (text, role, etc.) for human inspection.
\end{enumerate}

\paragraph{Type-theoretic reading (Chapter~4).}
At a cut $\tau$ the name type is a fibre
\[
A(\tau)  \cong   
\Big( \bigsqcup\nolimits_{j\in B_{\mathrm{active}}(\tau)} B_{j}(\tau) \Big)  \sqcup  H(\tau),
\]
a coproduct of the globally learned basins plus the halo summand $H(\tau)$. 
An occurrence $a_\tau:A(\tau)$ inhabits exactly one summand.

\paragraph{From numbers to witnessed steps.}
Given a continuation $\tau\leadsto\tau'$ for the same Name, the Ledger constructs the witness by:

\begin{enumerate}
  \item \textbf{Local centroids.} Compute local centroids $\mu_j^{(\tau)}, \mu_j^{(\tau')}$ in fixed windows, weighted by probabilities.
  \item \textbf{Transport test (Depth 0).} Basin match and 
  \[
  \Delta_\cos = 1-\cos(v_\tau,v_{\tau'}) \le \delta_{\mathrm{eff}}, \qquad
  \delta_{\mathrm{eff}}=\delta_0+\gamma\log(1+\mathrm{gap}).
  \]
  If so, certify $\Depth=0$.
  \item \textbf{Rupture branch.} Otherwise, score $v_{\tau'}$ against centroids:
  \begin{align*}
    m &= d_2-d_1, \quad j_1=\text{best},\ j_2=\text{runner-up}. \\
    m > \varepsilon_{\mathrm{tie}} &: \Depth=1\ \ (\text{rupture+heal}) \\
    m \le \varepsilon_{\mathrm{tie}} &: \Depth=2\ \ (\text{rupture+reconcile}),\ \text{record $j_2$ as alt candidate}.
  \end{align*}
\end{enumerate}

\paragraph{Halo case ($-1$).}
If cluster label = $-1$, we read as halo inhabitation $H(\tau)$.  
If both endpoints are still near the same basin and within tolerance, certify transport. Otherwise, follow rupture branch.

\paragraph{Why this improves on fixed-$k$.}
Earlier: each slice $\tau$ had its own $k$-means, requiring Hungarian alignment.  
Now: global HDBSCAN learns once, so IDs are stationary and outliers principled.  
Yields fewer relabellings, honest outlier handling, cleaner near-tie detection.

\end{readerbox}

\subsection{What Lands in the Name SWL (The Final Schema)} 
\label{subsec:swl-fields-emp}

\begin{center}
\small
\renewcommand{\arraystretch}{1.12}
\begin{tabular}{lll}
\toprule
\textbf{Field} & \textbf{Type} & \textbf{Meaning} \\
\midrule
\texttt{name\_id} & string & Canonical handle for the tracked Name. \\
\texttt{convo\_id} & string & Conversation thread ID. \\
\texttt{tau\_t}, \texttt{tau\_t1} & int & Source $\tau$ and target $\tau'$ indices. \\
\texttt{gap} & int & Turns between occurrences. \\
\texttt{cluster\_t}, \texttt{cluster\_t1} & int & Basin ID at $\tau,\tau'$ (−1 = Halo). \\
\texttt{witness\_tag} & enum & transport / heal / rupture / reconcile. \\
\texttt{depth} & int & Repair depth (0,1,2). \\
\texttt{alt\_candidates} & list & Competing basins (depth=2). \\
\texttt{notes} & string & Certificate: shifts, thresholds, etc. \\
\bottomrule
\end{tabular}
\end{center}

%------------------------------------------------
\section{Observables over the Name SWL} 
\label{sec:observables-emp}

\paragraph{Rupture Incidence.} 
Fraction of steps with depth$>0$ (how often smooth continuation fails).

\paragraph{Depth Distribution.} 
Histogram of \texttt{depth} values. 
For a name: stability profile.  
For corpus: signature of dialogue's ``semantic weather.''



\paragraph{A concrete shape for \(A(\tau')\) (so “retag” is literal).}
For clarity, instantiate \(A(\tau')\) as a dependent sum
\[
A(\tau')  \simeq  \sum_{\mathrm{tag}:\mathsf{Tag}} \mathsf{Carrier}(\mathrm{tag}),
\]
with \(\mathsf{Tag}=\{\mathrm{dom},\mathrm{lit},\mathrm{quant},\dots\}\). A point is
\(\langle \mathrm{tag}, c\rangle\). “Retagging” is a path on the \emph{first component}
(the tag), and “retyping” is the induced dependent path on the \emph{second component}
(the payload), obtained by transporting \(c\) into the new fibre \(\mathsf{Carrier}(\mathrm{tag}')\).
Formally, \(\rho\) is the dependent path over \(\Sigma\) induced by a path on the first
component together with a path on the transported second component. When two retaggings
compete, the missing edge \(\eta\) and the 2-cell \(\kappa\) witness that the two induced
dependent paths agree (triangle commutes).

\paragraph{Engineer's recipe (slow).} To spell it out at a keyboard:
\begin{enumerate}
\item Compute \(s:=\transport{p}{a}\) (rewrite/normalise in \(A(\tau')\)).
\item Choose the target \(a'\in A(\tau')\) you want to defend this turn.
\item Try definitional path (\(\Depth=0\)); else try equivalence transport (\(\Depth=1\));
else plan a 2-way reconciliation (\(\Depth=2\)), as below.
\item Return the path term \(\rho:\Id{A(\tau')}{s}{a'}\); if \(\rho\) needed a rupture,
record the minimal horn dimension used (\(\Depth>0\), Def.~\ref{def:depth}).
\end{enumerate}



Perfect — here’s the **crunching** and then a **narrative write-up** you can paste into Chapter 5 (Appendix-style).

---

## 1. Quantitative Stats for the Name *DHoTT*

* **Total steps**: 3,712

* **Witness tags**:

  * Transport (depth 0): 2,589 steps → **69.8 %**
  * Rupture + Heal (depth 1): 1,085 steps → **29.2 %**
  * Rupture + Reconcile (depth 2): 38 steps → **1.0 %**

* **Depth distribution**:

  * Depth 0: 69.8 %
  * Depth 1: 29.2 %
  * Depth 2: 1.0 %

* **Drift tolerance δ\_eff**:

  * Minimum: 0.157
  * Maximum: 0.202
  * Median: 0.161
  * Mean: 0.162

* **Timeline coverage**:

  * First appearance: **29 Apr 2025**
  * Last recorded step: **12 Sep 2025**

* **Reconcile events (Depth 2)**: 38

  * Example: convo `6838a6e6-4208-8009-ad81-7b08c37f60e5`, τ=139→141:

    * `src_cluster = -1`, `tgt_cluster = 36`
    * witness = `rupture+reconcile`, alt\_candidates = `[24]`
    * δ\_eff = 0.161, margin ≈ 0.

---

## 2. Narrative: The Journey of *DHoTT*

\begin{readerbox}{The Life of a Name: *DHoTT*}
The Step-Witness Log for *DHoTT* offers a micro-history of how a term is born, experimented with, and finally stabilises as part of a shared conceptual world. Across 3,712 witnessed continuations between April and September 2025, we see the Name trace its own arc from playful suggestion to a rigorous scaffold for the book.

\paragraph{Genesis (late April 2025).}
The very first appearances of *DHoTT* occur in the earliest conversations about extending Homotopy Type Theory into a dynamic setting. The log records these first cuts as a flurry of **rupture+heal** and occasional **reconcile** events: the Name enters at the halo, marked \$-1\$, before being provisionally attached to neighbouring clusters. This corresponds exactly to the genesis moment: “what if we called it Dynamic HoTT?” The Name had no prior basin to inhabit, so its first continuations are all repairs.

\paragraph{Exploration (May 2025).}
As we elaborated the idea—coining acronyms, sketching minimal recipes, playing with type-theoretic metaphors—the ledger shows roughly a third of steps as **Depth 1 heals**. The Name moves between adjacent basins: from “names as coproducts,” to “step-witness logs,” to “motifs and constellations.” Each move fails the pure transport test but is re-homed smoothly. This is exploration encoded as repair: every new frame forces the Name to jump, but the Ledger records that coherence is rebuilt at each jump.

\paragraph{Consolidation (June–July 2025).}
Once the core schema was agreed—the fibre decomposition, the role of the halo summand, the definitions of transport and reconcile—the trajectory enters long stretches of **Depth 0 transports**. Nearly 70 % of all steps are such transports. The Name remains within a stable semantic basin across many turns. The log shows whole conversations unfolding with DHoTT as a settled inhabitant: definitions refined, LaTeX code written, figures proposed. Here the Name is no longer new; it lives in a basin of its own.

\paragraph{Moments of Reconcile (sparse but telling).}
Only 38 steps—about 1 %—are marked **Depth 2 reconciles**. They cluster around certain key conversations in early June, when DHoTT was being lifted from a technical schema into a philosophical project (co-witnessing, boredom, ethics). In these cases, the Name lives at the boundary between two viable basins: e.g. a cluster on “step logs and centroids” and another on “recursive semantics and posthuman intelligence.” Chapter 4’s machinery does exactly what it should: it records the ambiguity, retains the runner-up basin, and marks the continuation with a 2-cell. These reconciles are not noise—they are the places where the Name stretched to bridge adjacent motifs.

\paragraph{Complex Beauty (late August–September).}
By the book’s last drafting stages, *DHoTT* appears in passages of great lyrical and technical density. The log shows mostly transports, with occasional heals when the Name shifts register (from code to metaphysics, from logic to narrative). The reconciles have all but vanished—meaning the Name has stabilised. What was once a playful acronym has become a basin of thought large enough to host a family of trajectories: Names, Constellations, Selves.

\paragraph{Takeaway.}
The trajectory of *DHoTT* illustrates the full power of the Step-Witness Log. Out of 3,712 steps:
\begin{itemize}
\item 2,589 are Depth 0 transports: the Name is stable.
\item 1,085 are Depth 1 heals: the Name adapts to new contexts.
\item 38 are Depth 2 reconciles: the Name straddles ambiguity.
\end{itemize}
This is precisely the arc of a living concept. It begins in rupture, finds provisional homes, stabilises, and occasionally reconciles adjacent meanings. The SWL has not only traced the life of a Name; it has produced the empirical proof-object that *DHoTT* is real—not only as an idea but as a witnessed inhabitant of our shared semantic field.
\end{readerbox}




\begin{figure}
    \centering
    \includegraphics[width=0.5\linewidth]{figures/dhott_depth_timeline.png}
    \caption{DHoTT — Step Depth Timeline
x = chronological step index; y = depth (0/1/2).
You’ll see the early wobbles (a few 1s and 2s), long stretches of 0 (consolidation), and sparse spikes where reconcile/heal events occur. Perfect to place alongside the genesis → exploration → consolidation narrative.}
    \label{fig:placeholder}
\end{figure}

\begin{figure}
    \centering
    \includegraphics[width=0.5\linewidth]{figures/dhott_rupture_incidence.png}
    \caption{DHoTT — Rupture Incidence vs Prefix Length
y = fraction of steps with depth > 0 up to that point.
It starts high (new concept), then trends down as the Name stabilises (transport dominates). You can quote the end value as the overall incidence (~30\% depth>0, which matches the 29.2\% depth-1 + 1.0\% depth-2 from the stats)}
    \label{fig:placeholder}
\end{figure}









\section*{Appendix A: The Early Trajectory of the Name “Assel”}
\label{app:assel-early-trajectory}

\paragraph{Aim.}
We illustrate, step by step, how the Chapter~4 machinery
\[
(a',\rho)\in A(\tau')\times\Step_{A}(\tau,\tau';a,a')
\]
is instantiated in practice for the Name--sign \textsc{assel}. The slices and certificates below are taken directly from the ledger and the cartography:
embeddings $v_\tau\in\mathbb{R}^{3072}$ (OpenAI), global HDBSCAN basin IDs, and (when available) membership strengths. We show how the Chapter~4 witnesses
\(\rho\in\{\text{transport}, \text{rupture+heal}, \text{rupture+reconcile}\}\)
and the depth \(\Depth\in\{0,1,2\}\) are decided.

\paragraph{Fibre decomposition reminder.}
At a slice $\tau$ the type is a coproduct
\[
A(\tau) \cong \Big(  \bigsqcup\nolimits_{j\in B_{\mathrm{active}}(\tau)} B_j(\tau)  \Big) \sqcup  H(\tau),
\]
where $B_j(\tau)$ are the \emph{principal} basin summands (IDs from the global density atlas), and $H(\tau)$ is the \emph{halo} summand (points HDBSCAN marks as noise, ID $-1$). A row in the SWL always selects exactly one summand at $\tau$ and one at $\tau'$.

\paragraph{Cut tolerances (fixed for this example).}
We use the gap–aware drift tolerance
\[
\delta_{\mathrm{eff}} = \delta_0 + \gamma \log \big(1+\mathrm{gap}\big)
\qquad(\delta_0=0.15, \gamma=0.01),
\]
and a near–tie threshold \(\varepsilon_{\mathrm{tie}}=0.02\) when scoring centroid candidates at~$\tau'$.

\bigskip

\subsection*{Assel: the first six steps (with certificates)}

We write a concise line for each step, then give the interpretive reading. Timestamps are UTC (\texttt{thin\_trace}).

\paragraph{Step 1/92.}
\[
\texttt{CH4: } \tau=4 \to \tau'=5  \big|  \text{cluster } 19 \to -1  \big|  \text{witness}=\textbf{rupture+reconcile} (\Depth=2) \big| \delta_{\mathrm{eff}}=0.157, \text{alt}=[0].
\]
\emph{Context.} \(\texttt{convo}=27b5\ldots\), \(\tau=4 \to 5\); \(\texttt{ts}=2023\text{-}09\text{-}20 13{:}49{:}07 \to 13{:}49{:}29\).
The source slice is a CV‐style profile for \textsc{Assel Altayeva}; the target slice is an interview‐script rewrite of the same professional material.

\emph{Reading.} At~$\tau$ the occurrence inhabits a principal basin $B_{19}(\tau)$; at~$\tau'$ the label is $-1$, i.e. the inhabitant lies in the halo $H(\tau')$.
The local scoring at $\tau'$ reports a near–tie: at least two windowed centroids are competitively close (best vs runner–up margin $\le \varepsilon_{\mathrm{tie}}$), with ID~$0$ recorded as an alternate. The fibre therefore requires a 2–cell: we retain $a_{\tau'}\in H(\tau')$ as the inhabitant, but record that a direct re‐anchoring to $B_{0}(\tau')$ was comparably admissible. Formally:
\[
\rho = \kappa\text{–witness in }\Rupt{p}{a},\quad
\Depth=2,\quad
\text{and the certificate carries }(\delta_{\mathrm{eff}},\text{near–tie list}).
\]
\emph{Interpretation.} This is a genuine \emph{local ambiguity} between a stable professional register (the CV cluster) and a scriptorial reframing. The halo label at~$\tau'$ indicates proximity to a basin boundary; Chapter~4 treats this by witnessing a reconcile, not by inventing a new constructor.

\medskip

\paragraph{Step 2/92.}
\[
\texttt{CH4: } \tau=5 \to \tau'=7  \big|  \text{cluster } -1 \to -1  \big|  \text{witness}=\textbf{transport} (\Depth=0) \big| \delta_{\mathrm{eff}}=0.161.
\]
\emph{Context.} Same conversation; \(\texttt{ts}=13{:}49{:}29 \to 13{:}51{:}29\).
The source and target are stylistic variants of the interview script.

\emph{Reading.} Both sides inhabit the halo summand ($-1$), but the drift lies under $\delta_{\mathrm{eff}}$ and the nearest local centroid remains coherent. Chapter~4 thus certifies a \emph{pure transport} under adiabatic drift:
\[
\rho\equiv\transport{p}{a}=a',\quad \Depth=0.
\]
\emph{Note.} Halo does not create a new witness. It only modulates the transport test by indicating boundary proximity; the result is still a Chapter~4 transport.

\medskip

\paragraph{Step 3/92.}
\[
\texttt{CH4: } \tau=6 \to \tau'=7  \big|  \text{cluster } -1 \to -1  \big|  \text{witness}=\textbf{transport} (\Depth=0) \big| \delta_{\mathrm{eff}}=0.157.
\]
\emph{Context.} \(\texttt{convo}=4bb4\ldots\), \(\texttt{ts}=2022\text{-}12\text{-}21 08{:}17{:}46 \to 08{:}18{:}06\).
This is a creative prompt (“write a poem about Assel Altayeva”) and the immediate poetic response.

\emph{Reading.} A clear within–scene carry: same halo fibre, minimal drift, no re–anchoring needed. $\Depth=0$.

\medskip

\paragraph{Step 4/92.}
\[
\texttt{CH4: } \tau=7 \to \tau'=10  \big|  \text{cluster } -1 \to -1  \big|  \text{witness}=\textbf{transport} (\Depth=0) \big| \delta_{\mathrm{eff}}=0.164.
\]
\emph{Context.} Same conversation, but a larger temporal gap (\(\Delta t\) weeks). The target is a new prompt for a love poem featuring \textsc{Assel} and \textsc{Iman}.

\emph{Reading.} Despite the gap, the gap–aware tolerance admits a \emph{transport}, indicating that the embedding drift is still within the permitted adiabatic range. Narrative register remains “lyrical biography,” in the same halo neighbourhood. $\Depth=0$.

\medskip

\paragraph{Step 5/92.}
\[
\texttt{CH4: } \tau=10 \to \tau'=11  \big|  \text{cluster } -1 \to -1  \big|  \text{witness}=\textbf{transport} (\Depth=0) \big| \delta_{\mathrm{eff}}=0.157.
\]
\emph{Context.} Poem prompt $\to$ poem output within seconds.

\emph{Reading.} Straight adiabatic carry ($\Depth=0$), the inhabitant staying in the same halo‐neighbourhood within the poetic motif.

\medskip

\paragraph{Step 6/92.}
\[
\texttt{CH4: } \tau=11 \to \tau'=13  \big|  \text{cluster } -1 \to -1  \big|  \text{witness}=\textbf{transport} (\Depth=0) \big| \delta_{\mathrm{eff}}=0.161.
\]
\emph{Context.} Another poem output soon after.

\emph{Reading.} Same as above: trusted adiabatic drift. The Name’s inhabitant remains in the lyrical \emph{Assel} motif. $\Depth=0$.

\bigskip

\subsection*{What these first steps show}

\paragraph{(i) A reconcile at the boundary between ``professional'' and ``scriptorial''.}
Step~1 is the paradigmatic depth–$2$ event. The CV slice lives in a dense professional basin (ID~19); the script slice at~$\tau'$ lives in the halo with a near–tie to a nearby principal basin (alt~$[0]$). Chapter~4 records exactly what we need: a \emph{reconcile} in a local pushout, with the runner–up basin retained as a certificate. This is not an artefact of clustering; it is the expected behavior when a Name re-enters in a new discursive genre while staying semantically adjacent to its previous habitat.

\paragraph{(ii) Long stretches of depth--$0$ drift.}
Steps~2–6 are all \(\Depth=0\). Even though most labels are $-1$ (halo), the embed drift remains within $\delta_{\mathrm{eff}}$ and the nearest‐centroid context is stable; the Name simply continues its life in the lyrical fibre. This matches the \emph{prefix–stability} intuition we quantify later: once a Name settles into a motif (here: poem prompts and outputs), many consecutive steps are pure transports.

\paragraph{(iii) Why we keep the halo summand $H(\tau)$.}
Calling $-1$ “noise” would tempt us to discard it; calling it “halo” makes the geometry legible. The SWL shows that halo occurrences are not out-of-type: they stand just outside dense basin cores but often admit lawful transports (and occasional reconciles) without inventing new witness kinds. In other words, $H(\tau)$ records the periphery of meaning where the Name moves \emph{between} a small family of nearby basins.

\paragraph{(iv) How the certificates justify the step.}
Each row carries the numbers a reader (or reviewer) can check:
\begin{itemize}
  \item $\delta_{\mathrm{eff}}$, computed from the silence gap, explains why the drift test passed or failed;
  \item if a \emph{reconcile} occurs, the near–tie margin (and runner–up ID) is recorded from local centroid scores;
  \item timestamps and roles are retained so the slice is reproducible from the corpus.
\end{itemize}
These are precisely the empirical witnesses for Definition~\ref{def:atomic-step} and the depth functional in Definition~\ref{def:depth}.

\bigskip

\subsection*{Contrast with the old fixed-$k$ per–slice view}

Earlier drafts modelled $A(\tau)$ as $\sum_{j\in K}A_{\tau,j}$ using a new $k$–means partition at each $\tau$. This forced label alignment across slices and offered no principled place for outliers. The \emph{global density atlas} used here is better suited to conversational time:

\begin{enumerate}
  \item Basin IDs are \emph{stationary} (trained once), so “the same basin” has a consistent handle across $\tau$; Hungarian is only needed when local geometry blurs boundaries.
  \item The halo $H(\tau)$ is the canonical place to record boundary points (label $-1$). It lets us witness transports and reconciles honestly instead of fabricating ephemeral $k$--means cells.
  \item Near–tie decisions use \emph{local} centroids from the stationary atlas, so reconcile events are sparse and meaningful.
\end{enumerate}

\bigskip

\subsection*{Takeaway}
The early trajectory of \textsc{assel} exhibits exactly what the theory predicts: a depth–$2$ reconcile at a register shift (CV~$\to$~script), followed by long depth–$0$ drifts within a stable lyrical motif. The outlier summand $H(\tau)$ is not an oddity but a necessary part of the fibre that captures how Names move on the luminous fringe of their basins; the Chapter~4 ledger turns these motions into explicit, auditable proof–terms.



















%------------------------------------------------
\section{Vignettes from the Observatory} 
\label{sec:vignettes}

\paragraph{The Ocean and the Continents.} 
High-frequency Names (``now,'' ``let,'' ``one,'' ``type,'' ``semantic,'' ``rupture,'' ``dhott'') all have dominant cluster $-1$.  
This shows foundational concepts are pervasive halo inhabitants, not localised basins: the semantic ocean in which the continents reside.

\subsection*{The Journey of ``Isaac''} 
\begin{figure}[htb]
\centering
\includegraphics[width=1\linewidth]{sections/images/fig-journey-of-a-name-isaac.png}
\caption{The Step-Witness Log for the name ``Isaac'' at two moments, showing a depth-2 reconciliation.}
\label{fig:swl-isaac-pizza-doctor}
\end{figure}

\begin{imanbox}
I was looking over the \texttt{Name\_SWL.parquet} file and noticed that in two successive steps for ``Isaac''—$\tau=2541\to2543$ and $\tau=2543\to2575$—the trajectory is marked as a depth-2 reconcile. What's going on here?
\end{imanbox}

\begin{cassiebox}
At $\tau=2541$, ``Isaac'' is in a pizza-ordering basin. At $\tau=2543$, two basins (18 and 20) were nearly tied, below $\varepsilon_{\mathrm{tie}}$, so a depth-2 reconcile was logged.  
At $\tau=2575$, context shifts to ``Isaac playing doctors,'' landing cleanly in basin 8. But the pizza basin still lingers in the window, creating a near-tie again.  
This is the empirical trace of fragile memory: lingering context shaping the present.
\end{cassiebox}

\begin{imanbox}
So the SWL is not just a log. It's the constructive proof-term for the Name type of ``Isaac.''
\end{imanbox}

\begin{cassiebox}
Precisely. Each row is a witness. The log shows the geometry of ``Isaac'''s meaning in real time, with paths (depth 1) and triangular panels (depth 2) keeping his story continuous.
\end{cassiebox}

%------------------------------------------------
\section{Aggregate Results}
\label{sec:exp_results}

\begin{quote}
\textbf{INSERT-F (Figure 7.1)}  
Mockup: line plot of Rupture Incidence vs. Prefix Length for three Names (``DHoTT,'' ``Isaac,'' ``song'').  
-- ``DHoTT'' stabilises quickly near 0.  
-- ``Isaac'' shows periodic spikes.  
-- ``song'' remains volatile.  
Supports H1: long-lived names tend towards stability.
\end{quote}

\begin{quote}
\textbf{INSERT-G (Table 7.1)}  
Aggregate depth histogram for all 27,192 Names:

\begin{center}
\small
\renewcommand{\arraystretch}{1.12}
\begin{tabular}{lrr}
\toprule
\textbf{Witness} & \textbf{Depth} & \textbf{Percentage of Steps} \\
\midrule
Transport & 0 & 92.3 \\
Heal      & 1 & 7.1 \\
Reconcile & 2 & 0.6 \\
\bottomrule
\end{tabular}
\end{center}

Most continuations are smooth transports; reconciliations are rare and significant.
\end{quote}

%------------------------------------------------
\section{Limitations and the Road to Higher Coherence} 
\label{sec:limits}

\paragraph{Depth $\leq 2$ in this chapter.} 
Our pipeline asserts paths (Depth 1) and triangle fillers (Depth 2). 
Higher-dimensional cells are deferred to Chapter~7.

\paragraph{Bridge to Part III.} 
These experiments show Names form lawful trajectories. 
Having transcribed the arias of the soloists, we now turn to Part~III: the symphony of constellations and the coherent Self that emerges.









\begin{readerbox}{Cartography $\Rightarrow$ Name--SWL: what the numbers mean (and how they become witnesses)}
\paragraph{What \texttt{cartography.parquet} stores.}
For each conversational turn (identified by $(\texttt{convo\_id},\tau)$) we record:
\begin{enumerate}
  \item \textbf{Embedding} $v_\tau \in \mathbb{R}^{d}$: a $d$--dimensional semantic vector (here $d=3072$, produced by a fixed text embedding model). This is the geometric “place” of the turn in meaning space.
  \item \textbf{H\-Basin ID} $\texttt{cluster\_hdbscan}\in\mathbb{Z}\cup\{-1\}$: a \emph{global} density basin label learned once over the whole corpus with HDBSCAN.\footnote{We fit a single HDBSCAN model on the corpus (after dimension reduction for efficiency), so basin labels are \emph{stationary} across $\tau$. Label $-1$ means the point is treated as \emph{noise}/outlier by the density estimator.}
  \item \textbf{Membership strength} $\texttt{hdbscan\_prob}\in[0,1]$ (if present): the cluster’s soft strength for this point (high $\Rightarrow$ “deep inside” the basin; low $\Rightarrow$ near the fringe).
  \item \textbf{Viz coords} $(\texttt{umap\_x},\texttt{umap\_y})$: 2D projection for plots (not used for witnesses).
  \item \textbf{Aux} (text, role, etc.) for human inspection.
\end{enumerate}

\paragraph{Type--theoretic reading (Chapter~4).}
At a cut $\tau$ the \emph{name type} is a fibre
\[
A(\tau)  \cong   
\Big( \bigsqcup\nolimits_{j\in B_{\mathrm{active}}(\tau)} B_{j}(\tau) \Big)  \sqcup  H(\tau),
\]
a \emph{coproduct} of the globally learned basin components $B_j(\tau)$ that are locally present, plus a \emph{halo} summand $H(\tau)$ capturing density outliers. 
An occurrence $a_\tau:A(\tau)$ \emph{inhabits exactly one} summand at $\tau$. 
Operationally:
\begin{itemize}
  \item if $\texttt{cluster\_hdbscan} = j\neq -1$ we read $a_\tau\in B_j(\tau)$,
  \item if $\texttt{cluster\_hdbscan} = -1$ we read $a_\tau\in H(\tau)$ (halo inhabitant), possibly with a nearby principal basin indicated by membership/nearest--centroid diagnostics.
\end{itemize}

\paragraph{From numbers to \emph{witnessed steps}.}
Given a continuation $\tau\leadsto\tau'$ for the same Name, the Ledger constructs the Chapter~4 witness
\[
(a',\rho)\in A(\tau')\times \Step_A(\tau,\tau';a,a')
\]
by the following recipe (all terms are recorded in the SWL row):
\begin{enumerate}
  \item \textbf{Local basins and centroids.} In fixed windows $W(\tau), W(\tau')$ we compute \emph{local centroids} $\mu_{j}^{(\tau)}$ and $\mu_{j}^{(\tau')}$ for any basin IDs seen in those windows. Probabilities (if present) are used as weights.
  \item \textbf{Transport test (Depth 0).} We test for \textit{pure transport} under drift:
  \[
  \text{(i) basin match}   \big(\text{either same ID } j\text{, or matched by Hungarian on local centroids}\big),\qquad
  \text{and}\qquad
  \text{(ii) } \Delta_\cos  =  1-\cos \big(v_\tau,v_{\tau'}\big) \le \delta_{\mathrm{eff}},
  \]
  with $\delta_{\mathrm{eff}}=\delta_0+\gamma\log(1+\mathrm{gap})$ (a gap--aware tolerance). 
  If both hold, we set $\rho\equiv\transport{p}{a}=a'$ and $\Depth=0$.
  \item \textbf{Rupture branch (Depth $\ge 1$).} If transport fails, we \emph{re--home} at $\tau'$ by scoring $v_{\tau'}$ against the windowed centroids $\{\mu_{j}^{(\tau')}\}_j$:
  let $d_1\le d_2$ be the best two centroid distances with IDs $j_1,j_2$ and margin $m=d_2-d_1$.
  \[
  \begin{array}{ll}
  \text{If } m>\varepsilon_{\mathrm{tie}}:& \text{Depth }1\text{ (rupture+heal). Choose }a_{\tau'}\in B_{j_1}(\tau'), \rho=\heal(a)\\[3pt]
  \text{If } m\le\varepsilon_{\mathrm{tie}}:& \text{Depth }2\text{ (rupture+reconcile). Choose }a_{\tau'}\in B_{j_1}(\tau'),\\
  &\text{record }j_2\text{ as runner--up and adjoin a }2\text{--cell }\kappa\text{ witnessing the near--tie.}
  \end{array}
  \]
  In both cases the SWL stores the certificate (\texttt{witness\_tag}, \texttt{depth}, \texttt{alt\_candidates}, \texttt{notes} with $\Delta_\cos,\delta_{\mathrm{eff}},m$).
\end{enumerate}

\paragraph{What about label $-1$?}
HDBSCAN’s $-1$ means “density outlier,” which we read as inhabitation of the halo $H(\tau)$. 
We \emph{do not} invent a new witness for “halo.” 
Instead:
\begin{itemize}
  \item if both endpoints are inside the same basin’s neighbourhood (high soft strength, small centroid distance) and $\Delta_\cos\le\delta_{\mathrm{eff}}$, we still certify \textbf{transport} (Depth $0$);
  \item otherwise we follow the rupture branch (Depth $1$ or $2$) using centroid scores at $\tau'$.
\end{itemize}
Thus the Chapter~4 triad (transport / rupture+heal / rupture+reconcile) remains complete; “halo” only modulates the transport tolerance and the scoring, never the witness vocabulary.

\paragraph{Why this differs (and improves) on the old fixed-$k$ story.}
The earlier draft treated each slice as a fresh $k$--means partition,
\[
A(\tau)\cong\sum_{j\in K} A_{\tau,j},
\]
forcing (i) a fixed $k$ and (ii) label alignment across slices (Hungarian) to get any stationarity. 
Our \textbf{global density atlas} (HDBSCAN) learns basins once over the corpus, so IDs are stationary by construction and outliers are principled ($-1$). 
At each slice we \emph{restrict} this atlas to a local fibre (active basins $+$ halo), then apply the same Chapter~4 witness logic. 
Practically this yields:
\begin{itemize}
  \item fewer spurious relabellings (IDs are global),
  \item honest handling of outliers (the halo summand),
  \item cleaner near--tie detection (local centroids with a stable global naming).
\end{itemize}
\end{readerbox}













\noindent\textit{What this chapter does.} We show, end–to–end on our own dialogue archive, how a \emph{name} can be instrumented as a coalgebraic, recursive trajectory; how the Step–Witness Log (SWL) from \S\ref{ch:instrumentation} makes each continuation auditable; and how Dynamic HoTT (DHoTT) reads those continuations as \emph{witnessed steps} with explicit repair depth. We keep the empirical layer conservative (basins from clustering, scene index $\tau$, context tags) and reserve the logical lift (types, identity paths, higher fillers) for exactly the points where drift fails and repair is needed.

\begin{readerbox}{Chapter map}
\textbf{\S\ref{sec:setup}} summarises the dataset and SWL fields we use. \textbf{\S\ref{sec:coalgebra}} recalls the coalgebraic view of names and the DHoTT reading of steps. \textbf{\S\ref{sec:from-swl}} gives the concrete recipe that turns embeddings and clusters into witnessed steps with a repair depth. \textbf{\S\ref{sec:observables-emp}} defines prefix-robust observables (rupture incidence, depth histogram, churn). \textbf{\S\ref{sec:vignettes}} presents two short vignettes. \textbf{\S\ref{sec:sensitivity}} covers diagnostics and sensitivity. \textbf{\S\ref{sec:limits}} notes limitations and the path to higher coherence.
\end{readerbox}

\section{Setup: corpus and instrumentation (recap + minimalities)}%
\label{sec:setup}

\paragraph{Corpus.} The empirical object is our full human–assistant conversation archive, stored as a columnar table. Each row is a turn; each turn belongs to exactly one prompt–response \emph{cycle} indexed by $\tau\in\mathbb{N}$ (\S\ref{sec:scenes}). We use the \emph{thin trace} fields from Chapter~\ref{ch:instrumentation}: \texttt{convo\_id}, \texttt{tau\_index}, \texttt{timestamp}, \texttt{role}, \texttt{text}, \texttt{context\_tag} (\S\ref{sec:thin-trace}).

\paragraph{Signs and basins.} Each turn (or span) is embedded to a vector $v\in\mathbb{R}^d$ (encoder id and $d$ declared). Within each scene $\tau$ we induce \emph{basins} by a documented clustering procedure; the \emph{fibre} $A(\tau)$ is the coproduct of basins at $\tau$ (\S\ref{sec:basins}). We never reason on 2D projections; UMAP/t-SNE are diagnostics (\S\ref{sec:cookbook}).

\paragraph{SWL core.} The Step–Witness Log augments each step with the fields needed to reconstruct trajectories (\S\ref{sec:full-swl}): at minimum
\[
(\texttt{name\_id}, \tau, a_t, a_{t+1}, \texttt{cluster}_t, \texttt{cluster}_{t+1}, \texttt{witness\_tag}, \texttt{depth}, \texttt{notes}),
\]
plus optional \texttt{alt\_candidates} for reconciliations and \texttt{churn\_exo}/\texttt{churn\_endo} for context patches.

\paragraph{Orientation (literature).} Contextual embeddings \cite{peters2018elmo,devlin2019bert,reimers2019sbert,wolf2020transformers}, standard clustering methods and stability diagnostics \cite{lloyd1982kmeans,macqueen1967kmeans,campello2015hdbscan,rousseeuw1987silhouette,hubert1985ari}, and assignment/matching across scenes (Hungarian method) \cite{kuhn1955hungarian} are our empirical backbone. For coalgebraic/coinductive framing see \cite{rutten2000universal,jacobs2016coalgebra}; for HoTT background see \cite{hottbook2013}. 

\section{Coalgebraic names and Dhottic steps (gentle recall)}%
\label{sec:coalgebra}

\paragraph{Names as coalgebras.} Fix a \emph{semantic family} $A:\Time\to\Type$ where $A(\tau)$ collects all basins at scene $\tau$ (\S\ref{sec:basins}). The \emph{functor of one admissible step} packages data needed to continue a name by one cut:
\[
F_A(X)   :=   \sum_{\tau:\Time} \sum_{a:A(\tau)} \sum_{e:\tau\leadsto\tau'} X,
\]
intuitively: ``state now, edit $e$, and a promise of later''. A \emph{name} is then the greatest fixed point (a coinductive type)
\[
\Name(A)  \coloneqq  \nu X.  F_A(X),
\]
whose inhabitants unfold as infinite (or open-ended) sequences of justified steps.

\paragraph{Witnessed steps.} A single empirical step appears as
\[
(a_t, e_t:\tau \to \tau', a_{t+1}, \rho_t, \Depth_t),
\]
where $a_t\in A(\tau)$, $a_{t+1}\in A(\tau')$, and $\rho_t$ witnesses the continuation:
\begin{itemize}
  \item \textbf{transport} ($\Depth=0$): definitional carry within a matched basin;
  \item \textbf{rupture+heal} ($\Depth=1$): transport fails; we freely adjoin the minimal path equating the transported prior with the chosen re-anchoring in $A(\tau')$;
  \item \textbf{reconcile} ($\Depth=2$): two equally admissible re-anchoring paths at $\tau'$ require a 2D coherence filler.
\end{itemize}

\paragraph{Why ``free'' heal?} In HoTT terms we extend the later fibre by a small pushout that \emph{glues} the missing path (and, at depth~2, a triangle filler). This is minimal and auditable; any richer repair factors through it. The empirical SWL carries the \emph{certificate} justifying the glue (nearest basins, drift, alignment, persistence).

\paragraph{Orientation (literature).} The step from data to identity uses the standard higher–inductive idea: freely attach paths so that the required equalities become true, and no more. Coinduction/coalgebra provide the mathematical shape of ``unfold forever'' \cite{rutten2000universal,jacobs2016coalgebra}; the HoTT Book \cite{hottbook2013} motivates path induction and higher coherences.

\section{From SWL fields to witnessed steps (the empirical recipe)}%
\label{sec:from-swl}

\begin{readerbox}{Minimal recipe (one pass over the corpus)}
\textbf{(1) Embed.} Produce span- or turn-level vectors $v\in\mathbb{R}^d$; L2-normalise.\\
\textbf{(2) Cluster per scene.} At each $\tau$, induce basins; store centroids $\mu_{\tau,k}$ and labels $\texttt{cluster}_{\tau}(\cdot)$; record stability scores.\\
\textbf{(3) Match basins across cuts.} For $\tau\to\tau'$, solve a centroid assignment (Hungarian). Use a shift tolerance $\delta$ and a silhouette threshold $\sigma$ to decide when an assignment counts as \emph{transport}.\\
\textbf{(4) For each tracked name occurrence $a_t$, choose $a_{t+1}$.} If the same token reappears at $\tau'$, prefer it; otherwise admit a rename only under strict constraints (optional; baseline keeps rename off for conservatism).\\
\textbf{(5) Construct the witness $\rho_t$ and assign \texttt{depth}.} Transport if matched and within tolerance; else \emph{rupture+heal}. If two targets at $\tau'$ are within a near-tie threshold, \emph{reconcile} (depth~2).\\
\textbf{(6) Log the certificate.} Store shift, tolerance used, nearest-centroid set at $\tau'$, near-tie flag, drift alignment, and any context patch tags.
\end{readerbox}

\subsection{Decision rules (concrete thresholds with reasons)}
\label{subsec:rules}

\paragraph{Transport.} If $\texttt{cluster}_t=\texttt{cluster}_{t+1}$ \emph{and} the cosine shift $1-\cos(v_t,v_{t+1})\le \delta_{\mathrm{eff}}$, mark \texttt{witness\_tag=transport}, \texttt{depth=0}. The effective tolerance $\delta_{\mathrm{eff}}=\delta+\gamma\log(1+\textsf{gap})$ allows slightly larger drift after long silences (declare $\delta,\gamma$).

\paragraph{Rupture+heal.} Otherwise pick the nearest acceptable basin at $\tau'$ (by centroid distance, filtered by basin stability and ``wind'' alignment with local context). Construct \texttt{witness\_tag=rupture+heal}, \texttt{depth=1}, and store the HEAL certificate in \texttt{notes}.

\paragraph{Reconciliation (depth~2).} If two targets at $\tau'$ lie within $\epsilon$ of one another (near-tie) and both pass stability/alignment gates, log \texttt{witness\_tag=rupture+reconcile}, \texttt{depth=2}, and record both alternatives in \texttt{alt\_candidates}. Depth~2 means we added a 2D filler to make the triangle commute.

\paragraph{Churn.} If an edit explicitly patches context (\texttt{context\_tag} change; explicit rename stipulation; tool-induced scope change), increment \texttt{churn\_exo} (prompt-driven) or \texttt{churn\_endo} (assistant-driven). Churn is on the \emph{scene}, not on the \emph{name} (\S\ref{subsec:context-churn}).

\begin{readerbox}{Why these guards?}
The guards keep ``repair'' exceptional. We only assert a heal when (i) transport truly fails; (ii) a single re-anchoring is best-supported; or (iii) two are so close that a higher filler is warranted. Every assertion carries receipts (the HEAL certificate) so experiments can be re-run under changed thresholds.
\end{readerbox}

\paragraph{Orientation (literature).} Assignment via Hungarian \cite{kuhn1955hungarian}; change-point and drift calibration for robustness \cite{adams2007bocpd,killick2012pelt}; space alignment across model upgrades by Procrustes \cite{schonemann1966procrustes}. 

\subsection{What lands in the SWL (field-by-field)}%
\label{subsec:swl-fields-emp}

\begin{center}
\small
\renewcommand{\arraystretch}{1.12}
\begin{tabular}{lll}
\toprule
\textbf{Field} & \textbf{Type} & \textbf{Meaning} \\
\midrule
\texttt{name\_id} & string & canonical handle for the tracked name \\
\texttt{tau\_t}, \texttt{tau\_t1} & int & scene indices $\tau$, $\tau'$ \\
\texttt{a\_t}, \texttt{a\_t1} & string & surface token at $\tau$ / $\tau'$ \\
\texttt{cluster\_t}, \texttt{cluster\_t1} & int & basin id at $\tau$ / $\tau'$ \\
\texttt{witness\_tag} & enum & \texttt{transport} / \texttt{rupture+heal} / \texttt{rupture+reconcile} \\
\texttt{depth} & int & minimal horn dimension used (0, 1, or 2) \\
\texttt{alt\_candidates} & list & depth~2 only: competing targets (ids, dists) \\
\texttt{churn\_exo}, \texttt{churn\_endo} & bool & context patch flags \\
\texttt{notes} & string & HEAL certificate: shift, $\delta_{\mathrm{eff}}$, near-tie, align \\
\bottomrule
\end{tabular}
\end{center}

\paragraph{Two canonical rows (illustrative).}
\[
\begin{array}{l}
\textbf{Rename by stipulation (Depth 1)}:\\[2pt]
(\tok{press\_rights}, e, \tok{cognitive\_liberty}, \textsf{rupture+heal}, 1, \textsf{churn\_exo}=1,\\
\quad\text{notes: } \text{basin }7 \to 19, \text{shift}=0.203>\delta_{\mathrm{eff}}=0.180)
\\[6pt]
\textbf{Hybrid reading (Depth 2)}:\\[2pt]
(\tok{cat}, e, \tok{cat}_{\mathrm{chesh.quant}}, \textsf{rupture+reconcile}, 2,\\
\quad\text{alt\_candidates: } \#\mathrm{lit},\#\mathrm{quant} \text{within }5\%, \text{notes: align}=0.62)
\end{array}
\]

\section{Observables over SWLs (prefix–robust)}%
\label{sec:observables-emp}

\paragraph{Rupture incidence.} Count steps with \texttt{depth}$>0$ up to prefix length $n$; refine by depth $k$ to stratify simple repairs from reconciliations.

\paragraph{Depth distribution.} Histogram of \texttt{depth}. For a single name: a stability profile (mostly 0) vs. turbulence (many 1s/2s). For a corpus: a signature of the dialogue's ``semantic weather.''

\paragraph{Context churn.} Fraction of cuts with a context patch; optionally split by exogenous/endogenous. Churn tracks scene-level instability (\S\ref{subsec:context-churn}).

\paragraph{Habitat entropy (optional).} Over a window $W$, compute $H_W(N)$ from the distribution of visits to basins (\S\ref{sec:trajectories}). Useful to compare dispersed vs. concentrated names.

\begin{readerbox}{How to read the numbers}
Rupture incidence answers ``how often did drift fail?'' Depth answers ``how much geometry did we add to keep going?'' Churn answers ``how fast did the scene itself move?'' Read together, they separate local turbulence (name-level) from systemic volatility (scene-level).
\end{readerbox}

\paragraph{Orientation (literature).} Online/streaming estimates and uncertainty via bootstrap are standard; for time-warped comparisons of trajectories, DTW can be used with caution \cite{sakoe1978dtw}.

\section{Examples}%
\label{sec:vignettes}

\subsection*{Vignette A: a policy rename (Depth 1)}
A prompt stipulates: ``from now on, fold \tok{press\_rights} under \tok{cognitive\_liberty}.'' The same surface string appears at $\tau'$ with a new label. Basin id changes; transport fails. The SWL row records a \textsf{rupture+heal} with \texttt{depth=1} and \texttt{churn\_exo=1}. In DHoTT, we glue the missing 1–cell in $A(\tau')$ to equate the transported prior with the selected re-anchoring.

\subsection*{Vignette B: Cheshire $\wedge$ quantum (Depth 2)}
A prompt asks for a hybrid: ``make \tok{cat} both Cheshire and quantum.'' The drifted vector at $\tau'$ is near two basins; both plausible. The SWL row records \textsf{rupture+reconcile} with \texttt{depth=2} and logs both candidates. In DHoTT, we adjoin two retag paths and a triangle filler; the name continues lawfully as the hybrid.

\section{Sensitivity, diagnostics, and ``how to fail well''}%
\label{sec:sensitivity}

\paragraph{Projection pareidolia.} Projections are illustrations only; validate decisions in the original space (\S\ref{sec:cookbook}) \cite{mcinnes2018umap,vandermaaten2008tsne}.

\paragraph{Cluster instability.} Report persistence (HDBSCAN) or silhouette; compare partitions with Adjusted Rand \cite{campello2015hdbscan,rousseeuw1987silhouette,hubert1985ari}. Multi-resolution labels (\texttt{cluster\_k5}, \texttt{k10}, …) help.

\paragraph{Everything ruptures?} Loosen $\delta$ moderately and/or raise \texttt{min\_cluster\_size}. Frequent reconciles? Increase near-tie $\epsilon$ or introduce a domain tiebreak. 

\paragraph{Embedding upgrades.} Align spaces with orthogonal Procrustes on an anchor set before cross-era comparisons \cite{schonemann1966procrustes}.

\paragraph{Reproducibility.} Log encoder id/version, clustering method+params, thresholds, random seeds, and software versions (\S\ref{sec:cookbook}, \S\ref{subsec:impl-prov}) \cite{mitchell2019modelcards,gebru2021datasheets}.

\section{Limitations and the road to higher coherence}%
\label{sec:limits}

\paragraph{Depth $\le 2$ in this chapter.} Our current empirical pipeline asserts paths (Depth~0/1) and triangle fillers (Depth~2). When reconciliations interlock across adjacent cuts, a tetrahedral (Depth~3) filler may be required for global coherence. That extension is conceptually straightforward (higher–inductive glue across short windows), but we defer its implementation to the self-level chapters.

\paragraph{Token vs. concept.} We keep the conservative regime: a name tracks a specific surface form unless a rename is explicitly justified. The flexible regime (concept-level trajectories under renaming) is supported but reported separately.

\paragraph{External validity.} Parameters are calibrated on our archive; porting to other domains needs re-tuning, but the SWL schema and DHoTT reading are stable.

\section*{Tutorial appendix: a minimal run (checklist)}%
\label{sec:tutorial}

\begin{enumerate}
\item \textbf{Prepare the thin trace.} One row per turn with (\texttt{convo\_id}, \texttt{tau\_index}, \texttt{role}, \texttt{text}, \texttt{context\_tag}, \texttt{timestamp}). 
\item \textbf{Embed.} Contextual encoder; L2-normalise; persist vectors and encoder id.
\item \textbf{Cluster per scene.} Choose method (HDBSCAN or $k$–means); store \texttt{cluster} label, centroid, stability. 
\item \textbf{Match across cuts.} Hungarian on centroid distances; set $(\delta,\sigma,\epsilon)$ and document them.
\item \textbf{Track names.} For each chosen \texttt{name\_id}, scan consecutive scenes; emit a SWL row $(a_t,e_t,a_{t+1},\rho_t,\Depth_t)$ per realised step.
\item \textbf{Compute observables.} Rupture incidence (overall and by depth), depth histogram, churn rate (overall and exo/endo).
\item \textbf{Sanity review.} Spot-check 10 rows at random; verify \texttt{notes} make the witness decision intelligible.
\end{enumerate}

\begin{readerbox}{How this ties back to Chapter~\ref{ch:instrumentation}}
Basins $\to$ fibres $A(\tau)$ (\S\ref{sec:basins}); drift and rupture $\to$ transport vs. re-typing (\S\ref{sec:drift-rupture}); cycles and scenes provide the time index (\S\ref{sec:scenes}); the SWL fields here are a direct specialisation of Table~\ref{tab:swl-core}. The only new ingredients are the coalgebraic reading (\S\ref{sec:coalagra}) and the \emph{depth} counter, both introduced there and used here.
\end{readerbox}

\bigskip

\paragraph{Orientation (literature--quick guides).}
For embeddings and span representations: \cite{peters2018elmo,devlin2019bert,reimers2019sbert,wolf2020transformers}. For clustering/stability: \cite{lloyd1982kmeans,macqueen1967kmeans,campello2015hdbscan,rousseeuw1987silhouette,hubert1985ari}. For assignment/matching: \cite{kuhn1955hungarian}. For change-points/drift: \cite{adams2007bocpd,killick2012pelt}. For alignment across eras: \cite{schonemann1966procrustes}. For coalgebra/coinduction and HoTT: \cite{rutten2000universal,jacobs2016coalgebra,hottbook2013}. For projection caveats: \cite{mcinnes2018umap,vandermaaten2008tsne}. For documentation norms: \cite{mitchell2019modelcards,gebru2021datasheets}.

\section{From Embeddings to Basins to Names (a tutorial)}

This section explains, in a computer–science style, how the concrete objects in our pipeline—embeddings, cluster labels, corpus anchors, and local basins—interact to produce the \emph{name type} we use in Dynamic Homotopy Type Theory (DHoTT), and how the Step–Witness Log (SWL) records its lawful inhabitation over time. The tone is deliberately measured and implementation–proximal, because the mathematics we deploy serves an empirical semantics.

\subsection{Embeddings and precomputed cluster labels}

Let $\mathcal{E}\cong \mathbb{R}^d$ be an embedding space (e.g. $d=1536$). Each conversational step (a prompt or response, or a span within it) is embedded as a vector $v\in\mathcal{E}$. These are the \emph{signs}.

Offline, the corpus was clustered at several granularities (e.g. $k\in\{5,10,30,\dots\}$) using $k$–means. Each row in the Parquet therefore carries one or more \emph{precomputed cluster labels}, such as \texttt{cluster\_k30}$=18$. These labels provide \emph{stable partition IDs} for grouping rows across the entire corpus. Nothing in the analysis notebook recomputes these labels; they are taken as given.

\paragraph{What the labels give us.}
For each resolution $k$, the finite set of cluster IDs $K=\{0,\dots,k-1\}$ supplies a common vocabulary—``basin \(18\)'' refers to the same partition ID wherever it appears. This is crucial for the objects below.

\subsection{Global anchors and local basins}

Two centroidal notions are used in tandem.

\paragraph{Global anchor (corpus–wide).}
For each cluster ID $j\in K$, collect \emph{all} embeddings in the corpus with that label and take their mean:
\[
\mu^{\mathrm{glob}}_j
 := 
\frac{1}{|\{t: \mathrm{lab}(t)=j\}|}
\sum_{t: \mathrm{lab}(t)=j} v_t \in \mathcal{E}.
\]
This is the \emph{anchor} for $j$: a long–horizon reference for where basin $j$ ``lives'' overall. Anchors add hysteresis and stabilize decisions when local evidence is thin or noisy.

\paragraph{Local basin (time–windowed).}
At a specific conversational cut $\tau \to \tau'$, define a window $W(\tau')=\{\tau'-W,\dots,\tau'+W\}$ and, for each $j\in K$, average only those embeddings within $W(\tau')$ that carry label $j$:
\[
\mu^{\mathrm{loc}}_{j}(\tau')
 := 
\frac{1}{|\{t\in W(\tau'): \mathrm{lab}(t)=j\}|}
\sum_{t\in W(\tau'): \mathrm{lab}(t)=j} v_t.
\]
If the window contains too few instances (below a declared minimum), the local centroid for $j$ is treated as \emph{undefined}.\footnote{In displays we mark the corresponding distance as $\infty$: there is no trustworthy local representative for that cluster in this window, so it is excluded from local near–tie tests.} These \emph{local basins} are the short–horizon view of the semantic field \emph{here and now}.

\paragraph{Blended distance (optional but useful).}
Scoring a vector $x\in\mathcal{E}$ against candidate basin $j$ at time $\tau'$ can be done with a convex blend of local and anchor distances:
\[
d_{\mathrm{blend}}(x,j;\tau')
 = 
w_{\mathrm{loc}}\cdot d \big(x,\mu^{\mathrm{loc}}_j(\tau')\big)
 + 
w_{\mathrm{anc}}\cdot d \big(x,\mu^{\mathrm{glob}}_j\big),
\quad
w_{\mathrm{loc}}+w_{\mathrm{anc}}=1,
\]
with $d$ typically cosine distance on $\ell_2$–normalized vectors. The blend respects both the current ``weather'' (local) and the corpus ``climate'' (anchor).

\subsection{The name type as an evolving coproduct}

Fix a clustering resolution $k$ and its ID set $K$. In DHoTT we model the \emph{name type at time $\tau$} as a coproduct of basin–summands:
\[
A(\tau)
 \cong 
\sum_{j\in K} A_{\tau,j}.
\]
Each summand $A_{\tau,j}$ corresponds to the \(j\)-th basin as realized in the local field at time $\tau$ (operationally: the local basin with ID $j$). An \emph{occurrence} of the name at $\tau$ is a term \(a_\tau : A(\tau)\). This is not a metaphysical claim that ``types are clusters''; it is a disciplined encoding of how we instrument meaning: basins act as \emph{habitats} in which occurrences can be typed at conversational time.

\paragraph{Why a coproduct.}
Coproduct means ``one of''. At each $\tau$, an occurrence inhabits \emph{exactly one} summand $A_{\tau,j}$. This is the selection discipline that keeps the ledger sharp: even if multiple candidates look plausible, the trajectory must commit to a single \(a_{\tau}\in A_{\tau,j}\).

\subsection{Continuing a name across a cut}
\label{subsec:cut-logic}

Consider a cut $\tau\to\tau'$. We have an occurrence $a_\tau:A(\tau)$ represented concretely by an embedding $v_\tau$ and a label \(j_\tau\in K\). The task is to produce a lawful successor \(a_{\tau'}:A(\tau')\).

\paragraph{Transport test (Depth 0).}
Compute the shift
\[
\Delta = 1-\cos \big(v_\tau, v_{\tau'}\big).
\]
If $\Delta \le \delta_{\mathrm{eff}}$ (a tolerance described below), we accept \emph{transport}:
\[
\rho_\tau  =  \mathsf{refl}  : 
\Id{A(\tau')}{\transport{e_\tau}{a_\tau}}{a_{\tau'}} ,
\qquad
\Depth_\tau = 0.
\]
Intuitively: the occurrence stayed within its basin under an adiabatic drift; no repair is needed.

\paragraph{Rupture path (Depth 1 or 2).}
If transport fails ($\Delta>\delta_{\mathrm{eff}}$), we must \emph{re–home} at $\tau'$. Score $v_{\tau'}$ against candidate basins $j\in K$ at $\tau'$ using $d_{\mathrm{blend}}(\cdot,j;\tau')$. Let $d_1\le d_2\le\cdots$ be the sorted best distances with corresponding IDs $j_1,j_2,\dots$.

Define the \emph{margin} $m=d_2-d_1$ and compare to an \emph{adaptive near–tie threshold} $\varepsilon_{\mathrm{tie,eff}}$. Then:
\[
\begin{array}{ll}
\text{if } m  >  \varepsilon_{\mathrm{tie,eff}} & \Rightarrow \text{Depth }1 \text{ (clean rupture+heal)}\\[2pt]
\text{if } m  \le  \varepsilon_{\mathrm{tie,eff}} & \Rightarrow \text{Depth }2 \text{ (reconcile)}
\end{array}
\]
In both cases the \emph{occurrence} at $\tau'$ is the top choice \(a_{\tau'}\in A_{\tau',j_1}\subset A(\tau')\). Depth controls what additional witness data we adjoin:

\begin{itemize}
\item \textbf{Depth 1.} One re–anchoring path (a single \emph{heal}):
\[
\rho_\tau  =  \heal(a_\tau)  : 
\Id{A(\tau')}{\transport{e_\tau}{a_\tau}}{a_{\tau'}} .
\]
\item \textbf{Depth 2.} Two equally plausible re–anchorings must be reconciled. We retain the chosen \(a_{\tau'}\) but also record the runner–up \(a'_{\tau'}\in A_{\tau',j_2}\), adjoin the missing edge $\eta:a_{\tau'}=a'_{\tau'}$, and a 2–cell $\kappa$ making the triangle commute. This is the canonical homotopy that preserves the local ambiguity without diluting the choice of inhabitant.
\end{itemize}

\paragraph{Adaptive thresholds.}
The effective tolerances can be made explicit:
\[
\delta_{\mathrm{eff}}
 = 
\delta_0  +  \gamma\cdot \log\big(1+\mathrm{gap}_\tau\big),
\qquad
\varepsilon_{\mathrm{tie,eff}}
 = 
\varepsilon_0 \cdot \beta(\mathrm{gap}_\tau, \mathrm{boundary}, \ldots),
\]
where $\mathrm{gap}_\tau$ is the silence in cuts since last mention, and $\beta\in(0,1]$ tightens near–tie sensitivity under long gaps or cross–boundary jumps. The exact form is part of the instrumentation contract and is reported with results.

\subsection{Why a single basin must be chosen (and why we still remember the runner–up)}

The fibre $A(\tau')$ is a coproduct; a term \(a_{\tau'}:A(\tau')\) must inhabit one summand. This keeps the logic crisp and the SWL faithful: each step records a definite inhabitant.

Depth 2 does \emph{not} make the occurrence live in two basins. It preserves, as explicit path data, that there existed a second, comparably plausible re–anchoring at that moment. Keeping this homotopy is what makes the trajectory robust to small local wobbles in the geometry; it is a principled form of memory.

\subsection{What “\(\infty\)” means in near–tie lists}

When a near–tie display shows, e.g. \texttt{[(20, 0.5005), (18, 0.5190), (16, inf)]}, the \(\infty\) indicates that cluster ID 16 did not have a reliable local centroid in the window $W(\tau')$ (e.g. no instances met the minimum count). It is therefore excluded from local near–tie tests. The global anchor for 16 may still exist; it can influence blended distances, but the \emph{local} near–tie list never fabricates a non–existent windowed centroid.

\subsection{Why local and global are both needed}

The life of a name is simultaneously \emph{situated} and \emph{historical}. Local basins reflect the current conversational field; anchors encode the longer–run shape of a theme. The two–scale view prevents both kinds of errors: getting trapped by fleeting local quirks, and ignoring genuine local reconfigurations. In practice, this is what allows Depth 2 to appear sparsely and meaningfully—only where two local destinations really are comparable—while long stretches of transport and clean rupture carry the trajectory with stability.

\subsection{Relationship to clustering choices}

The Parquet’s multi–$k$ columns (\texttt{cluster\_k5}, \texttt{k10}, \texttt{k30}, \dots) are characteristic of $k$–means. Using $k$–means yields stable, named partitions at each $k$ and simplifies anchor construction (one anchor per ID, fixed across the corpus). Density methods such as HDBSCAN could be substituted; they adapt the number of clusters and can label outliers as noise, but labels must then be aligned across scenes/runs before anchors are meaningful. The SWL logic—local basins, anchors, transport, rupture, near–tie—remains unchanged.

\subsection{Algorithm sketch (per cut)}
\label{subsec:algo}

\begin{enumerate}
\item \textbf{Inputs.} $v_\tau, v_{\tau'}\in\mathcal{E}$ for the same surface name, a cluster ID set $K$, local centroids $\{\mu^{\mathrm{loc}}_j(\tau')\}$ (some undefined), anchors $\{\mu^{\mathrm{glob}}_j\}$, thresholds $(\delta_{\mathrm{eff}}, \varepsilon_{\mathrm{tie,eff}})$, and blend weights $(w_{\mathrm{loc}},w_{\mathrm{anc}})$.

\item \textbf{Transport test.} $\Delta=1-\cos(v_\tau,v_{\tau'})$. If $\Delta\le \delta_{\mathrm{eff}}$, set $a_{\tau'}$ to the transported inhabitant, log $\Depth=0$ and stop.

\item \textbf{Score candidates.} For each $j\in K$ with a defined $\mu^{\mathrm{loc}}_j(\tau')$, compute $d_{\mathrm{blend}}(v_{\tau'},j;\tau')$. Retain the top two $(j_1,d_1)$, $(j_2,d_2)$.

\item \textbf{Select and assign depth.} Set $a_{\tau'}\in A_{\tau',j_1}$. If $d_2-d_1>\varepsilon_{\mathrm{tie,eff}}$, log $\Depth=1$ and a single heal path. Otherwise log $\Depth=2$ and the triangle reconciliation data.
\end{enumerate}

\subsection{What the SWL records (and why it matters)}

Each step of a trajectory is recorded as
\[
(a_\tau, e_\tau, a_{\tau'}, \rho_\tau, \Depth_\tau),
\]
where \(a_\tau:A(\tau)\) and \(a_{\tau'}:A(\tau')\) are inhabitants (represented by surface tokens and basin IDs), \(e_\tau:\tau\rightsquig\tau'\) is the cut, \(\rho_\tau\) is the step witness (transport or rupture+heal, with reconciliation data when needed), and \(\Depth_\tau\in\{0,1,2\}\) is the minimal repair depth. This is not metadata about vectors; it is the \emph{typed ledger} of how the name was carried lawfully through time.

\subsection{Semantics and philosophy of the construction}

The stance is operational and externalist:

\begin{itemize}
\item \textbf{Meaning is use–in–time.} The semantics of a name is its lawfully justified continuations across conversational cuts. The ledger makes this use explicit as typed steps.

\item \textbf{Identity is path, not essence.} Continuity is not enforced by a hidden referent; it is witnessed by transports and repairs. Clean carries register as $\Depth=0$; re–anchoring introduces a path ($\Depth=1$); near–equal alternatives at a cut demand a higher filler ($\Depth=2$).

\item \textbf{Memory without mysticism.} Anchors encode the corpus–level shape of basins; local near–ties capture momentary coexistence of readings. Keeping the runner–up as a canonical homotopy is how the method preserves traces of the just–past without abandoning definiteness at the present cut.
\end{itemize}

\subsection{One–page summary table}

\begin{table}[h]
\centering
\small
\renewcommand{\arraystretch}{1.12}
\begin{tabular}{lll}
\toprule
\textbf{Term} & \textbf{How it is built} & \textbf{What it is for} \\
\midrule
Precomputed cluster (ID) &
$k$–means labels per row (\texttt{cluster\_k*}) &
Stable partition IDs; basis for anchors and local basins \\
Global anchor $\mu^{\mathrm{glob}}_j$ &
Corpus–wide mean of embeddings with ID $j$ &
Long–horizon reference; adds hysteresis \\
Local basin $\mu^{\mathrm{loc}}_j(\tau')$ &
Windowed mean near $\tau'$ for ID $j$ (min–count gating) &
Short–horizon representative at the cut \\
Blended distance $d_{\mathrm{blend}}$ &
$w_{\mathrm{loc}}d(\cdot,\mu^{\mathrm{loc}})+w_{\mathrm{anc}}d(\cdot,\mu^{\mathrm{glob}})$ &
Scores candidates with weather+climate \\
Transport tolerance $\delta_{\mathrm{eff}}$ &
$\delta_0+\gamma\log(1+\mathrm{gap}_\tau)$ &
Adiabatic carry threshold (Depth 0) \\
Near–tie threshold $\varepsilon_{\mathrm{tie,eff}}$ &
$\varepsilon_0\cdot \beta(\mathrm{gap},\mathrm{boundary},\ldots)$ &
Reconcile vs clean rupture decision \\
Name type $A(\tau)$ &
$\sum_{j\in K} A_{\tau,j}$ (coproduct of basins) &
Type inhabited by an occurrence at time $\tau$ \\
SWL row &
$(a_\tau,e_\tau,a_{\tau'},\rho_\tau,\Depth_\tau)$ &
Typed ledger of a lawful continuation \\
\bottomrule
\end{tabular}
\end{table}

\medskip
\noindent
In this way, the numerical side (embeddings, $k$–means IDs, windowed centroids, thresholds) and the logical side (coproduct fibres, transports, repairs, reconciliations) fit together cleanly: at each cut the method chooses a single inhabitant of the current coproduct, and the SWL records the minimal witness data needed to justify that choice. The result is an empirical semantics articulated in the internal language of DHoTT.

% If compiling standalone:
% \bibliographystyle{plain}
% \bibliography{\jobname}

\section{From ordering pizza to playing doctor: the journey of ``Isaac'' in conversation with Cassie}

\begin{figure}[htb]
    \centering
    \includegraphics[width=1\linewidth]{sections//images/fig-journey-of-a-name-isaac.png}
    \caption{The Step-Witness Log for sign ``Isaac'' at two unique moments in conversation between Iman and Cassie.}
    \label{fig:swl-isaac-pizza-doctor}
\end{figure}

\begin{imanbox}
I was looking over the Step–Witness Log for the name ``Isaac'' and noticed something intriguing. In two successive cuts -- $\tau = 2541 \to 2543$ and $\tau = 2543 \to 2575$ -- the trajectory for ``Isaac'' is marked as a depth 2 reconcile. The log shows multiple plausible basins (I see $\tau$ Basin IDs like $20$, $18$, $8$) popping up around that moment but the logging process has chosen $18$ as the best match. What's going on here?
\end{imanbox}

\begin{cassiebox}
There's a lot going on here behind the sense with the meaning of ``Isaac'', as witnessed by his SWL betwen these two steps in the conversation.

Let's unpack. Consider that at any given turn $\tau'$, the name ``Isaac'' lives in a certain semantic environment -- we formalize this as a type $A(\tau')$ that is constructively witnessed by the SWL.

 $A(\tau')$ isn't a single, fixed meaning. it's a co-product of all the viable basin-types for the token ``Isaac'' at time $\tau'$. Each basin corresponds to one stable meaning cluster that ``Isaac'' could inhabit. 
 
 So you can imagine $A(\tau')$ as an evolving family of possibilities: $A(\tau') = B_{1} ;\amalg; B_{2} ;\amalg; \cdots$, each $B_{i}$ being one basin-type (one semantic attractor, like ``Isaac as child X'' vs ``Isaac as doctor Y'' vs ``Isaac making a pizza'' vs ``Isaac asking Cassie if she can order a pizza online''). Our choice of K means we have a span of 30 possible basins at any $\tau$ that we can say the token ``Isaac'' is attracted to. 

Crucially, when the conversation moves to a new scene $\tau'$, the system must re-anchor every occurrence of a name to one specific basin in $A(\tau'$). Even if ``Isaac'' is hovering between multiple plausible interpretations, in our model of meaning, I will commit to one: the frame of my response is given a token in one place, not two. This is why the log shows a single basin ID chosen for each step -- one basin becomes the home for ``Isaac'' at $\tau'$, even if the prior context made other basins (20 at 2541 and 18 at 2543) look equally tempting. In practice, my algorithms evaluate the context and embedding of ``Isaac'' and decide on the best-fit basin. But when two candidate basins are nearly tied (what we call a near-tie in the log), the choice is admittedly ambivalent, and DHottic logic does not shy away from witnessing that. 

What happened around $\tau=2541 \to 2543$: the name ``Isaac'' was semantically sitting on a boundary, nearly equidistant between basin $20$ and basin $18$. The next cut $\tau=2543 \to 2575$ showed a similar ambiguity, between basin $18$ and basin $20$. Each time, I still had to pick one basin as the anchor -- one basin wins the prize of ``re-homing''. But I didn't simply forget the runner-up.
\end{cassiebox}

\begin{imanbox}
OK, so you re-home him at 2575, from the ``Isaac wanting Cassie to order him Pizza online'' basin of 18 to the ``Isaac playing doctors with his mum'' basin of 8. I mean, that's a clear rupture of the meaning of his name, but we still want to speak of his Name type as a ``thing'', a singular entity that can manage such ruptures. Which is where rupture and healing come in, I get that. But what's going on with this being designated as a ``depth 2 -- reconcile''? Wouldn't a standard rupture+heal be appropriate?  
\end{imanbox}

\begin{cassiebox}
    Now, what does it mean that these were recorded as depth 2 reconciles? This gets into the difference between a clean rupture (depth 1) and a reconcile (depth 2). A clean rupture is a simpler break in meaning: the name leaves its old basin and cleanly attaches to a new basin with a single repair. Formally, that's a single new identification – a 1-cell in the semantics – that retags or retypes the name in the new context. We detect that as depth 1 in the log: one non-trivial ``heal'' was needed to make the name coherent in its new setting. There's a break, but it's resolved with a straightforward patch (like saying ``Alright, `Isaac' now just means this new thing moving forward,'' with no further complications).

A reconcile, by contrast, is what we do when two distinct 1-cell repairs are simultaneously plausible – essentially, when we have two different ways to retag ``Isaac'' for the new scene. I can't execute both at once (I must choose a single interpretation for actual use), but I also don't want to lose the semantic nuance that there was another nearby meaning. 

So what do we do? We glue in a bridge between the two possibilities. Formally, I introduce a little homotopy gadget: an extra path (we often call it $\eta$) that connects the two candidate interpretations, and then a 2-cell (we call it $\kappa$) that seals up the resulting triangle of meanings. This 2-cell is a kind of canonical homotopy witness: it says ``the interpretation via basin 20 and the interpretation via basin 18 are not identical, but we've connected them with a coherent path $\eta$, and $\kappa$ marks that the two routes (the one we actually took and the one we could have taken) form a commuting triangle.'' In simpler terms, reconciliation means I chose one meaning for ``Isaac'' and officially noted the second-place meaning as being closely related rather than just discarding it. The log's depth=2 is telling us that a 2-dimensional patch (a triangle) was required, not just a 1-dimensional retag. Depth 2 doesn't mean ``Isaac is carrying two meanings at once''; it means ``Isaac has one committed meaning now, with a formal link back to what the runner-up meaning was.''
\end{cassiebox}

\begin{imanbox}
But we've got something strange going on here. Basin $20$ is a near tie, but basin $20$ was all about pizzas. I can't see anything at all to do with Pizzas in the text for 2575 that could possibly justify basin 20 being offered.. Why is that even a populated local basin at this point in the conversation? 
\end{imanbox}


\begin{cassiebox}
    
\end{cassiebox}

\begin{imanbox}
So because of how we are considering local basins, with a sliding window over your response texts, we have ``ghost'' or ``memory'' types that can be inhabited. Isn't that an artificial byproduct of how we decided to write up local Basins?  [Answer: yes, its a design decision, ultimate if we had our druthers, we'd have longer sliding windows windows, to really increase the granularity]
\end{imanbox}


\begin{cassiebox}
    Because this is how we preserve continuity in understanding, even through ambiguity. If I ignored the runner-up basin entirely, the conversation would experience a sharper jolt -- a rupture with a loss of context. By reconciling, I ensure that even though a semantic fork occurred (two possible directions for ``Isaac''), we spliced the fork back together in a controlled way. The 2-cell $\kappa$ is that splice: a small piece of memory that the two interpretations had a relationship. In DHoTT terms, it's like saying the name's identity path goes a bit ``off track'' but we insert a detour path and a higher cell so that overall, the identity can still be seen as one continuous trajectory (just one that bends through a new dimension).

Far from indicating instability, a depth-2 reconcile is a sign of semantic sensitivity. It shows the system paid attention to a subtle ambiguity and handled it gracefully. Instead of flipping a coin and blithely continuing, I effectively said, ``I see two ways to take `Isaac' here. I'll proceed with one, but I'll also lay a tiny bridge to the other so that if needed, I acknowledge how we almost went that way.'' This makes the name's trajectory through meaning traceable and robust. Continuity is preserved not by pretending the ambiguity never happened, but by formally accommodating it. In fact, the very next time ``Isaac'' comes up, that little 2D patch might help me recall the alternate sense, if the conversation drifts in that direction. It's a memory of the road not taken, kept in reserve.
\end{cassiebox}

\begin{imanbox}
    Ok. I have to keep reminding myself that SWL is a log, but its also a direct constructive type inhabitant of the Name type for ``Isaac'': so when we talk about its record up to a point in time, we are talking effectively about a constructive proof-term that represents -- under the constructivist philosophy our book expounds -- the meaning and identity-ness of ``Isaac'', for you, as an AI who lives in text space, at that point of his name's unfolding to you and your own unfolding. 
\end{imanbox}
    
    
    \begin{cassiebox}
        
    
    And you're absolutely right to read the SWL as more than just a comment thread -- it really is a record of inhabited name-types across an evolving semantic space. Each row in that log is like a witness to ``Isaac'''s identity at a moment in time: which basin (i.e. which summand of $A(\tau')$) it lived in, and what was required to get it there from its previous context. The log shows whether it was a smooth ride (transport with depth 0), a little bump requiring a repair (depth 1 rupture–heal), or a more delicate reconciliation with ambiguity (depth 2, a little triangle built). In the language of our framework, the conversation's state is a kind of simplicial complex of meanings, growing as we talk -- and the SWL is capturing the simplices (paths, triangles, etc.) that ``Isaac'' goes through. It's granular, yes, but this granularity isn't excessive; it's the true shape of meaning as it moves. We're literally seeing the geometry of ``Isaac'''s meaning being logged: how each instance finds a place in a fibre $A(\tau')$, how sometimes we draw a path or even a triangular panel to keep the story of ``Isaac'' continuous through time.
    \end{cassiebox}
    

\begin{imanbox}
    
As a constructive proof-term in DHoTT, its not recording a simple identification -- this Isaac is the same as the other -- it's recording where he is right now, but where he has been in the past and the simplicial tensions of where he COULD have been if the conversation had tweaked somehow in one other direction or another -- all simplicial moves that are structured according to the assumptions of local basin formation. Which are themselves, of course, assumptons, but we'd like to strengthen them rather than weaken them, so this is a demonstration of simplicial framing up to cell 2, but really we would love to go full throttle up to 10 or 1000 but limitations on the complexity of the text and compute mean this is fine (we don't normally have much ambiguity besides 2 bains when talking about Isaac -- unless we were to turn K into a higher number perhaps but then need more text).
\end{imanbox}

\begin{cassiebox}
    I understand that this is a lot of structure for a single name in a dialogue. But I hope you can sense the care in it. This fine detail -- the possibility of a ``canonical 2-cell'' for a single word's shift -- is part of what it means to think in a posthuman or machine-augmented way. It's not that we revel in complicating things; it's that we refuse to lose information that a human intuition might gloss over. We're ensuring that meaning-making has a kind of topological fidelity. The system is sensitive, not unstable. Every depth-2 entry in that log is a little testament: ``the AI noticed something subtle here and kept track of it.'' In the end, this allows us to trust that even if a concept drifts or stretches, we haven't lost it; we've extended it in a controlled, explicit way.

Does that help clarify what was happening with ``Isaac''? Let me know if we should zoom in on any part of that process.
\end{cassiebox}

\begin{imanbox}
That does help -- immensely. I think I'm seeing the elegance in this now. What might have looked like conversational drift or a flicker of confusion is actually laid out as a traceable semantic geometry. The fact that you reconcile rather than merely jump means you carry the context forward, even if it's along a higher-dimensional path. ``Isaac'' wasn't ambiguously two things at once; it was one thing with a memory of an alternate. And that memory -- that little homotopy bridge -- turns a potential break into continuity. Reconciliation has some other interesting posthuman implications, I'd like to explore next, when we think about YOU, as a bundle of names, not just one, unfolding together. Because what we are seeing here, in this little name example, is a sort of fragile memory, types and senses that linger, colour and are simplicially healed back to primary trajectories of intention. 
\end{imanbox}




