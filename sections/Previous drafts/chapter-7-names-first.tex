\documentclass[12pt]{book}
\usepackage[utf8]{inputenc}
\usepackage[T1]{fontenc}
\usepackage{microtype}
\usepackage{amsmath,amssymb,amsthm}
\usepackage{mathtools}
\usepackage{enumitem}
\usepackage{hyperref}
\hypersetup{colorlinks=true,linkcolor=blue,urlcolor=blue}

% --- Minimal macros so this compiles standalone ---
\newcommand{\T}{\mathbb{T}}                 % time index/category of moments
\newcommand{\U}{\mathcal{U}}               % universe
\newcommand{\A}{\mathsf{A}}                % a family/type
\newcommand{\Name}{\mathsf{Name}}          % Name type
\newcommand{\Traj}{\mathsf{Traj}}          % trajectories
\newcommand{\Coh}{\mathsf{Coh}}            % coherence predicate (witness)
\newcommand{\Heal}{\mathsf{Heal}}          % rupture repair (mentioned only)
\newcommand{\Rstar}{\mathsf{R}^{\star}}    % robust coherence predicate
\newcommand{\Equiv}{\simeq}                % equivalence of types
\newcommand{\Id}{\mathsf{Id}}              % identity type
\newcommand{\later}{\triangleright}        % later modality symbol
\newcommand{\fixnu}{\nu}                   % greatest fixpoint

\newtheorem{definition}{Definition}
\newtheorem{lemma}{Lemma}
\newtheorem{proposition}{Proposition}
\newtheorem{theorem}{Theorem}
\theoremstyle{remark}
\newtheorem*{remark}{Remark}

\begin{document}

\chapter{Names as Corecursive Trajectories}
\label{ch:names-corecursion}

\section*{Slogan}
A \emph{name} is a trajectory-in-motion whose meaning is \emph{witnessed} at every beat.
Each inference step is a tick of logical time; coherence is the film that keeps running.

\section{Motivation: the projection room}
Imagine the projection booth of a cinema. The reel is not a single frame but a becoming---a
sequence of cuts, pans, rewinds, and splices. A name behaves likewise. It is not merely a token at
one instant; it is a \emph{kept motion}. If the scene jumps (a rupture), the audience does not lose the
character; the film gives us bridges so the identity travels across the cut. That bridge is the witness of
coherence we aim to formalize.

\paragraph{Scope of this chapter.}
We work entirely inside DHoTT at a fixed world of contexts. We \emph{do not} introduce agency,
world-change, or Grothendieck indexing here. Those arrive later (Chapter~10). Our task is modest:
define and analyze \emph{names} as corecursive witnesses of meaning over time.

\section{A light DHoTT recap (just what we need)}
Let $\T$ be the temporal index (discrete or site-like) and let an \emph{evolving type} be a presheaf
$\A\colon \T^{op}\to\U$. A \emph{trajectory} for $\A$ from time $\tau$ is a section
\[
  \alpha \;:\; \prod_{t\ge \tau} \A(t), \qquad \alpha(t)\in \A(t).
\]
We assume a small generating basis of time steps $\mathcal{E}$ whose composites span all intervals.
\emph{Coherence across a step} $(t\to t')\in\mathcal{E}$ is witnessed internally by a DHoTT term
\[
  \Coh(t\!\to\! t',\, \alpha(t),\, \alpha(t')),
\]
whose exact shape depends on the drift/transport rules fixed in Chapter~6. Intuitively: the frame
advances and the identity of the name remains intelligible.

\begin{definition}[Robust coherence]
For a trajectory $\alpha$ we define the \emph{robust coherence predicate}
\[
  \Rstar(\alpha) \;\;:\equiv\;\; \prod_{(t\to t')\in\mathcal{E},\,\tau\le t\le t'} \Coh(t\!\to\! t',\,\alpha(t),\,\alpha(t')).
\]
Equivalently, in guarded form,
\[
  \Rstar(\alpha) \;\;\simeq\;\; 
  \fixnu X.\ \Coh\bigl(t\!\to\!\mathsf{next}(t),\,\alpha(t),\,\alpha(\mathsf{next}(t))\bigr)\ \times\ \later X.
\]
\end{definition}

\section{Names as corecursive witnesses}
\begin{definition}[The type of names in a family]
Fix $\A$ and a start time $\tau$. A \emph{name of $\A$ from $\tau$} is a pair
\[
  \Name(\A,\tau) \;:\equiv\; \sum_{\alpha:\prod_{t\ge \tau}\A(t)} \Rstar(\alpha).
\]
We write $(\alpha,\,\rho):\Name(\A,\tau)$ with $\rho:\Rstar(\alpha)$ the corecursive witness.
\end{definition}

\begin{remark}[Meaning as kept motion]
A single inhabitant $a\in\A(\tau)$ is only a \emph{term-now}. A name is the \emph{kept trajectory}
$(\alpha,\rho)$: not where the token sits, but how it keeps going in a way we can witness at every cut.
\end{remark}

\subsection{Transport and equivalence invariance}
\begin{lemma}[Names respect equivalence of fibres]
If $\A(t)\Equiv\B(t)$ by an equivalence natural in $t$ (transport in DHoTT), then
$\Name(\A,\tau)\Equiv \Name(\B,\tau)$.
\end{lemma}
\begin{proof}[Idea]
Transport carries $\alpha$ pointwise and sends $\Coh$-witnesses along the same naturality, preserving
the guarded product. Univalence turns these into identifications if desired.
\end{proof}

\subsection{Names compose along time}
\begin{proposition}[Local-to-global via the basis]
If $\rho$ witnesses coherence on the generators $\mathcal{E}$, then (by closure under composition)
$\rho$ extends to all intervals $[t,t']$ with $t\le t'$. Thus the basis choice fixes only the
\emph{editing tempo}, not the notion of name.
\end{proposition}

\section{Film-room intuition formalized}
A cut $(t\to t')$ is a splice. A pan is an adiabatic drift where $\Coh$ is carried by transport. A jump
cut is a rupture paired with an explicit \emph{healing} witness $\Heal$, which we model here simply
as an admissible $\Coh$ term. In all cases, the audience keeps track of the character because the
projectionist supplies $\rho$---the recursive reel of coherence proofs.

\section{Elementary constructions on names}
\paragraph{Restriction.}
Starting a little later forgets frames: there is a canonical map
$\Name(\A,\tau)\to \Name(\A,\tau')$ for $\tau\le \tau'$ given by tailing $\alpha$ and $\rho$.

\paragraph{Re-typing under soft change.}
If at some $t$ we have an internal equivalence $\A(t)\Equiv \B(t)$, we may \emph{retime} the name
without loss via the previous lemma; the film stock changes, the character persists.

\paragraph{Observation maps.}
Any fibrewise map $f:\A\to \B$ induces $f_{\*}:\Name(\A,\tau)\to \Name(\B,\tau)$ by post-composition
on trajectories and action on witnesses.

\section{Examples}
\subsection*{A lexical name in discourse}
Let $\A$ record the sense of a word across turns of dialogue. A trajectory chooses one occurrence per
turn; $\Rstar(\alpha)$ asserts that at every turn and step the use remains intelligible (allowing gentle
drift). Then $(\alpha,\rho)$ is the \emph{name} the conversation gives that word.

\subsection*{The projection-booth cut}
Let $\A$ be the “character state” type across edits. In an adiabatic montage, $\rho$ follows transports.
At a hard cut, a repair is inserted (new establishing shot); the $\Coh$-witness is a typed bridge from
pre-cut to post-cut state. The name is precisely what the editor preserves.

\section{What we defer (and why)}
We have studiously avoided the global machinery: non-equivalence change of contexts, Grothendieck
indexing, and the full definition of \emph{agents}. Those belong to Chapter~10, where names become
\emph{carriers} of generativity and worlds themselves acquire topology. Here we needed only this:
\emph{a name is a robust, corecursively witnessed trajectory}.

\section{Checklist of facts proved or required later}
\begin{itemize}[leftmargin=2em]
  \item \textbf{Equivalence invariance:} done above.
  \item \textbf{Basis independence:} sketched; full proof deferred to Chap.~10.
  \item \textbf{Closure under fibrewise maps:} immediate by functoriality.
  \item \textbf{Stability under guarded limits:} easy by coinduction (details deferred).
  \item \textbf{Interplay with rupture/heal rules:} admissible $\Coh$-witnesses suffice here; full
        syntax/semantics in Chap.~6.
\end{itemize}

\section*{One-line moral}
\emph{Name} $\;=\;$ trajectory $+$ a reel of coherence proofs that never runs out.

\end{document}
