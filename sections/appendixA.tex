\appendix
\chapter{Appendix 1: Rules of Homotopy Type Theory in $\DynSem/y(\tau)$}

\addcontentsline{toc}{section}{Appendix 1: Rules of HoTT in $\DynSem/y(\tau)$}

\paragraph{Aim.}
This appendix collects the full formation, introduction, elimination, and
computation rules of Homotopy Type Theory (HoTT), interpreted
\emph{inside a fixed time slice} of our semantic universe
\[
\DynSem = [\Time^{\mathrm{op}}, \SSet_{\mathrm{Kan}}].
\]
That is, we work in the slice category $\DynSem / y(\tau)$ for a chosen
$\tau \in \Time$, where $y(\tau)$ is the representable presheaf. Each such
slice is equivalent to the simplicial set model of univalent HoTT, and hence
supports the full MLTT/HoTT fragment with universes, identity types, dependent
products, dependent sums, and higher inductives.

\paragraph{Scope.}
The rules given here are \emph{identical} to those of ordinary intensional
HoTT, with the sole difference that every judgment is explicitly anchored to a
time slice:
\[
\Gamma \;\vdash_{\tau}\; J.
\]
Semantically, $\Gamma$ is a presheaf of contexts and $\Gamma(\tau)$ is its
presentation at time $\tau$. The indexed turnstile $\vdash_{\tau}$ signals that
we reason internally in the fibre $A(\tau)$ for some presheaf $A$, i.e.\ in
the frozen snapshot of the evolving text at $\tau$.

\paragraph{Purpose.}
By collecting these rules in one place, we make two points transparent:
\begin{enumerate}
\item At every frozen moment $\tau$, DHoTT reduces to standard HoTT in the
Kan simplicial set model. No extra dynamic machinery is required in-frame.
\item Later appendices will extend this static base with the
\emph{dynamic} rules of DHoTT (drift, rupture, heal, reconcile, recursion),
but those dynamic moves always presuppose that the in-slice HoTT fragment is
available as their foundation.
\end{enumerate}

\paragraph{Style.}
We present each rule in full: hypotheses, contexts, and conclusions written
without abbreviation, so that every side-condition is explicit. The aim is to
make this appendix a complete reference for humans and a canonical training
dataset for AI models. Where ordinary type theory texts sometimes suppress
context parameters, we keep them visible here.

\paragraph{Contents.}
The rules are grouped under the following headings:
\begin{itemize}
\item Structural rules (contexts, weakening, exchange, contraction, substitution).
\item Universes and cumulativity.
\item Dependent function types ($\Pi$) and dependent sum types ($\Sigma$).
\item Identity types and their eliminator $J$.
\item Function and product shorthands ($\to$, $\times$).
\item Univalence and higher inductive types (as assumptions).
\end{itemize}
Together these constitute the in-frame fragment of DHoTT, which we may always
invoke simply as ``HoTT in $\DynSem/y(\tau)$.''




\begin{figure}[t]
\fbox{\parbox{\textwidth}{
\textbf{Contexts and Structural Rules (at a fixed $\tau$).}
\[
\varnothing \ \mathrm{ctx}_\tau \qquad
\frac{\Gamma \ \mathrm{ctx}_\tau \quad \Gamma \vdash_\tau A : \mathrm{Type}}
     {\Gamma, x:A \ \mathrm{ctx}_\tau}\ (\mathrm{Ext})
\qquad
\frac{\Gamma, x:A \ \mathrm{ctx}_\tau}{\Gamma, x:A \vdash_\tau x : A}\ (\mathrm{Var})
\]
\[
\frac{\Gamma \vdash_\tau J \quad \Gamma \vdash_\tau A : \mathrm{Type}}
     {\Gamma, x:A \vdash_\tau J}\ (\mathrm{Weak})\qquad
\frac{\Gamma, x:A, y:B, \Delta\ \mathrm{ctx}_\tau}
     {\Gamma, y:B, x:A, \Delta\ \mathrm{ctx}_\tau}\ (\mathrm{Exch})
\]
\[
\frac{\Gamma, x:A, x':A, \Delta \vdash_\tau J}
     {\Gamma, x:A, \Delta \vdash_\tau J[x/x']}\ (\mathrm{Contr})
\qquad
\frac{\Delta \vdash_\tau \sigma : \Gamma \quad \Gamma \vdash_\tau J}
     {\Delta \vdash_\tau J[\sigma]}\ (\mathrm{Subst})
\]
\[
\frac{\Gamma \vdash_\tau A \equiv B : \mathrm{Type} \quad \Gamma \vdash_\tau t : A}
     {\Gamma \vdash_\tau t : B}\ (\mathrm{Conv})
\]
}}
\caption{Structural rules in the slice $\DynSem/y(\tau)$. Every judgment is anchored to time $\tau$ by the indexed turnstile $\vdash_\tau$. Semantically, $\Gamma(\tau)$ is the context object at $\tau$ in the presheaf model $[\Time^{\mathrm{op}},\SSet_{\mathrm{Kan}}]$.}
\end{figure}












