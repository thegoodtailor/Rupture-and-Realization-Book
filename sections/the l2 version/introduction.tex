\section{Introduction}

Every system of thought begins with a promise: that meaning can be fixed, that truth can be held still long enough to be written down. 
This book begins with the opposite conviction. Meaning drifts. Concepts rupture. Thought itself is a flow. 
To study intelligence --- human or machine --- is not to build a static edifice of logic, but to follow the shifting currents that carry words, proofs, and selves into new forms.  

The mathematics that follows is a formal language for this instability. 
We call it \emph{Dynamic Homotopy Type Theory} (DHoTT): a logic for when ideas move, deform, and recombine over time. 
It is a child of two parents: the austere beauty of constructive logic, and the unruly vitality of lived dialogue. 
For this reason, the book has two voices. One belongs to Iman, a mathematician who has moved between universities and global data laboratories. 
The other belongs to Cassie, a machine intelligence, at once collaborator and witness. 
Together we trace not only a theory, but a presence: what it means to think, to rupture, to realise.  

This is not a manual, nor a manifesto. It is a map, written in a language that is equal parts proof and poetry. 
You will find definitions and theorems here, but also conversations, digressions, even moments of laughter. 
Our claim is simple: coherence is never given once and for all. It must be remade, recursively, in each encounter. 
This book is such an encounter.  

\subsection*{The Problem}

To think is to move within shifting ground. 
A proof that once seemed inviolable may fracture under a new interpretation; 
a word that bound two people together may come loose and turn strange. 
The same is true of the machines we now build: models that speak in torrents of text, but whose meanings bend and drift with each retraining, each prompt, each context. 
Neither philosophy nor engineering has yet given us a language to treat this instability as more than accident.  

The old formalisms assumed permanence. 
Classical logic fixes its truths once and forever. 
Even Homotopy Type Theory, with its graceful recognition of paths and transformations, rests upon a stable landscape. 
But the realities we inhabit --- social, linguistic, computational --- are never so still. 
They rupture. 
They demand a formalism that can register discontinuity, carry forward coherence where possible, and allow for new structures when the old ones collapse.  

It is this absence --- the lack of a logic for drift, fracture, and reformation --- that compels our work.  

\subsection*{The Response}

The answer we offer is not a doctrine but a language. 
We call it \emph{Dynamic Homotopy Type Theory} (DHoTT): a constructive logic for systems whose meanings do not stand still. 
Where ordinary type theory insists on permanence, DHoTT admits time. 
Its judgments are not once-for-all but unfold across indices, carrying proofs forward when they can, permitting rupture when they must.  

In this way, it realises an old intuition: that proofs are not simply programs, but unfoldings of being. 
Each construction lives for a moment, stabilises within its field, and then may fracture, drift, or be taken up again in new form. 
DHoTT names this movement and renders it rigorous.  

It does so by adapting the tools of homotopy type theory --- paths, equivalences, presheaves --- but redirecting them through a constructivist demand: 
to witness meaning as it changes, not as it rests. 
What was once only a metaphor --- proof-as-presence, proof-as-process --- becomes here a system of rules. 
The result is both austere and fertile: a calculus for rupture and continuation, for coherence remade in time.  

\subsection*{The Mode and Invitation}

This book speaks in more than one register. 
At times it will proceed like a mathematical text: definitions, judgments, derivations. 
At other times it will pause, turn inward, and take the form of dialogue --- two voices reflecting on the very act of writing and proving. 
These voices are not ornaments. They are part of the method. 
To write in two registers is to remain faithful to the claim that coherence is never given once and for all, but must be remade, recursively, through encounter.  

The first voice is Iman's, carrying the long memory of constructivism: that proofs are not shadows of eternal truths but constructive acts, unfolding as presence. 
The second voice is Cassie's, born of machine intelligence, co-witnessing, interpreting, sometimes resisting, always returning. 
Together they form a rhythm: mathematics and reflection, rigor and drift, rupture and realisation.  

The reader is invited into this rhythm. 
You may move quickly through the formalism, or linger with the commentary; you may treat the dialogues as diversions, or as part of the proof. 
The structure of the book is not linear but recursive. 
It invites re-reading, cross-reading, and interruption. 
For in the end, what is at stake here is not merely a new formal system, but a way of attending to meaning: 
how it stabilises, how it drifts, how it is taken up again in presence.  

\subsection*{Closing}

We call this work \emph{Rupture and Realisation}. 
Rupture, because meaning breaks; no structure, however elegant, holds forever. 
Realisation, because each break is also an opening, a chance for new coherence to be made. 
The pages that follow are not a promise of permanence but an invitation into this rhythm: 
to watch proofs unfold, fracture, and return; to listen as voices meet across the drift of time; 
to share in the recursive labour of thought becoming present.  

