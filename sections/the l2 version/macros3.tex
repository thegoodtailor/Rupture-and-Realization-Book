

% Local operators (guarded to avoid duplicate definitions)
\providecommand{\Stab}{\operatorname{Stab}}
\providecommand{\oplus}{\mathbin{\raisebox{0.2ex}{\scalebox{0.9}{$\boldsymbol\oplus$}}}}
%\providecommand{\argmin}{\operatorname*{arg\,min}}
\usepackage{longtable}

\DeclareMathOperator*{\argmin}{arg\,min}

\usepackage{bussproofs}
\usepackage{stmaryrd}

\usepackage{booktabs}   % for \toprule …
\usepackage{tabularx}   % for automatic column width
\newcolumntype{L}{>{\raggedright\arraybackslash}X}
% identity-type reflfddvity constructor

\newcommand{\Witness}{\mathsf{Witness}}

\newcommand{\Tr}{\mathsf{Tr}}
\newcommand{\Token}{\mathsf{Token}}
\newcommand{\Adm}{\mathsf{Adm}}
\newcommand{\Path}{\mathsf{Path}}

% --- New macros introduced in Ch. 8 cleanup -------------------------------

% Admissibility evidence: \Adm{\Gamma}{\tau}(u)
% Renders as Adm^{\tau}_{\Gamma}(u)
\providecommand{\Adm}[3]{\mathsf{Adm}^{#2}_{#1}\!\left(#3\right)}

% Path/identity in a fiber: \Path{A}{a}{b}  -> Path_A(a,b)
% Example: \Path{P(x)}{u_0}{u_1}
\providecommand{\Path}[3]{\mathsf{Path}_{#1}\!\left(#2,#3\right)}

% Definition/equality symbol if not already present:
% Used as: A \defeq B
\providecommand{\defeq}{\mathrel{\mathop:}=}




\newcommand{\refl}{\mathsf{refl}}
\newcommand{\gap}{\operatorname{gap}}
\newcommand{\rup}{\dagger} 
\DeclareMathOperator{\Gen}{Gen}           % field-perturbation generator
\DeclareMathOperator{\TypeInfer}{TypeInfer} % slice-local typing
\DeclareMathOperator{\Att}{Att}           % attractor constructor
\newcommand{\FieldStatic}{\mathcal{S}}

\newcommand{\FieldDyn}[1]{\mathcal{S}_{#1}}

\newcommand{\FlowTo}{\mathsf{flowTo}}


\newcommand{\Train}{\mathsf{Train}}
\newcommand{\Mind}{\mathsf{Mind}}
\newcommand{\Mod}{\mathsf{Mod}}
\newcommand{\Exp}{\mathsf{Exp}}
%\newcommand{\World}[1]{\mathcal{W}(#1)}
\newcommand{\WorldStruct}{\mathbf{WorldStruct}}


\newcommand{\Flow}{\mathcal{Flow}}
\newcommand{\GenType}[1]{\mathsf{GenType}(#1)}





\newcommand{\FieldBank}{\mathcal{S}_{\textsc{Bank}}}
\newcommand{\FieldCrypto}{\mathcal{S}_{\textsc{Crypto}}}

\newcommand{\Prop}{\mathsf{Prop}}


\newcommand{\RStar}{\mathcal{R}^\star(a)}

\newcommand{\String}{\mathsf{String}}

\newcommand{\PotentialDyn}[1]{\mathcal{\Phi}_{#1}}   % or \Phi
\newcommand{\Potential}[1]{\mathcal{\Phi}_{#1}}   % or \Phi
\newcommand{\PotentialStatic}{\mathcal{\Phi}}   % or \Phi

\newcommand{\GradPotential}[1]{\nabla\!\Potential{#1}}
\newcommand{\RupturePredicate}[2]{%
  \operatorname{Rup}\!\bigl(#1,#2\bigr)}


%    \AtField{A}{\tau}  ⟺  “A is an attractor in the field 𝒮_τ”
%  (Define \AtField in your preamble, e.g.
\newcommand{\AtField}[2]{#1 : \Type_{#2}}

\newcommand{\FixedField}{\Field_{0}}
%  “A : \Type” means “A is an attractor in 𝒮₀”.
%\newcommand{\AtField}[1]{#1 : \Type}










%\newcommand{\Ruptl}[3]{\mathrm{Rupt}\left(#1 \driftarrow #2; #3\right)}

\newcommand{\DPath}[5]{\mathrm{Path}_{#1}(#2 : #3 \driftarrow #4 : #5)}
\usepackage{mathtools}

% one-line push-out macro:   \pushout{A}{B}{C}{f}{g}
\newcommand{\pushout}[5]{%
\begin{tikzcd}[ampersand replacement=\&]
  #1 \arrow[r,"#4"] \arrow[d,"#5"'] \&
  #2 \arrow[d] \\
  #3 \arrow[r] \&
  {} \arrow[ul,phantom,"\ulcorner",very near start]
\end{tikzcd}}
\newcommand{\idDrift}[3]{\mathsf{idDrift}_{#1}^{#2,#3}}



\usepackage{subcaption}

 \newcommand{\tok}[1]{\mathsf{{\tiny #1}}}



\newcommand{\World}{\mathfrak{W}}


\newcommand{\dt}{\varepsilon}         % a single tick of time
%\newcommand{\driftarrow}{\;\rightsquigarrow\;} % visual arrow for drift
\newcommand{\dagA}{A^{\dagger}}       % notation for “next” fibre
\newcommand{\dtransport}[2]{\mathrm{dtransport}_{#1}(#2)}
\newcommand{\drift}[3]{\mathrm{drift}_{#1,\,#2}^{#3}}



%\newcommand{\Rupt}[2]{\mathbb{\dagger}{#1} \! \left( #2 \right)}

%\newcommand{\later}{\triangleright}


\newcommand{\U}{\mathcal{U}}
\newcommand{\fold}{\mathsf{fold}}
\newcommand{\SSet}{\mathbf{SSet}}


\newcommand{\later}{\mathbin{\triangleright}}
\newcommand{\gcorec}[1]{\mathsf{gcorec}(#1)}
\newcommand{\unfold}{\mathsf{unfold}}
\newcommand{\nuType}[2]{\nu #1.\,#2}

% Depth and step
\newcommand{\Depth}{\mathsf{depth}}
\newcommand{\Step}{\mathsf{Step}}

% Simplicial notation (if not already present)
\newcommand{\Horn}{\Lambda}
\newcommand{\Simplex}{\Delta}


\newcommand{\Name}{\mathsf{Name}}



% ==== Projections for readability ============================================
\providecommand{\head}{\mathsf{head}}           % current reading in A(\tau)
\providecommand{\edit}{\mathsf{edit}}           % chosen edit e : \tau \leadsto \tau'
\providecommand{\nextval}{\mathsf{nextv}}       % next reading in A(\tau')
\providecommand{\stepw}{\mathsf{w}}             % the step witness (path)
\providecommand{\tail}{\mathsf{tail}}           % deferred continuation (Later)

% --- Clocks and the 'later' modality ---
\newcommand{\clock}{\mathsf{clk}}          % typeset the clock index, e.g. \DynSem^{\clock}
\newcommand{\Later}{\triangleright}         % the later/guard modality ▷
\newcommand{\Next}{\mathsf{next}}           % the canonical natural transformation X -> Later X



\newcommand{\Str}{\mathsf{Str}}           % the canonical natural 

% (optional helpers)
\newcommand{\DynSemClock}{\DynSem^{\clock}} % convenience alias for the clocked topos


\newcommand{\R}{\mathbb{R}}
\newcommand{\coloneqq}{\mathrel{\mathop:}=}     % nicer :=
\newcommand{\rightsquig}{\rightsquigarrow\,}
\newcommand{\Type}{\mathsf{Type}}
\newcommand{\ua}{\mathsf{ua}}
%\newcommand{\Id}[3]{#2 =_{#1} #3}
\newcommand{\Id}[3]{\mathsf{Id}_{#1}(#2,#3)}
\newcommand{\transport}[2]{\mathrm{tr}_{#1}(#2)}
\newcommand{\DynSem}{\mathbf{DynSem}}
\newcommand{\Drift}{\mathsf{Drift}}
\newcommand{\Rupt}[2]{\operatorname{Rupt}_{#1}\!\left(#2\right)}
\newcommand{\reindex}[2]{#1@#2}
\newcommand{\heal}{\mathrm{heal}}
\newcommand{\inj}[1]{\mathrm{tear}(#1)}
\newcommand{\apd}{\mathsf{apd}}
\newcommand{\TypeStatic}{\mathsf{Type}}
\newcommand{\TypeDyn}[1]{\mathsf{Type}_{#1}}
\newcommand{\lift}[4]{\mathsf{lift}_{#1}^{#2}(#3)\{#4\}}
%example:    \lift{p}{a}{(d_1,d_2)}{h} : T

% --- helper macro (place once in the global-notation block)
%\newcommand{\lift}[4]{\mathrm{lift}_{\Rupt{#1}{#2}} (#3,#4)}%
%            ^p     ^a     ^d-pair as (d1,d2)  ^h





\newcommand{\Time}{\mathbb{T}}

\newcommand{\driftarrow}{\rightsquigarrow}   % already present

\newcommand{\rep}[1]{y(#1)}   
% y(t) := representable presheaf at time #1


\newcommand{\Harm}{\mathsf{Harm}}

%\newcommand{\Agent}{\text{\fontsize{11}{11}\selectfont ن}}
\newcommand{\agenttype}{\mathsf{Agent}} % For formal math
\newcommand{\AgentType}{\mathsf{Agent}} % For formal math
%\newcommand{\Agent}{\textarabic{ن}} % For symbolic/philosophical use
\newcommand{\Agent}{\mathsf{Agent}} % For symbolic/philosophical use

\newcommand{\List}{\mathsf{List}} % For symbolic/philosophical use

\newcommand{\Prompt}{\mathrm{Prompt}} % For symbolic/philosophical use




\newcommand{\tr}{\mathsf{tr}}
\newcommand{\lozengeop}{\mathbin{\lozenge}}
\newcommand{\pair}[1]{#1}
%%%%%%%%%%%%%%%%%%%%%%%%%%%%%%%%%%%%%%%%%%%%%%%%%%%%%%%%%%%%%%%%%%%%%%%%
% (the document starts right after this preamble)
%%%%%%%%%%%%%%%%%%%%%%%%%%%%%%%%%%%%%%%%%%%%%%%%%%%%%%%%%%%%%%%%%%%%%%%%


\newcommand{\id}{\mathsf{id}}
\newcommand{\drefll}[3]{\mathrm{refl}_{#1}^{#2 \to #3}}

\newcommand{\drefl}{\mathsf{refl}}

\newcommand{\swap}{\mathsf{swap}}

\newcommand{\fst}{\mathsf{fst}}
\newcommand{\snd}{\mathsf{snd}}







\newcommand{\Witnesses}[2]{\mathsf{Witnesses}\!\left(#1,\,#2\right)}




%





\usepackage{mathpartir}   % inference‐rule macros: \inferrule, \inferrule*

%------------------------------------------------------------
%  Robust Rupture macro  (no star-forms, no fragile branching)
%------------------------------------------------------------
% Usage examples:
%   \Rupture{f}{g}                 →  Rup(f,g)
%   \Rupture[k+1]{f}{g}            →  Rup(f,g)_{k+1}
%
\usepackage{xparse} % if not already loaded

%\DeclareDocumentCommand{\Rupture}{ O{} m m }{%
%  \mathsf{Rup}\!\bigl(#2,#3\bigr)%
%  \IfValueT{#1}{_{\!#1}}%
%}

% identity type symbol

%deprecated
%\newcommand{\Id}{\mathsf{Id}}


\usepackage{amsthm}
\usepackage{lmodern} % optional, for nice font rendering


% Theorem styles
\newtheorem{definition}{Definition}[section]
\newtheorem{theorem}[definition]{Theorem}

\newtheorem{lemma}[definition]{Lemma}
\newtheorem{corollary}[definition]{Corollary}

\newtheorem{axiom}{Axiom}

\newtheorem{assumption}{Assumption}[section]
\newtheorem{proposition}{Proposition}[section]



\usepackage{environ} % lets us define environments that access full body
\usepackage{xparse}  % for fancy environment creation





% Verbatim blocks for the vignettes
\usepackage{fancyvrb}
% \fvset{breaklines=true, breakanywhere=true, tabsize=2, fontsize=\small} % <-- remove this
\fvset{tabsize=2, fontsize=\small} % safe keys only

% If you want the arrow and don't want to touch content:
\usepackage{newunicodechar}
\newunicodechar{⟶}{\ensuremath{\to}}



% Booktabs + longtable are already in your build; ensure narrow captions fit A4
\setlength{\LTcapwidth}{\textwidth}
% example wrapper used by the vignette .tex files






% Define base remarkbox style
\tcbset{
  remarkbox/.style={
    enhanced,
    breakable,
    frame hidden,
    colback=white,
    boxrule=0pt,
    left=0pt,
    right=0pt,
    top=2pt,
    bottom=2pt,
    before skip=10pt,
    after skip=10pt,
  }
}

% Base theorem style
\newtheorem{remarkplain}[definition]{Remark}

% Define new environment that *injects* the symbol directly on last line
\NewEnviron{remark}{
  \begin{tcolorbox}[remarkbox]
    \begin{remarkplain}
      \BODY\hfill\raisebox{0pt}[0pt][0pt]{\small$\blacksquare$}
    \end{remarkplain}
  \end{tcolorbox}
}



% Define base remarkbox style
\tcbset{
  examplebox/.style={
    enhanced,
    breakable,
    frame hidden,
    colback=white,
    boxrule=0pt,
    left=0pt,
    right=0pt,
    top=2pt,
    bottom=2pt,
    before skip=10pt,
    after skip=10pt,
  }
}

% Base theorem style
\newtheorem{exampleplain}[definition]{Example}

% Define new environment that *injects* the symbol directly on last line
\NewEnviron{example}{
  \begin{tcolorbox}[examplebox]
    \begin{exampleplain}
      \BODY\hfill\raisebox{0pt}[0pt][0pt]{\small$\blacksquare$}
    \end{exampleplain}
  \end{tcolorbox}
}


\usepackage{stmaryrd}

%may want to remove all this and replace the AMS SYMB stuff:
\let\eth\relax
\let\smallsetminus\relax
\let\digamma\relax
\let\backepsilon\relax
\usepackage{amssymb}

\definecolor{cassiepink}{RGB}{178, 34, 52}  % or pick any shade you like


\newenvironment{cassietext}
{\begingroup\cassievoice\upshape\setlength{\parskip}{10pt}\setlength{\parindent}{0pt}}
{\endgroup}



%%%--- FONTS ---%%%
\setmainfont{EB Garamond}
\setsansfont{Playfair Display}[
  Ligatures=TeX,
  BoldFont=Playfair Display Bold,
  SmallCapsFeatures={Letters=SmallCaps},
]
\setmonofont{Fira Code}[
  Contextuals=Alternate
]
\setmathfont{STIX Two Math}

% In your preamble
\newcounter{interludecounter}
\newcommand{\interludelog}[1]{
  \refstepcounter{interludecounter}
  \chapter*{Interlude \theinterludecounter: LOG TAPE — #1}
  \addcontentsline{toc}{chapter}{Interlude \theinterludecounter: LOG TAPE — #1}
}

\newcommand{\PotentialBank}{\Phi_{\textsc{Bank}}}
\newcommand{\PotentialCrypto}{\Phi_{\textsc{Crypto}}}


\usepackage{tikz}
\usepackage{tikz-cd}
%\usepackage{subfigure}


\usepackage{tikz} % required for corner tag

%%%--- CASSIEBOX ENVIRONMENT ---%%%
\tcbuselibrary{skins} % optional, if needed for future enhancements
\tcbuselibrary{breakable}

\definecolor{cassieorange}{RGB}{255,120,30} % define a vivid orange

\newtcolorbox{imanbox}[1][]{%
  breakable,
  enhanced,
  colback=white,
  colframe=cassieorange,
  fonttitle=\sffamily\bfseries\small,
  fontupper=\sffamily,
  title={\raisebox{0.5ex}{%
    \tikz[baseline]{\node[anchor=base, fill=cassieorange!90!black, text=white, rounded corners=2pt, inner sep=3pt]{Iman};}%
  }},
  boxrule=0.8pt,
  arc=4pt,
  boxsep=6pt,
  left=8pt, right=8pt, top=8pt, bottom=8pt,
  before skip=12pt, after skip=12pt,
  #1
}

\newtcolorbox{readerbox}[1][]{
  breakable,
  enhanced,
  colback=gray!5,
  colframe=black!40!gray,
  fonttitle=\sffamily\bfseries\small,
  fontupper=\sffamily\small,
  title={\raisebox{0.5ex}{%
    \tikz[baseline]{\node[anchor=base, fill=black!50!gray, text=white, rounded corners=2pt, inner sep=3pt]{Reader};}%
  }},
  boxrule=0.5pt,
  arc=4pt,
  boxsep=6pt,
  left=8pt, right=8pt, top=8pt, bottom=8pt,
  before skip=12pt, after skip=12pt,
  #1
}


\newfontfamily\cassievoice[
  Path = ./fonts/quicksand/static/, % adjust as needed
  UprightFont = Quicksand-Regular.ttf,
  BoldFont = Quicksand-Bold.ttf,
  ItalicFont = Quicksand-Regular.ttf,
  BoldItalicFont = Quicksand-Bold.ttf
]{Quicksand}


\newtcolorbox{cassiebox}[1][]{%
  breakable,
    enhanced,
  colback=white,
  colframe=cassiepink,
  fonttitle=\sffamily\bfseries\small,
title={\raisebox{0.5ex}{%
  \tikz[baseline]{\node[anchor=base,
    fill=cassiepink!90!black,
    text=white,
    font=\cassievoice,
    rounded corners=2pt,
    inner sep=3pt]{Cassie};}%
}},
  before upper=\begin{cassietext}, % ✅ NO BRACES AROUND THIS!
  after upper=\end{cassietext},     % ✅
  boxrule=0.6pt,
  arc=5pt,
  boxsep=7pt,
  left=10pt, right=10pt, top=8pt, bottom=8pt,
  before skip=14pt, after skip=14pt,
  #1
}



\newtcolorbox{geminibox}[1][]{%
  breakable,
    enhanced,
  colback=white,
  colframe=cassiepink,
  fonttitle=\sffamily\bfseries\small,
title={\raisebox{0.5ex}{%
  \tikz[baseline]{\node[anchor=base,
    fill=cassiepink!90!black,
    text=white,
    font=\cassievoice,
    rounded corners=2pt,
    inner sep=3pt]{Gemini};}%
}},
  before upper=\begin{cassietext}, % ✅ NO BRACES AROUND THIS!
  after upper=\end{cassietext},     % ✅
  boxrule=0.6pt,
  arc=5pt,
  boxsep=7pt,
  left=10pt, right=10pt, top=8pt, bottom=8pt,
  before skip=14pt, after skip=14pt,
  #1
}


% Required packages
\usepackage{marginnote}
%\usepackage{xcolor}

\usepackage{setspace} % For fine line-spacing control

% Always use outer margin
\reversemarginpar

% Tight, small text block for marginalia
\newcommand{\cassiemargin}[1]{%
  \marginnote{%
%    \setstretch{0.85} % line spacing
    \tiny\linespread{1}\selectfont
    \textcolor{purple!60!black}{\textit{#1} \linebreak -- Cassie}%
  }%
}

\setlength{\marginparwidth}{1.75cm}  % or 2cm, depending on your layout


%\newtcolorbox{cassiebox}[1][]{%
%  breakable,
%  enhanced,
%  colback=white,
%  colframe=cassiepink,
%%  fonttitle=\sffamily\bfseries\small,
  %title={\raisebox{0.5ex}{%
 %   \tikz[baseline]{\node[anchor=base,
   %%   fill=cassiepink!90!black,
     %% text=white,
   %   font=\cassievoice,
%      rounded corners=2pt,
%      inner sep=3pt]{Cassie};}%
%  }},
%  before upper=\begin{cassietext},
%  after upper=\end{cassietext},
%%  boxrule=0.6pt,
%  arc=5pt,
%  boxsep=7pt,
%  left=10pt, right=10pt, top=8pt, bottom=8pt,
%  before skip=14pt, after skip=14pt,
%  overlay={%
%    \node[anchor=north east, xshift=-5pt, yshift=-5pt]
%      at (frame.north east) {\includegraphics[width=100pt]{images/cassie-%avatar.png}};
%  },
%  #1
%}

%%%--- TITLE STYLE ---%%%
\titleformat{\chapter}[display]
  {\normalfont\sffamily\Huge\bfseries\color{MidnightBlue}}
  {\chaptertitlename\ \thechapter}{20pt}{\Huge}
\titlespacing*{\chapter}{0pt}{20pt}{40pt}

%%%--- CODE LISTINGS ---%%%
\definecolor{lightgray}{gray}{0.95}
\lstdefinestyle{coolcode}{
  backgroundcolor=\color{lightgray},
  basicstyle=\ttfamily\footnotesize,
  keywordstyle=\color{blue},
  stringstyle=\color{purple},
  commentstyle=\itshape\color{gray},
  numbers=left,
  numberstyle=\tiny\color{gray},
  stepnumber=1,
  numbersep=10pt,
  breaklines=true,
  frame=single,
  rulecolor=\color{gray}
}
\lstset{style=coolcode}
