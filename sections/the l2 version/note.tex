





























you have said recently:

  \item \textbf{Non‑equivalence topic switches.} Reindexing across a change of conversational domain
        (new scene) is not just transport; it needs a world/base change. We postpone the Grothendieck
        fibration and its Beck–Chevalley laws to Chapter~9.
        
        
        
        this wasn't originally how i understood rupture in dhott. i thought until a few days ago that new scenes/new contexts were possible, with rupture being captured ... and the possibility open for finding healing paths present ... simply because we have such constructs in dhott. i understand the need for worlds+base change in order to grasp AGENTIC trajectories (where a self is sort of a bundle of types that are inhabted but traverses many domains of discourse ...) ... but what happens to pooor of rupture in the light of your statement above on grothendieck? i think i'm missing something, you'll need to explain to me.