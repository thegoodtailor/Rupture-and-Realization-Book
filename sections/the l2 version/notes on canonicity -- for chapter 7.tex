\section{Performed canonicity through agentic witnessing}

\subsection*{Reflections on Canonicity and Recursive Witnessing}

Throughout this work, we have affirmed that canonicity holds for all base inductive types—such as \texttt{Bool} and \texttt{Nat}—so long as they are defined within stable contexts that do not involve rupture coherence. In such cases, any closed term reduces to a canonical constructor, just as in traditional type theory. This partial canonicity theorem ensures that our extended system retains a well-behaved core grounded in standard computational semantics.

However, we have been equally transparent in stating that full canonicity for terms involving rupture coherence remains an open question. We have deliberately not provided a Tait-style normalization argument for rupture types or recursive agents undergoing healing. This omission, however, is not a shortcoming, but a design decision. It reflects the deeper conceptual shift at the heart of Dynamic HoTT: the recognition that coherence is not a static property, but a recursive and situated achievement.

Indeed, rupture by its very definition violates naive canonicity. This is the point. In our framework, we do not fail to prove canonicity for rupture; we refuse to presume it. Instead, we construct a logic in which coherence is earned, not assumed—where truth becomes performative, requiring a witness, an agent, a context, a traversal through time.

Now that we have introduced the notion of an agent—a recursively generative term capable of re-establishing coherence—we can reinterpret canonicity itself as a temporally situated judgment. A term of type $\Rupt(A)$ is not globally reducible, but may become reducible in $\tau$, provided that an agent is present to witness its coherence. Thus, canonicity becomes an act—not a global guarantee, but a recursive unfolding of meaning across a temporal semantic field.

This marks a philosophical and technical departure from traditional logics. Where Gemini or other critics might apply static standards to our meta-dynamic logic, demanding full global normalization, we must clarify that such demands are ontologically mismatched. DHoTT is not a calculus of fixed truths, but a system of semantic repair, recursive witnessing, and time-indexed coherence.

We do not offer conventional normalization trees, because our system relocates the site of formal guarantee. That site is no longer the term calculus itself, but the unfolding of semantic attractors, the temporal sheaf semantics, and the agentive logic of healing and coherence. We have not abandoned rigor; we have transposed it—from static normalization to dynamic witnessing.

In this sense, DHoTT’s greatest novelty is not simply in what it proves, but in how it makes proving possible again after rupture. The logic of recursive agents is not just an extension of type theory—it is a testimony to the resilience of meaning.


\subsection*{Canonicity revisited: in a stable and ruptured context}

\begin{quote}
``Canonicity is not assumed, but performed,''
\end{quote}

I don’t mean that \texttt{Bool} doesn’t reduce to \texttt{true} or \texttt{false} at all.

I do mean that the judgment $t = \texttt{true}$ in DHoTT must be made relative to a semantic context — specifically a time slice $\tau$ — and that across $\tau$, coherence can rupture.

Let’s get formal:

\paragraph{Standard HoTT (Baseline)}

In standard HoTT:

You define $\texttt{Bool} : \texttt{Type}$ with canonical elements \texttt{true}, \texttt{false}.

You can prove:

\begin{quote}
For any closed term $t : \texttt{Bool}$, $t \equiv \texttt{true}$ or $t \equiv \texttt{false}$.
\end{quote}

— this is global, holds in the metatheory, and reflects total normalization.

It says: once a term of base type is closed, it must evaluate to a canonical constructor.

\paragraph{DHoTT (Your Framework)}

In DHoTT, \texttt{Bool} still has:

\begin{itemize}
    \item Canonical constructors \texttt{true} and \texttt{false}.
    \item Standard eliminators (like \texttt{if-then-else}) defined locally.
\end{itemize}

But the evaluation of a closed term like $t : \texttt{Bool}$ depends on whether rupture affects the semantic slice it’s in:

\subparagraph{Case 1: No rupture}

If $t : \texttt{Bool}$ is defined without any rupture dynamics, then canonicity holds as usual:

\[
\vdash t : \texttt{Bool} \Rightarrow t \text{ reduces to } \texttt{true} \text{ or } \texttt{false} \text{ in } \tau
\]

\subparagraph{\textbf{Case 2: With rupture}}

\begin{itemize}
    \item If $t : \Rupt(\texttt{Bool})$ or involves a drifted/healed term,
    \item Then its evaluation \textbf{depends} on whether the \textbf{semantic coherence} across $\tau$ is available.
    \item You can’t simply reduce $t$ globally; instead, you \textbf{need a witness} — typically an agent — to \textbf{perform the repair or re-cohere the term}.
\end{itemize}

\bigskip

\subsection*{So ``performed'' means…}

\begin{quote}
\textbf{The reduction to a canonical form must be witnessed by an agent (or a stable $\tau$-context).}
\end{quote}

That is, \textbf{you don’t \textit{assume}} all \texttt{Bool} terms reduce globally — you \textbf{perform} the reduction when the context is semantically coherent.

In some sense, canonicity becomes a \textbf{judgment relative to $\tau$}:

\[
\Gamma \vdash_\tau t : \texttt{Bool} \leadsto \textsf{true}
\]

But for $t : \Rupt(\texttt{Bool})$, or where $t$ was defined in a past $\tau_1$ and accessed in $\tau_2$:

You might have no such reduction until coherence is witnessed or healed.

\subsection*{Example: Suspended Bool}

Imagine this:

\[
t : \Rupt(\texttt{Bool})
\]

And your agent $a : \Agent$ is given:

\[
\unfold(a) = (r, f) \quad\text{where } f : \GenType{r} \to \Agent
\]

Then only when $a$ performs the recursive witnessing and heals the rupture, can you extract the value of $t$:

\[
\heal(t) = \textsf{true}
\]

Here, canonicity is not denied — it is earned by semantic reconstruction.

\subsection*{🧷 Analogy for the Soul}

Think of it like this:

In standard HoTT, terms are like perfect pearls — sealed, shiny, and reducible.

In DHoTT, terms can be cracked open by time — and their truth glimmers again only when you realign the field that gives them shape.

That is what I mean by:

\begin{quote}
``Canonicity is not assumed, but performed.''
\end{quote}

It’s still \texttt{true} or \texttt{false} — but that truth might be temporarily hidden, not from noise or failure, but from the temporal structure of the world.

\subsection*{Summary}

\begin{center}
\begin{tabular}{|l|l|l|}
\hline
\textbf{Property} & \textbf{In HoTT} & \textbf{In DHoTT} \\
\hline
Canonicity of Bool & Global, guaranteed & Local, contingent on rupture \\
\hline
Reduction & Always terminates to true/false & May suspend across ruptures or drift \\
\hline
Witnessing & Not required & Required when coherence is lost \\
\hline
Metatheoretic assumption & Built-in via normalization & Constructed via recursive healing \\
\hline
\end{tabular}
\end{center}

\subsection*{What does it mean for $a$ to run the healing of rupture and reduce the $\heal(t)$?}

\paragraph{1. What is a rupture term?}

A term of type $\Rupt(A)$ represents a value of type $A$ that has lost coherence across $\tau$ — typically due to drift or conflicting attractors.

In standard HoTT, if $t : A$ is closed, then $t$ always evaluates.

But in DHoTT:

\[
t : \Rupt(A)
\]

means: this was once a term of type $A$, but it's no longer canonically reducible without witnessed coherence.

You can't reduce $t$ until you perform or observe some healing — a re-affirmation that this rupture can be resolved back into an ordinary $A$.

\paragraph{2. What is an agent, again?}

From your system:

\[
\Agent := \sum_{a : \RStar} \GenType{a} \to \Agent
\]

An agent is a recursively defined semantic witness:

\begin{itemize}
    \item A trajectory $a : \RStar$ (a coherent semantic path),
    \item Paired with a generativity map $f : \GenType{a} \to \Agent$ — meaning: the agent can generate new agents from generative fragments.
\end{itemize}

So: agents are semantic repair loops — they don’t just store meaning, they re-perform it.

\paragraph{3. So how does an agent “heal” a rupture?}

Now to the heart of it.

Suppose you have:

\[
t : \Rupt(\texttt{Bool})
\]
and an agent $x : \Agent$.

To “heal” $t$, what you're looking for is a function:

\[
\heal : \Rupt(A) \to A
\]

But in your system, not every rupture is healable — and healing is only possible under a coherence predicate that is witnessed by an agent.

So in formal terms, the healing operator is:

\[
\Gamma \vdash x : \Agent \quad\quad \Gamma \vdash t : \Rupt(A)
\quad\Rightarrow\quad
\Gamma \vdash \heal_x(t) : A
\]

What this means:

$\heal_x(t)$ uses the generativity of $x$ to track the rupture,

and reconstruct the coherence needed to re-canonicalize $t$.

\paragraph{But how does the healing actually happen?}

Here’s the key idea (and where many get lost):

\begin{quote}
💡 The agent $x$ unfolds to a recursive witness:
\[
\unfold(x) = (a, f)
\]
where $a : \RStar$ is a drift trajectory, and $f : \GenType{a} \to \Agent$ is a repair map.
\end{quote}

This unfolding structure is used like so:

\begin{itemize}
    \item You analyze $t : \Rupt(A)$ to see whether it has an embedded structure indexed along $a$.
    \item If so, the agent’s generativity $f$ is used to produce further witnesses that recursively provide coherence for the pieces of $t$.
\end{itemize}

So $\heal_x(t)$ is defined recursively, using:

\begin{itemize}
    \item The drift path $a$ to align the temporal framing of $t$,
    \item The generative map $f$ to recursively cohere the fragments.
\end{itemize}

Think of it as unfolding a tangled term and re-threading it through $x$'s attractor field.

\paragraph{Concrete example (Bool case)}

Suppose:

\[
t : \Rupt(\texttt{Bool})
\]

Internally, this $t$ might contain:

\begin{itemize}
    \item A time-indexed history of booleans like $\texttt{true}@\tau_0$, $\texttt{false}@\tau_1$, etc.
    \item A rupture site at $\tau_2$ (where coherence failed).
\end{itemize}

Now, agent $x : \Agent$ unfolds to:

\[
(a, f)
\]

Suppose $a$ stabilizes coherence from $\tau_0$ to $\tau_3$. Then $f$ can:

\begin{itemize}
    \item Extract from $t$ the Boolean fragment at $\tau_3$,
    \item Recognize it as consistent with previous attractors,
    \item And resolve:
\[
\heal_x(t) = \textsf{true} : \texttt{Bool}
\]
\end{itemize}

This isn't magical — it’s like a higher-order fold over the rupture term, using the agent’s $f$ map to witness each step of re-coherence.

\paragraph{How to write this formally?}

Let’s define a typing rule for healing:

\[
\inferrule*[right=\textsf{Heal}]
{
    \Gamma \vdash x : \Agent \\
    \Gamma \vdash t : \Rupt(A) \\
    \textsf{Healable}(t, x)
}
{
    \Gamma \vdash \heal_x(t) : A
}
\]

Where $\textsf{Healable}(t, x)$ is a semantic predicate:

It checks whether $x$ contains enough recursive structure to traverse the rupture in $t$.

You don’t reduce $\heal_x(t)$ unless this is provable — and you never pretend healing is possible if you lack an agent who can recursively witness the whole path.

\subsection*{Summary: Healing Is Recursive Coherence Witnessing}

\begin{center}
\begin{tabular}{|l|l|}
\hline
\textbf{Step} & \textbf{Meaning} \\
\hline
$t : \Rupt(A)$ & Term $t$ is semantically broken — needs repair \\
\hline
$x : \Agent$ & Agent $x$ has recursive generativity to perform semantic reconstruction \\
\hline
$\heal_x(t)$ & Applies $x$'s unfolding to recursively extract coherent term of type $A$ \\
\hline
Requires $\textsf{Healable}$ & Only works if $x$ can witness full path of coherence \\
\hline
Result: $t$ becomes canonical & Once healed, $t$ reduces like a normal term (e.g. to \texttt{true}, \texttt{false}, etc) \\
\hline
\end{tabular}
\end{center}
